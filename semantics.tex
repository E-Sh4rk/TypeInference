% Header
\documentclass[a4paper]{article}%      autres choix : book, report
\usepackage[utf8]{inputenc}%           gestion des accents (source)
\usepackage[T1]{fontenc}%              gestion des accents (PDF)
\usepackage[francais]{babel}%          gestion du francais
\usepackage{textcomp}%                 caracteres additionnels
\usepackage{mathtools,amssymb,amsthm}% packages de l'AMS + mathtools
\usepackage{lmodern}%                  police de caractere
\usepackage[top=2cm,left=2cm,right=2cm,bottom=2cm]{geometry}%     gestion des marges
\usepackage{graphicx}%                 gestion des images
\usepackage{array}%                    gestion amelioree des tableaux
\usepackage{calc}%                     syntaxe naturelle pour les calculs
\usepackage{titlesec}%                 pour les sections
\usepackage{titletoc}%                 pour la table des matieres
\usepackage{fancyhdr}%                 pour les en-tetes
\usepackage{titling}%                  pour le titre
\usepackage{enumitem}%                 pour les listes numerotees
\usepackage{hyperref}%                 gestion des hyperliens
\usepackage{syntax}
\usepackage{minted}
\usepackage[parfill]{parskip}
\usepackage{amsmath}
\usepackage{fourier}
\usepackage{heuristica}
\usepackage[overload]{empheq}
\usepackage{cleveref}
\usepackage{alltt}
\usepackage{bm}
\usepackage{nicefrac}
\usepackage{stmaryrd}
\usepackage{mathpartir}
\usepackage{color}
\usemintedstyle{vs}

\hypersetup{pdfstartview=XYZ}%         zoom par defaut

\setlength{\droptitle}{-5em}   % This is your set screw
\title{\vspace{1.5cm}Type inference - M2 Internship}
\author{Mickael LAURENT}
\date{\vspace{-5ex}}

\pagenumbering{gobble}
\DeclarePairedDelimiter\ceil{\lceil}{\rceil}
\DeclarePairedDelimiter\floor{\lfloor}{\rfloor}

\DeclareFontFamily{U}{mathb}{}
\DeclareFontShape{U}{mathb}{m}{n}{
  <-5.5> mathb5
  <5.5-6.5> mathb6
  <6.5-7.5> mathb7
  <7.5-8.5> mathb8
  <8.5-9.5> mathb9
  <9.5-11.5> mathb10
  <11.5-> mathb12
}{}
\DeclareSymbolFont{mathb}{U}{mathb}{m}{n}
\DeclareMathSymbol{\sqdot}{\mathbin}{mathb}{"0D}
\newcommand{\worra}[2]{#1\mathop{\,\sqdot\,} #2}
\newcommand{\apply}[2]{#1\circ#2}
\newcommand{\dom}[1]{\textsf{dom}(#1)}
\newcommand{\alt}{~|~}
\newcommand{\Empty} {\textsf{Empty}}%\MyMathBb{0}}
\newcommand{\Any} {\textsf{Any}}%\MyMathBb{1}}
\newcommand{\Int} {\textsf{Int}}
\newcommand{\Bool} {\textsf{Bool}}
\newcommand{\True} {\textsf{True}}
\newcommand{\False} {\textsf{False}}
\newcommand{\true} {\textsf{true}}
\newcommand{\false} {\textsf{false}}
\newcommand{\Even} {\textsf{Even}}
\newcommand{\Odd} {\textsf{Odd}}
\newcommand{\ite}[4]{\ensuremath{\texttt{if}\;#1\in#2\;\texttt{then}\;#3\;\texttt{else}\;#4}}
\newcommand{\basic}[1]{\text{\fontshape{\itdefault}\fontseries{\bfdefault}\selectfont b}_{#1}}
\newcommand{\subst}[2]{\{#1 \mapsto #2\}}
\newcommand{\tyof}[2]{\textsf{typeof}_{#2}(#1)}
\newcommand{\typep}[3]{\textsf{Constr}^{#1}_{#3}(#2)}
\newcommand{\Gp}[2]{\textsf{Env}^{#1}_{#2}}

\theoremstyle{definition}
\newtheorem{theorem}{Theorem}
\newtheorem{lemma}{Lemma}
\newtheorem{definition}{Definition}
\newtheorem{corollary}{Corollary}

\newcommand {\Infer} [4] [] {
  \inferrule*[%
    left={\ifx\\#1\\#1\else\ifx\\#2\\[\textsc{#1}]\:\else[\textsc{#1}]\fi\fi},%
    right={$ #4 $}%
  ]%
  {#2}{#3}}

\newcommand {\Rule}[1] {[\textsc{#1}]}

\begin{document}

	\maketitle

    \section{Semantics}

    \subsection{Semantics}

    \[
    \begin{array}{lrcl}
    \textbf{Expressions} & e & ::= & c\alt x \alt \lambda^tx.e \alt \ite e t e e \alt e e\\
    \textbf{Values} & v & ::= & c \alt \lambda^tx.e
    \end{array}
    \]

    \[
    \begin{array}{lrcl}
    \textbf{Context} & C[] & ::= & e [] \alt [] v \alt \ite {[]} t e e
    \end{array}
    \]

    \begin{mathpar}
        \Infer[Ctx]
        {e \leadsto e'}
        {C[e] \leadsto C[e']}
        {}\\
        \Infer[App]
        {}
        {(\lambda^tx.e) v \leadsto e\subst x v}
        {}\\
        \Infer[Ite]
        {i \in \{1,2\}\hspace{1cm} (i=1 \Leftrightarrow v \in t)}
        {\ite v t {e_1} {e_2} \leadsto e_i}
        {}\\
    \end{mathpar}

    \begin{align*}
      n &\in \{n\}\\
      \true &\in \True\\
      \false &\in \False\\
      \lambda^tx.e &\in t\\
      v \in t \text{ and } t \leq t' \Rightarrow v &\in t'
    \end{align*}

    \subsection{Recall: Type System}

    \begin{mathpar}

      \Infer[Empty]
        { }
        {\Gamma\vdash e:\Empty}
        {\exists e' \in \dom\Gamma.\ \Gamma(e') \leq \Empty}
      \\
      \Infer[Occ]
          {
          }
          { \Gamma \vdash e: \Gamma(e) }
          { e\in\dom\Gamma}
      \qquad
      \Infer[Const]
          { }
          {\Gamma\vdash c:\basic{c}}
          {c\not\in\dom\Gamma}
       \\
      \Infer[Abs]
          {\Gamma,x:s_i\vdash e:t_i'\\t_i'\leq t_i}
          {
          \Gamma\vdash\lambda^{\wedge_{i\in I}s_i\to t_i}x.e:\textstyle\bigwedge_{i\in I}s_i\to t_i
          }
          {\lambda^{\wedge_{i\in I}s_i\to t_i}x.e\not\in\dom\Gamma}
          \\
      \Infer[App]
          {
            \Gamma \vdash e_1: t_1 \\
            \Gamma\vdash e_2: t_2\\
            t_1 \leq \Empty \to \Any\\
            t_2 \leq \dom {t_1}
          }
          { \Gamma \vdash {e_1}{e_2}: t_1 \circ t_2 }
          { {e_1}{e_2}\not\in\dom\Gamma}
          \\
      \Infer[If]
            {\Gamma\vdash e:t_\circ\\
              \Gamma, \Gamma^+_{\Gamma,e,t}\vdash e_1 : t_1\\
              \Gamma, \Gamma^-_{\Gamma,e,t}\vdash e_2 : t_2}
            {\Gamma\vdash \ite {e} t {e_1}{e_2}: t_1\vee t_2}
            {\makebox[3cm][l]{$\ite {e} t {e_1}{e_2}\not\in\dom\Gamma$}}
      \\\end{mathpar}

    \subsection{Definitions}

  \begin{definition}[Environment equivalence]
    Two environments $\Gamma$ and $\Gamma'$ are equivalent (noted $\Gamma \equiv \Gamma'$) iff:
    \begin{align*} 
      &\forall e \in \dom \Gamma.\ \Gamma(e)=t \Rightarrow \Gamma' \vdash e:t \hspace{1cm} \text{and} \hspace{1cm}
      \forall e' \in \dom {\Gamma'}.\ \Gamma(e')=t' \Rightarrow \Gamma \vdash e':t' \\
      \text{and} \hspace{1cm} &\exists e\in \dom \Gamma.\ \Gamma(e)=\Empty \Leftrightarrow \exists e'\in \dom {\Gamma'}.\ \Gamma'(e')=\Empty
    \end{align*}
  \end{definition}

  \begin{definition}[Environment inclusion]
    Let $\Gamma$ and $\Gamma'$ two environments. We says that $\Gamma' \leq \Gamma$ iff:
    \begin{align*}
      &\exists e\in \dom \Gamma.\ \Gamma(e)=\Empty \Rightarrow \exists e'\in \dom {\Gamma'}.\ \Gamma'(e')=\Empty\\
      &\forall e \in \dom \Gamma.\ \Gamma' \vdash e : t \text{ with } t \leq \Gamma(e)
    \end{align*}
    Note: this relation is a preorder (the transitivity can easily be deduced from the monotonicity lemma).
    We also have: $\Gamma' \leq \Gamma \text{ and } \Gamma \leq \Gamma' \Leftrightarrow \Gamma \equiv \Gamma'$.
  \end{definition}

  \begin{definition}[Environment substitution]
    Let $\Gamma$ an environment and $\rho$ a substitution from variables to expressions.
    The environment $\Gamma\rho$ is defined by:
    \begin{align*}
      &\dom {\Gamma\rho} = \dom \Gamma \rho\\
      &\forall e' \in \dom {\Gamma\rho}, (\Gamma\rho)(e') = \bigwedge_{e'' \in \dom \Gamma \text{ s.t. } e''\rho=e'}\Gamma(e'')
    \end{align*}
  \end{definition}

  \begin{definition}[Ordinary environments]
    We say that an environment $\Gamma$ is ordinary iff its domain only contains variables.
  \end{definition}

    \subsection{Theorems}

    \begin{lemma}[Monotonicity]
      Let $\Gamma$ and $\Gamma'$ two environments such that $\Gamma' \leq \Gamma$. Let $e$ an expression such that $\Gamma \vdash e:t$.
      Then:\\
      \begin{align*}
        &\Gamma' \vdash e:t' \text{ with } t' \leq t\\
        &\forall p \forall t' \forall \varpi.\ \Gp p {\Gamma',e,t'} (\varpi) \leq \Gp p {\Gamma,e,t'} (\varpi)
      \end{align*}
    \end{lemma}

    \begin{corollary}[Equivalence]
      If $\Gamma \vdash e:t$ and $\Gamma' \equiv \Gamma$, then $\Gamma' \vdash e:t$.
    \end{corollary}

    \begin{corollary}[Strengthening]
      If $\Gamma, (x:t_1^x) \vdash e:t_1$ and $\Gamma, (x:t_2^x) \vdash e:t_2$, then
      $\Gamma, (x:t_1^x\land t_2^x) \vdash e:t$ with $t \leq t_1\land t_2$.
    \end{corollary}

    \begin{lemma}[Substitution]
      Let $\Gamma$ an environment, $x$ a variable, $e$ an expression, $e'$ an expression that does not contain $x$, $t$ and $t_x$ two types.
      If $\Gamma \vdash e:t$ and $\Gamma(x) = t_x$ and $\Gamma \vdash e':t_{e'}$ with $t_{e'} \leq t_x$, then:
      \begin{align*}
        &\Gamma \subst x {e'} \vdash e \subst x {e'}:t' \text{ with } t'\leq t\\
        &\forall p \forall t' \forall \varpi.\ \Gp p {\Gamma\subst x {e'},e\subst x {e'},t'} (\varpi) \leq \Gp p {\Gamma,e,t'} (\varpi)
      \end{align*}
    \end{lemma}

    \begin{corollary}[Substitution 2]
      Let $\Gamma$ an environment, $x$ a variable that does not appear in $\dom \Gamma$, $e$ an expression, $e'$ an expression that does not contain $x$, $t$ a type and $t_x$ a type that is not $\Empty$.\\
      If $\Gamma, (x:t_x) \vdash e:t$ and $\Gamma \vdash e':t_{e'}$ with $t_{e'} \leq t_x$, then $\Gamma \vdash e \subst x {e'}:t'$ with $t'\leq t$.
    \end{corollary}

    \begin{lemma}[Value test]
      If $v \in t$ and $\Gamma \vdash v : t'$ with $\Gamma$ ordinary, then $t' \leq t$.
    \end{lemma}

    \begin{lemma}[Test monotonicity]
      Let $\Gamma$ an environment. Let $e$ a well typed expression and $t$, $t_{if}$, $t_{if}'$ some types.
      If $\Gamma \vdash e : t$ and $t_{if}' \leq t_{if}$ then $\Gamma^+_{\Gamma,e,t_{if}'} \leq \Gamma^+_{\Gamma,e,t_{if}}$
      and $\Gamma^-_{\Gamma,e,t_{if}} \leq \Gamma^-_{\Gamma,e,t_{if}'}$.
    \end{lemma}

    \begin{lemma}[Test reduction]
      Let $\Gamma$ an ordinary environment, $e$ and $e'$ two expressions and $t$, $t_{if}$ two types.
      If $\Gamma \vdash e : t$, $e \leadsto e'$ and
      if the subject reduction lemma holds for every subexpression of $e$ that satisfies its hypotheses,
      then $\forall p.\ \Gamma^p_{\Gamma,e',t_{if}} \leq \Gamma^p_{\Gamma,e,t_{if}}$.
    \end{lemma}

    \begin{theorem}[Subject reduction]
      Let $\Gamma$ an ordinary environment, $e$ and $e'$ two expressions and $t$ a type.
      If $\Gamma \vdash e : t$ and $e \leadsto e'$, then there exists $t' \leq t$ such that $\Gamma \vdash e' : t'$.
    \end{theorem}

    \section{Proofs}

    \subsection{Monotonicity}

    We proceed by induction on $e$. We suppose that, for any $e'$ strict subexpression of $e$, for any $\Gamma' \leq \Gamma$,
    \begin{align*}
      &\Gamma \vdash e':t' \Rightarrow \Gamma' \vdash e':t'' \text{ with } t'' \leq t'\\
      &\Gamma \vdash e':t' \Rightarrow \forall p \forall t'' \forall \varpi.\ \Gp p {\Gamma',e',t''} (\varpi) \leq \Gp p {\Gamma,e',t''} (\varpi)
    \end{align*}

    Let $\Gamma$, $\Gamma'$ and $e$ as in the lemma statement.
    If $\exists e' \in \dom {\Gamma'}.\ \Gamma'(e') = \Empty$, then the properties are trivial.
    So let's suppose it is not the case (in particular, we know that the last rule applied to type $\Gamma \vdash e:t$ can't be $\Rule{Empty}$).

    First, let's show that $\Gamma' \vdash e:t' \text{ with } t' \leq t$.

    If $e\in\dom\Gamma$, we know that $\Gamma(e)=t$ (as $\Gamma \vdash e:t$). The definition of $\Gamma' \leq \Gamma$ suffices to conclude.
    So let's suppose $e\not\in\dom\Gamma$ (in particular, we know that the last rule applied to type $\Gamma \vdash e:t$ can't be $\Rule{Occ}$).

    We now proceed by case analysis on $e$:
    \begin{description}
      \item[$c$] Trivial.
      \item[$x$] Impossible case, as we have already treated the rules $\Rule{Empty}$ and $\Rule{Occ}$.
      \item[$\lambda^{\bigwedge_{i\in I} s_i \rightarrow t_i}x.e_x$] We know that the last rule in the typing derivation of $\Gamma \vdash e:t$ is $\Rule {Abs}$.
      We can conclude directly by noticing that $\forall i.\ \Gamma',x:s_i \leq \Gamma,x:s_i$ and applying the induction hypothesis.
      \item[$e_1\ e_2$] We know that the last rule in the typing derivation of $\Gamma \vdash e:t$ is $\Rule {App}$.
      We can conclude directly by applying the induction hypothesis and noticing that the $\circ$ operator is monotonic.
      \item[$\ite {e_0} {t_{if}} {e_1} {e_2}$] We know that the last rule in the typing derivation of $\Gamma \vdash e:t$ is $\Rule {If}$.
      Using the induction hypothesis, we know that $\Gamma^p_{\Gamma',e_0,t_{if}} \leq \Gamma^p_{\Gamma,e_0,t_{if}}$ (it follows from the monotonicity of $\Gp {} {}$).
      Thus, we can conclude by applying the induction hypothesis again.\\
    \end{description}

    Now, let's show that $\forall p \forall t' \forall \varpi.\ \Gp p {\Gamma',e,t'} (\varpi) \leq \Gp p {\Gamma,e,t'} (\varpi)$.

    Let $p\in \{+,-\}$ and $t'$ a type.
    We proceed by induction on $\varpi$.
    
    The base case is straightforward (as we already proved that $\Gamma' \vdash e:t' \text{ with } t' \leq t$).\\
    For the case $\varpi.1$, we apply the outer induction hypothesis to get $\tyof {\Gamma'} {e\downarrow\varpi.0} \leq \tyof {\Gamma} {e\downarrow\varpi.0}$.
    We apply the induction hypothesis to get $\Gp p {\Gamma',e,t'} (\varpi) \leq \Gp p {\Gamma,e,t'} (\varpi)$. We can then conclude using the monotonicity of $\worra {} {}$.\\
    The case $\varpi.0$ follows from the previous case and the induction hypothesis.

    \qed

    \subsection{Substitution lemma}

    Let $x,e'$ as in the lemma statement. We will denote the substitution $\subst x {e'}$ by $\rho$.

    We proceed by induction on $e$. We suppose that, for any $e_s$ strict subexpression of $e$,
    \begin{align*}
      \forall \Gamma \forall t_s.\ &\Gamma \vdash e_s:t_s \Rightarrow \exists t'.\ \Gamma \rho \vdash e_s \rho:t' \text{ with } t'\leq t_s\\
      \forall \Gamma \forall t_s.\ &\Gamma \vdash e_s:t_s \Rightarrow \forall p \forall t' \forall \varpi.\ \Gp p {\Gamma\rho,e_s\rho,t'} (\varpi) \leq \Gp p {\Gamma,e_s,t'} (\varpi)
    \end{align*}

    Let $\Gamma,t,t_x,t_{e'}$ as in the lemma statement.

    If $\exists e_\bot \in \dom {\Gamma}.\ \Gamma(e_\bot) = \Empty$, then $(\Gamma\rho)(e_\bot\rho) = \Empty$ and thus the properties are trivial.
    So let's suppose it is not the case (in particular, we know that the last rule applied to type $\Gamma \vdash e:t$ can't be $\Rule{Empty}$).

    First, let's show that $\Gamma \rho \vdash e \rho:t'$ with $t'\leq t$.

    If $e\in\dom\Gamma$, we know that $\Gamma(e)=t$ (as $\Gamma \vdash e:t$). Thus, $(\Gamma\rho)(e\rho)=\bigwedge_{e'' \in \dom \Gamma \text{ s.t. } e''\rho=e\rho} \Gamma(e'') \leq \Gamma(e) \leq t$.
    So we can derive the wanted result using the rule $\Rule {Occ}$. It concludes this case. So let's suppose now that $e\not\in\dom\Gamma$ (in particular, we know that the last rule applied to type $\Gamma \vdash e:t$ can't be $\Rule{Occ}$).
    
    We now proceed by a case analysis on $e$:
    
    \begin{description}
      \item[$c$] Trivial.
      \item[$x$] Impossible case, as we have already treated the rules $\Rule{Empty}$ and $\Rule{Occ}$.
      \item[$\lambda^{\bigwedge_{i\in I} s_i \rightarrow t_i}y.e_y$]
      We know that the last rule in the typing derivation of $\Gamma \vdash e:t$ is $\Rule {Abs}$.
      We have, for each $i \in I$, $\Gamma,y:s_i \vdash e_y:t_i'$ with $t_i'\leq t_i$.
      By induction hypothesis, for each $i$, we get  $\Gamma\rho,y:s_i \vdash e_y\rho:t_i''$ with $t_i''\leq t_i'\leq t_i$. We conclude by applying $\Rule {Abs}$.
      \item[$e_1\ e_2$] We know that the last rule in the typing derivation of $\Gamma \vdash e:t$ is $\Rule {App}$.
      We have $\Gamma\vdash e_1:t_1$ and $\Gamma\vdash e_2:t_2$ with $t_1$ an arrow type and $t_2 \in \dom {t_1}$.
      By induction hypothesis, we get $\Gamma\rho\vdash e_1\rho:t_1'$ with $t_1' \leq t_1$ and $\Gamma\rho\vdash e_2\rho:t_2'$ with $t_2' \leq t_2$. We conclude by applying $\Rule {App}$.
      \item[$\ite {e_0} {t_{\text{if}}} {e_1}{e_2}$] We know that the last rule in the typing derivation of $\Gamma \vdash e:t$ is $\Rule {If}$.
      We have $\Gamma\vdash e_0:t_0$, $\Gamma,\Gamma^+_{\Gamma,e_0,t_{\text{if}}}\vdash e_1 : t_1$ and $\Gamma,\Gamma^-_{\Gamma,e_0,t_{\text{if}}}\vdash e_2 : t_2$.
      \begin{itemize}
        \item By induction hypothesis, we directly get $\Gamma\rho\vdash e_0\rho:t_0'$ with $t_0'\leq t_0$.
        \item Let $\Gamma_1 = \Gamma,\Gamma^+_{\Gamma,e_0,t_{\text{if}}}$. Let $\Gamma_1'=\Gamma_1,(e',\bigwedge_{\{e''\in \dom {\Gamma_1}\alt e''\rho=e'\}} \Gamma_1(e''))$.
        In particular, $\Gamma_1'(e')\leq\Gamma_1(x)=\Gamma_1'(x)$.
        
        Moreover, we have $\Gamma_1' \leq \Gamma_1$ ($e'$ does not contain $x$ and so $e'\rho=e'$).
        As $\Gamma_1 \vdash e_1:t_1$, we have by using the monotonicity lemma $\Gamma_1' \vdash e_1:t_1'$ with $t_1'\leq t_1$.

        Thus, we get by induction hypothesis $\Gamma_1'\rho\vdash e_1\rho:t_1'$ with $t_1' \leq t_1$.

        Let's show that we have $\Gamma^+_{\Gamma\rho,e_0\rho,t_{\text{if}}} \leq \Gamma^+_{\Gamma,e_0,t_{\text{if}}}\rho$.
        For any $e_s$ subexpression of $e_0$ such that $\exists \varpi.\ e_0\downarrow\varpi=e_s$, we have:
        \begin{align*}
          \Gamma^+_{\Gamma\rho,e_0\rho,t_{\text{if}}}(e_s\rho) &= \bigwedge_{\{\varpi\alt e_0\rho\downarrow\varpi=e_s\rho\}} \Gp + {\Gamma\rho,e_0\rho,t_{\text{if}}} (\varpi)\\
          &\leq \bigwedge_{\{\varpi\alt (e_0\downarrow\varpi)\rho=e_s\rho\}} \Gp + {\Gamma\rho,e_0\rho,t_{\text{if}}} (\varpi)\\
          &= \bigwedge_{\substack{e_s' \text{ subexpression of } e_0\\\text{s.t. } \dots \text{ and } e_s'\rho=e_s\rho}} \left(\bigwedge_{\{\varpi\alt e_0\downarrow\varpi=e_s'\}} \Gp + {\Gamma\rho,e_0\rho,t_{\text{if}}} (\varpi)\right)\\
          &\leq \bigwedge_{\substack{e_s' \text{ subexpression of } e_0\\\text{s.t. } \dots \text{ and } e_s'\rho=e_s\rho}} \left(\bigwedge_{\{\varpi\alt e_0\downarrow\varpi=e_s'\}} \Gp + {\Gamma,e_0,t_{\text{if}}} (\varpi)\right)\\
          &= \bigwedge_{\substack{e_s' \text{ subexpression of } e_0\\\text{s.t. } \dots \text{ and } e_s'\rho=e_s\rho}} \Gamma^+_{\Gamma,e_0,t_{\text{if}}} (e_s')\\
          &= (\Gamma^+_{\Gamma,e_0,t_{\text{if}}}\rho)(e_s\rho)
        \end{align*}

        Thus, we have $\Gamma\rho, \Gamma^+_{\Gamma\rho,e_0\rho,t_{\text{if}}} \leq \Gamma\rho,\Gamma^+_{\Gamma,e_0,t_{\text{if}}}\rho = \Gamma_1\rho$.

        Moreover, we have $\Gamma_1'\rho=\Gamma_1\rho$ (recall that $e'\rho=e'$).
        It gives $\Gamma\rho, \Gamma^+_{\Gamma\rho,e_0\rho,t_{\text{if}}} \leq \Gamma_1\rho = \Gamma_1'\rho$.

        We deduce, by using the monotonicity lemma, $\Gamma\rho, \Gamma^+_{\Gamma\rho,e_0\rho,t_{\text{if}}} \vdash e_1\rho:t_1''$ with $t_1'' \leq t_1' \leq t_1$.
        \item We get $\Gamma\rho, \Gamma^-_{\Gamma\rho,e_0\rho,t_{\text{if}}}\vdash e_2\rho:t_2''$ with $t_2'' \leq t_2$ in a similar way.
      \end{itemize}
      We conclude by applying the rule $\Rule {If}$.
    \end{description}

    Now, let's show that $\forall p \forall t' \forall \varpi.\ \Gp p {\Gamma\rho,e\rho,t'} (\varpi) \leq \Gp p {\Gamma,e,t'} (\varpi)$.

    Let $p\in \{+,-\}$ and $t'$ a type.
    We proceed by induction on $\varpi$.

    The base case is straightforward (as we already proved that $\Gamma \rho \vdash e \rho:t'$ with $t'\leq t$).\\
    The inductive cases are also straightfoward using the induction hypothesis (same reasonning as for the proof of monotonicity).

    \qed

    \subsection{Substitution lemma 2}

    Let $\Gamma,e,x,e',t,t_x,t_{e'}$ as in the lemma statement.
    We will denote the substitution $\subst x {e'}$ by $\rho$.

    By the substitution lemma, we get $(\Gamma, (x:t_x))\rho \vdash e\rho:t'$ with $t' \leq t$.
    As $x$ does not appear in $\dom \Gamma$, we have $(\Gamma, (x:t_x))\rho = \Gamma, (e':t_x)$.
    We deduce $\Gamma, (e':t_x) \vdash e\rho:t'$ with $t' \leq t$.

    As $\Gamma \vdash e':t_{e'}$ with $t_{e'} \leq t_x$ and $t_x \neq \Empty$, we have $\Gamma \leq \Gamma, (e':t_x)$.
    Thus, by using the monotonicity lemma, we get $\Gamma\vdash e\rho:t''$ with $t'' \leq t' \leq t$.

    \qed

    \subsection{Test reduction}

    We proceed by induction on $e$.

    Let $\Gamma,t,t_{if},e'$ as in the theorem statement.
    
    We will prove here that $\Gamma^+_{\Gamma,e',t_{if}} \leq \Gamma^+_{\Gamma,e,t_{if}}$.
    The proof for $\Gamma^-_{\dots}$ follows as $\Gamma^-_{\Gamma,e,t_{if}}=\Gamma^+_{\Gamma,e,\neg t_{if}}$.

    We will suppose that all lambda abstracted variables are different in $e$ and $e'$. When it is not the case, we can alpha-rename the terms.
    Let $t'$ such that $\Gamma \vdash e':t'$ and $t'\leq t$ (using subject reduction).

    If $\exists e_{\bot} \in \dom\Gamma.\ \Gamma(e_{\bot}) = \Empty$, thus the property is trivial.
    Let's suppose it is not the case (in particular we know that the last rule applied to type $\Gamma \vdash e:t$ can't be $\Rule{Empty}$).

    If we had $e\in\dom\Gamma$, then $e$ would not be reducible (because $\Gamma$ is ordinary, and variables are not reducible).
    Thus, we can suppose that $e\not\in\dom\Gamma$ (in particular, we know that the last rule applied to type $\Gamma \vdash e:t$ can't be $\Rule{Occ}$).

    We now proceed by case analysis on $e$ (impossible cases are not shown)::

    \begin{description}
      \item[$v_1\ v_2$] In this case, we must have $v_1=\lambda^{\bigwedge_{i\in I} s_i \rightarrow t_i}x.e_x$ and $e'=e_x\subst x {v_2}$.

      The last rule applied to derive $\Gamma \vdash e:t$ is $\Rule{App}$.
      Thus, we know that $\Gamma \vdash v_1 : t_1$ with $t_1$ an arrow type, $\Gamma \vdash v_2 : t_2$, $t_2 \in \dom {t_1}$ and $t=\apply {t_1} {t_2}$.
      
      TODO

      \item[$e_1\ v_2$] In this case, only $e_1$ can be reduced, and the last rule in the derivation of $\Gamma \vdash e:t$ is $\Rule{App}$.
      
      So we have $\Gamma \vdash e_1:t_1$ and $\Gamma \vdash v_2:t_2$ with $t=\apply {t_1} {t_2}$.

      We also have $e' = e_1' v_2$ with $e_1 \leadsto e_1'$.

      \begin{itemize}
        \item Let $t_\epsilon = \Gp + {\Gamma,e,t_{if}} (\epsilon)$ and $t_\epsilon' = \Gp + {\Gamma,e',t_{if}} (\epsilon)$.
        
        Notice that $t_\epsilon' = t_{if} \land t' \leq t_{if} \land t = t_\epsilon$.

        \item Let $t_{left} = \Gp + {\Gamma,e,t_{if}} (0)$ and $t_{left}' = \Gp + {\Gamma,e',t_{if}} (0)$.

        We have $t_{left}' \leq t_{left}$ (direct consequence of $t_\epsilon' \leq t_\epsilon$).
  
        Using the induction hypothesis, we get $\Gamma^+_{\Gamma,e_1',t_{left}'} \leq \Gamma^+_{\Gamma,e_1,t_{left}'}$,
        and using the test monotonicity lemma we deduce $\Gamma^+_{\Gamma,e_1,t_{left}'} \leq \Gamma^+_{\Gamma,e_1,t_{left}}$.

        Thus $\Gamma^+_{\Gamma,e_1',t_{left}'} \leq \Gamma^+_{\Gamma,e_1,t_{left}}$.

        \item Let $t_{right} = \Gp + {\Gamma,e,t_{if}} (1)$ and $t_{right}' = \Gp + {\Gamma,e',t_{if}} (1)$.
        
        As in the previous point, we also have $t_{right}' \leq t_{right}$.

        Thus, using the test monotonicity lemma, we deduce $\Gamma^+_{\Gamma,v_2,t_{right}'} \leq \Gamma^+_{\Gamma,v_2,t_{right}}$.
      \end{itemize}
      
      $\Gamma^+_{\Gamma,e,t_{if}} = \Gamma^+_{\Gamma,v_2,t_{right}} \land \Gamma^+_{\Gamma,e_1,t_{left}} \land (e:t_\epsilon)$ TODO

      \item[$e_1\ e_2$] This case is similar to the previous one.
      \item[$\ite {v_0} {t_{if}} {e_1} {e_2}$] This case is trivial as $\Gamma^+_{\Gamma,e,t_{if}} = (e, t\land t_{if})$. TODO: Actually not trivial.
      \item[$\ite {e_0} {t_{if}} {e_1} {e_2}$] This case is similar to the previous one.
    \end{description}

    \qed

    \subsection{Subject reduction}

    We proceed by induction on $e$.

    Let $\Gamma,t,e'$ as in the theorem statement.
    We will suppose that all lambda abstracted variables are different in $e$ and $e'$. When it is not the case, we can alpha-rename the terms.

    If $\exists e_{\bot} \in \dom\Gamma.\ \Gamma(e_{\bot}) = \Empty$, thus the property is trivial
    (we can apply the rule $\Rule {Empty}$) to type the beta-reduced expression).
    Let's suppose it is not the case (in particular we know that the last rule applied to type $\Gamma \vdash e:t$ can't be $\Rule{Empty}$).

    If we had $e\in\dom\Gamma$, then $e$ would not be reducible (because $\Gamma$ is ordinary, and variables are not reducible).
    Thus, we can suppose that $e\not\in\dom\Gamma$ (in particular, we know that the last rule applied to type $\Gamma \vdash e:t$ can't be $\Rule{Occ}$).

    We now proceed by case analysis on $e$ (impossible cases are not shown):

    \begin{description}
      \item[$v_1\ v_2$] In this case, we must have $v_1=\lambda^{\bigwedge_{i\in I} s_i \rightarrow t_i}x.e_x$ and $e'=e_x\subst x {v_2}$.

      The last rule applied to derive $\Gamma \vdash e:t$ is $\Rule{App}$.
      Thus, we know that $\Gamma \vdash v_1 : t_1$ with $t_1$ an arrow type, $\Gamma \vdash v_2 : t_2$, $t_2 \in \dom {t_1}$ and $t=\apply {t_1} {t_2}$.

      From this last equality, we deduce $t_1 \leq t_2 \rightarrow t$ using the declarative definition of $\circ$.
      It gives $\bigwedge_{i\in I} s_i \rightarrow t_i \leq t_2 \rightarrow t$.
      Let's show that $\Gamma \vdash e_x\subst x {v_2}:t''$ for a certain type $t''\leq t$.

      Let $I'=\{i\in I\alt t_2 \leq s_i\}$. TODO: Alain Frisch magics here...
      We can deduce, as $\bigwedge_{i\in I} s_i \rightarrow t_i \leq t_2 \rightarrow t$,
      that $\bigwedge_{i\in I'} t_i \leq t$ (see lemma 4.9 of Alain Frisch thesis).

      The last rule of the derivation of $\Gamma \vdash v_1 : t_1$ can only be $\Rule{Abs}$ (it can't be $\Rule{Occ}$ because $\Gamma$ is ordinary).
      Thus, we can derive $\Gamma,(x:s_i) \vdash e_x : t_i'$ with $t_i' \leq t_i$ for any $i\in I'$. By using the strengthening lemma, we deduce that
      $\Gamma, (x:\bigwedge_{i\in I'} s_i) \vdash e_x:t'$ with $t'\leq \bigwedge_{i\in I'} t_i' \leq \bigwedge_{i\in I'} t_i \leq t$.
      As $t_2 \leq \bigwedge_{i\in I'} s_i$, we can apply the substitution lemma to get $\Gamma \vdash e_x\subst x {v_2}:t''$ with $t'' \leq t' \leq t$.

      \item[$e_1\ v_2$] In this case, only $e_1$ can be reduced, and the last rule in the derivation of $\Gamma \vdash e:t$ is $\Rule{App}$.

      We just apply the induction hypothesis on $e_1$ and conclude using the monotonicity of the $\circ$ operator.
      \item[$e_1\ e_2$] In this case, only $e_2$ can be reduced, and the last rule in the derivation of $\Gamma \vdash e:t$ is $\Rule{App}$.

      We just apply the induction hypothesis on $e_2$ and conclude using the monotonicity of the $\circ$ operator.
      \item[$\ite {v_0} {t_{if}} {e_1} {e_2}$] In this case, the last rule in the derivation of $\Gamma \vdash e:t$ is $\Rule{If}$.
       
      So we have $\Gamma \vdash v_0 : t_0$, $\Gamma, \Gamma^+_{\Gamma,v_0,t_{if}}\vdash e_1 : t_1$ and $\Gamma, \Gamma^-_{\Gamma,v_0,t_{if}}\vdash e_2 : t_2$ with $t=t_1\vee t_2$.

      There are two cases:
      \begin{description}
        \item[$v_0 \in t_{if}$] In this case, we have $e'=e_1$.
        By the value test lemma, we have $t_0 \leq t_{if}$.
        Thus, we can easily show that $\Gamma^+_{\Gamma,v_0,t_{if}} \leq \Gamma$:
        For each expression in its domain, we proceed by induction on the associated paths.
        Since $v_0$ is a value, the paths $\varpi.0$ and $\varpi.1$ are never encountered. Other cases are trivial.

        Thus $\Gamma, \Gamma^+_{\Gamma,v_0,t_{if}} \equiv \Gamma$.

        By using the equivalence lemma, we can deduce that $\Gamma \vdash e_1 : t_1$, which ends this case since $t_1 \leq t$.
        \item[$v_0 \not\in t_{if}$] This case is similar to the previous one.
      \end{description}
        
      \item[$\ite {e_0} {t_{if}} {e_1} {e_2}$] In this case, only $e_0$ can be reduced, and the last rule in the derivation of $\Gamma \vdash e:t$ is $\Rule{If}$.
      
      So we have $\Gamma \vdash e_0 : t_0$, $\Gamma, \Gamma^+_{\Gamma,e_0,t_{if}}\vdash e_1 : t_1$ and $\Gamma, \Gamma^-_{\Gamma,e_0,t_{if}}\vdash e_2 : t_2$ with $t=t_1\vee t_2$.
      
      We also have $e'=\ite {e_0'} {t_{if}} {e_1} {e_2}$ with $e_0\leadsto e_0'$.

      By applying the test reduction lemma, we get $\Gamma^+_{\Gamma,e_0',t_{if}} \leq \Gamma^+_{\Gamma,e_0,t_{if}}$ and
      $\Gamma^-_{\Gamma,e_0',t_{if}} \leq \Gamma^+_{\Gamma,e_0,t_{if}}$.

      Using the monotonicity lemma, we get $\Gamma, \Gamma^+_{\Gamma,e_0',t_{if}}\vdash e_1 : t_1'$ with $t_1'\leq t_1$
      and $\Gamma, \Gamma^-_{\Gamma,e_0',t_{if}}\vdash e_2 : t_2'$ with $t_2'\leq t_2$, and so we conclude easily.
    \end{description}

    \qed

\end{document}