% Header
\documentclass[a4paper]{article}%      autres choix : book, report
\usepackage[utf8]{inputenc}%           gestion des accents (source)
\usepackage[T1]{fontenc}%              gestion des accents (PDF)
\usepackage[francais]{babel}%          gestion du francais
\usepackage{textcomp}%                 caracteres additionnels
\usepackage{mathtools,amssymb,amsthm}% packages de l'AMS + mathtools
\usepackage{lmodern}%                  police de caractere
\usepackage[top=2cm,left=2cm,right=2cm,bottom=2cm]{geometry}%     gestion des marges
\usepackage{graphicx}%                 gestion des images
\usepackage{array}%                    gestion amelioree des tableaux
\usepackage{calc}%                     syntaxe naturelle pour les calculs
\usepackage{titlesec}%                 pour les sections
\usepackage{titletoc}%                 pour la table des matieres
\usepackage{fancyhdr}%                 pour les en-tetes
\usepackage{titling}%                  pour le titre
\usepackage{enumitem}%                 pour les listes numerotees
\usepackage{hyperref}%                 gestion des hyperliens
\usepackage{syntax}
\usepackage{minted}
\usepackage[parfill]{parskip}
\usepackage{amsmath}
\usepackage{fourier}
\usepackage{heuristica}
\usepackage[overload]{empheq}
\usepackage{cleveref}
\usepackage{alltt}
\usepackage{bm}
\usepackage{nicefrac}
\usepackage{stmaryrd}
\usepackage{mathpartir}
\usepackage{color}
\usemintedstyle{vs}

\hypersetup{pdfstartview=XYZ}%         zoom par defaut

\setlength{\droptitle}{-5em}   % This is your set screw
\title{\vspace{1.5cm}Type inference - M2 Internship}
\author{Mickael LAURENT}
\date{\vspace{-5ex}}

\pagenumbering{gobble}
\DeclarePairedDelimiter\ceil{\lceil}{\rceil}
\DeclarePairedDelimiter\floor{\lfloor}{\rfloor}

\DeclareFontFamily{U}{mathb}{}
\DeclareFontShape{U}{mathb}{m}{n}{
  <-5.5> mathb5
  <5.5-6.5> mathb6
  <6.5-7.5> mathb7
  <7.5-8.5> mathb8
  <8.5-9.5> mathb9
  <9.5-11.5> mathb10
  <11.5-> mathb12
}{}
\DeclareSymbolFont{mathb}{U}{mathb}{m}{n}
\DeclareMathSymbol{\sqdot}{\mathbin}{mathb}{"0D}
\newcommand{\worra}[2]{#1\mathop{\,\sqdot\,} #2}
\newcommand{\apply}[2]{#1\circ#2}
\newcommand{\dom}[1]{\textsf{dom}(#1)}
\newcommand{\alt}{~|~}
\newcommand{\Empty} {\textsf{Empty}}%\MyMathBb{0}}
\newcommand{\Any} {\textsf{Any}}%\MyMathBb{1}}
\newcommand{\Int} {\textsf{Int}}
\newcommand{\Even} {\textsf{Even}}
\newcommand{\Odd} {\textsf{Odd}}
\newcommand{\ite}[4]{\ensuremath{\texttt{if}\;#1\in#2\;\texttt{then}\;#3\;\texttt{else}\;#4}}
\newcommand{\basic}[1]{\text{\fontshape{\itdefault}\fontseries{\bfdefault}\selectfont b}_{#1}}

\theoremstyle{definition}
\newtheorem{theorem}{Theorem}
\newtheorem{lemma}{Lemma}

\newcommand {\Infer} [4] [] {
  \inferrule*[%
    left={\ifx\\#1\\#1\else\ifx\\#2\\[\textsc{#1}]\:\else[\textsc{#1}]\fi\fi},%
    right={$ #4 $}%
  ]%
  {#2}{#3}}

\newcommand {\Rule}[1] {[\textsc{#1}]}

\begin{document}

	\maketitle

    \section{Semantics}

    \subsection{Semantics}

    \[
    \begin{array}{lrcl}
    \textbf{Expressions} & e & ::= & c\alt x \alt \lambda^tx.e \alt \ite e t e e \alt e e\\
    \textbf{Occurrences} & o & ::= & c\alt x \alt \lambda^tx.e \alt \ite e t e e \alt o o\\
    \textbf{Values} & v & ::= & c \alt \lambda^tx.e
    \end{array}
    \]

    \[
    \begin{array}{lrcl}
    \textbf{Context} & C[] & ::= & e [] \alt [] v \alt \ite {[]} t e e
    \end{array}
    \]

    \begin{mathpar}
        \Infer[Ctx]
        {e \leadsto e'}
        {C[e] \leadsto C[e']}
        {}\\
        \Infer[App]
        {}
        {(\lambda^tx.e) v \leadsto e\{x\mapsto v\}}
        {}\\
        \Infer[Ite]
        {i \in \{1,2\}\hspace{1cm} (i=1 \Leftrightarrow v \in t)}
        {\ite v t {e_1} {e_2} \leadsto e_i}
        {}\\
    \end{mathpar}

    \subsection{Recall: Type System}

    \begin{mathpar}

      \Infer[Empty]
        { }
        {\Gamma\vdash e:\Empty}
        {\exists e' \in \dom\Gamma.\ \Gamma(e') \leq \Empty}
      \\
      \Infer[Occ]
          {
          }
          { \Gamma \vdash e: \Gamma(e) }
          { e\in\dom\Gamma}
      \qquad
      \Infer[Const]
          { }
          {\Gamma\vdash c:\basic{c}}
          {c\not\in\dom\Gamma}
       \\
      \Infer[Abs]
          {\Gamma,x:s_i\vdash e:t_i}
          {
          \Gamma\vdash\lambda^{\wedge_{i\in I}s_i\to t_i}x.e:\textstyle\bigwedge_{i\in I}s_i\to t_i
          }
          {\lambda^{\wedge_{i\in I}s_i\to t_i}.e\not\in\dom\Gamma}
          \\
      \Infer[App]
          {
            \Gamma \vdash e_1: t_1 \\
            \Gamma\vdash e_2: t_2\\
            t_1 \leq \Empty \to \Any\\
            t_2 \in \dom {t_1}
          }
          { \Gamma \vdash {e_1}{e_2}: t_1 \circ t_2 }
          { {e_1}{e_2}\not\in\dom\Gamma}
          \\
      \Infer[If]
            {\Gamma\vdash e:t_\circ\\
              \Gamma, \Gamma^+_{\Gamma,e,t}\vdash e_1 : t_1\\
              \Gamma, \Gamma^-_{\Gamma,e,t}\vdash e_2 : t_2}
            {\Gamma\vdash \ite {e} t {e_1}{e_2}: t_1\vee t_2}
            {\makebox[3cm][l]{$\ite {e} t {e_1}{e_2}\not\in\dom\Gamma$}}
      \\\end{mathpar}

    \subsection{Theorems}

    \begin{lemma}[Compositionality]
      If $\Gamma \vdash C[e]:t$, then there exists $t'$ such that $\Gamma \vdash e:t'$ and all $e'$ verifying $\Gamma \vdash e':t'$ also verify $\Gamma \vdash C[e']:t$.
      \textbf{False.}
    \end{lemma}

    \begin{lemma}[Substitution lemma]
      If $\Gamma, (x:t') \vdash e:t$ and $\Gamma \vdash e':t'$ with $t' \neq \Empty$, then $\Gamma\{x\mapsto e'\} \vdash e\{x\mapsto e'\}:t$.
    \end{lemma}

    \begin{theorem}[Subject reduction]
      If $\Gamma \vdash e : t$ and $e \leadsto e'$, then there exists $t' \leq t$ such that $\Gamma \vdash e' : t'$.
    \end{theorem}

    \section{Proofs}

    We will suppose that all variables introduced by lambda abstractions are unique. When it is not the case, we can alpha-rename the terms.

    %\subsection{Compositionality}

    \subsection{Substitution lemma}

    Let $\Gamma,x,t,t',e,e'$ as in the lemma statement.
    We will denote $\Gamma, (x:t')$ by $\Gamma'$ and $\Gamma\{x\mapsto e'\}$ by $\Gamma''$.

    We proceed by induction on the typing derivation of $\Gamma' \vdash e:t$. We proceed by a case analysis on the last derivation rule used:
    \begin{description}
      \item[Empty] In this case we know that $\exists e_{\bot} \in \dom\Gamma.\ \Gamma(e_{\bot}) \leq \Empty$ ($\Gamma(x) \neq \Empty$ by hypothesis).\\
      Thus, we can derive $\Gamma'' \vdash e\{x\mapsto e'\}:\Empty$ by applying the rule $\Rule {Empty}$.
      \item[Occ] If $e=x$, then $t=t'$. As $\Gamma \vdash e':t'$, we can conclude immediatly. Else, we have $e \in \dom \Gamma$ and $\Gamma(e) = t$.
      Thus, we have $e\{x\mapsto e'\} \in \Gamma''$ and $\Gamma''(e\{x\mapsto e'\})$, so we can apply the rule $\Rule {Occ}$.
      \item[Const] This case is trivial.
      \item[Abs] $e=\lambda^{\bigwedge_{i\in I} s_i \rightarrow t_i}y.e_y$. We have, for each $i \in I$, $\Gamma',y:s_i \vdash e_y:t_i$.
      By induction, for each $i$, we get  $\Gamma'',y:s_i \vdash e_y\{x\mapsto e'\}:t_i$. We conclude by applying $\Rule {Abs}$.
      \item[App] $e=e_1 e_2$. We have $\Gamma'\vdash e_1:t_1$ and $\Gamma'\vdash e_2:t_2$ with $t_1$ an arrow type and $t_2 \in \dom {t_1}$.
      By induction, we get $\Gamma''\vdash e_1\{x\mapsto e'\}:t_1$ and $\Gamma''\vdash e_2\{x\mapsto e'\}:t_2$. We conclude by applying $\Rule {App}$.
      \item[If] TODO.
    \end{description}

    \subsection{Subject reduction}

    Let $\Gamma,e,t,e' \text{ s.t. } \Gamma \vdash e : t \text{ and } e \leadsto e'$.

    We proceed by induction on the typing derivation of $\Gamma \vdash e : t$. We proceed by a case analysis on the last derivation rule used:

    \begin{description}
      \item[Empty] In this case we know that $\exists e_{\bot} \in \dom\Gamma.\ \Gamma(e_{\bot}) \leq \Empty$.\\
      Thus we can apply the same rule to type $\Gamma \vdash e' : \Empty$.
    \end{description}


\end{document}