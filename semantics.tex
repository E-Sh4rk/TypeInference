% Header
\documentclass[a4paper]{article}%      autres choix : book, report
\usepackage[utf8]{inputenc}%           gestion des accents (source)
\usepackage[T1]{fontenc}%              gestion des accents (PDF)
\usepackage[francais]{babel}%          gestion du francais
\usepackage{textcomp}%                 caracteres additionnels
\usepackage{mathtools,amssymb,amsthm}% packages de l'AMS + mathtools
\usepackage{lmodern}%                  police de caractere
\usepackage[top=2cm,left=2cm,right=2cm,bottom=2cm]{geometry}%     gestion des marges
\usepackage{graphicx}%                 gestion des images
\usepackage{array}%                    gestion amelioree des tableaux
\usepackage{calc}%                     syntaxe naturelle pour les calculs
\usepackage{titlesec}%                 pour les sections
\usepackage{titletoc}%                 pour la table des matieres
\usepackage{fancyhdr}%                 pour les en-tetes
\usepackage{titling}%                  pour le titre
\usepackage{enumitem}%                 pour les listes numerotees
\usepackage{hyperref}%                 gestion des hyperliens
\usepackage{syntax}
\usepackage{minted}
\usepackage[parfill]{parskip}
\usepackage{amsmath}
\usepackage{fourier}
\usepackage{heuristica}
\usepackage[overload]{empheq}
\usepackage{cleveref}
\usepackage{alltt}
\usepackage{bm}
\usepackage{nicefrac}
\usepackage{stmaryrd}
\usepackage{mathpartir}
\usepackage{color}
\usemintedstyle{vs}

\hypersetup{pdfstartview=XYZ}%         zoom par defaut

\setlength{\droptitle}{-5em}   % This is your set screw
\title{\vspace{1.5cm}Type inference - M2 Internship}
\author{Mickael LAURENT}
\date{\vspace{-5ex}}

\pagenumbering{gobble}
\DeclarePairedDelimiter\ceil{\lceil}{\rceil}
\DeclarePairedDelimiter\floor{\lfloor}{\rfloor}

\DeclareFontFamily{U}{mathb}{}
\DeclareFontShape{U}{mathb}{m}{n}{
  <-5.5> mathb5
  <5.5-6.5> mathb6
  <6.5-7.5> mathb7
  <7.5-8.5> mathb8
  <8.5-9.5> mathb9
  <9.5-11.5> mathb10
  <11.5-> mathb12
}{}
\DeclareSymbolFont{mathb}{U}{mathb}{m}{n}
\DeclareMathSymbol{\sqdot}{\mathbin}{mathb}{"0D}
\newcommand{\worra}[2]{#1\mathop{\,\sqdot\,} #2}
\newcommand{\apply}[2]{#1\circ#2}
\newcommand{\dom}[1]{\textsf{dom}(#1)}
\newcommand{\alt}{~|~}
\newcommand{\Empty} {\textsf{Empty}}%\MyMathBb{0}}
\newcommand{\Any} {\textsf{Any}}%\MyMathBb{1}}
\newcommand{\Int} {\textsf{Int}}
\newcommand{\Bool} {\textsf{Bool}}
\newcommand{\Even} {\textsf{Even}}
\newcommand{\Odd} {\textsf{Odd}}
\newcommand{\ite}[4]{\ensuremath{\texttt{if}\;#1\in#2\;\texttt{then}\;#3\;\texttt{else}\;#4}}
\newcommand{\basic}[1]{\text{\fontshape{\itdefault}\fontseries{\bfdefault}\selectfont b}_{#1}}
\newcommand{\subst}[2]{\{#1 \mapsto #2\}}
\newcommand{\tyof}[2]{\textsf{typeof}_{#2}(#1)}
\newcommand{\typep}[3]{\textsf{Constr}^{#1}_{#3}(#2)}
\newcommand{\Gp}[2]{\textsf{Env}^{#1}_{#2}}

\theoremstyle{definition}
\newtheorem{theorem}{Theorem}
\newtheorem{lemma}{Lemma}
\newtheorem{definition}{Definition}
\newtheorem{corollary}{Corollary}

\newcommand {\Infer} [4] [] {
  \inferrule*[%
    left={\ifx\\#1\\#1\else\ifx\\#2\\[\textsc{#1}]\:\else[\textsc{#1}]\fi\fi},%
    right={$ #4 $}%
  ]%
  {#2}{#3}}

\newcommand {\Rule}[1] {[\textsc{#1}]}

\begin{document}

	\maketitle

    \section{Semantics}

    \subsection{Semantics}

    \[
    \begin{array}{lrcl}
    \textbf{Expressions} & e & ::= & c\alt x \alt \lambda^tx.e \alt \ite e t e e \alt e e\\
    \textbf{Occurrences} & o & ::= & c\alt x \alt \lambda^tx.e \alt \ite e t e e \alt o o\\
    \textbf{Values} & v & ::= & c \alt \lambda^tx.e
    \end{array}
    \]

    \[
    \begin{array}{lrcl}
    \textbf{Context} & C[] & ::= & e [] \alt [] v \alt \ite {[]} t e e
    \end{array}
    \]

    \begin{mathpar}
        \Infer[Ctx]
        {e \leadsto e'}
        {C[e] \leadsto C[e']}
        {}\\
        \Infer[App]
        {}
        {(\lambda^tx.e) v \leadsto e\{x\mapsto v\}}
        {}\\
        \Infer[Ite]
        {i \in \{1,2\}\hspace{1cm} (i=1 \Leftrightarrow v \in t)}
        {\ite v t {e_1} {e_2} \leadsto e_i}
        {}\\
    \end{mathpar}

    \begin{align*}
      n &\in \Int\\
      b &\in \Bool\\
      \lambda^tx.e &\in t\\
      v \in t \text{ and } t \leq t' \Rightarrow v &\in t'
    \end{align*}

    \subsection{Recall: Type System}

    \begin{mathpar}

      \Infer[Empty]
        { }
        {\Gamma\vdash e:\Empty}
        {\exists e' \in \dom\Gamma.\ \Gamma(e') \leq \Empty}
      \\
      \Infer[Occ]
          {
          }
          { \Gamma \vdash e: \Gamma(e) }
          { e\in\dom\Gamma}
      \qquad
      \Infer[Const]
          { }
          {\Gamma\vdash c:\basic{c}}
          {c\not\in\dom\Gamma}
       \\
      \Infer[Abs]
          {\Gamma,x:s_i\vdash e:t_i}
          {
          \Gamma\vdash\lambda^{\wedge_{i\in I}s_i\to t_i}x.e:\textstyle\bigwedge_{i\in I}s_i\to t_i
          }
          {\lambda^{\wedge_{i\in I}s_i\to t_i}x.e\not\in\dom\Gamma}
          \\
      \Infer[App]
          {
            \Gamma \vdash e_1: t_1 \\
            \Gamma\vdash e_2: t_2\\
            t_1 \leq \Empty \to \Any\\
            t_2 \in \dom {t_1}
          }
          { \Gamma \vdash {e_1}{e_2}: t_1 \circ t_2 }
          { {e_1}{e_2}\not\in\dom\Gamma}
          \\
      \Infer[If]
            {\Gamma\vdash e:t_\circ\\
              \Gamma, \Gamma^+_{\Gamma,e,t}\vdash e_1 : t_1\\
              \Gamma, \Gamma^-_{\Gamma,e,t}\vdash e_2 : t_2}
            {\Gamma\vdash \ite {e} t {e_1}{e_2}: t_1\vee t_2}
            {\makebox[3cm][l]{$\ite {e} t {e_1}{e_2}\not\in\dom\Gamma$}}
      \\\end{mathpar}

    \subsection{Definitions}

  \begin{definition}[Environment equivalence]
    Two environments $\Gamma$ and $\Gamma'$ are equivalent (noted $\Gamma \equiv \Gamma'$) iff:
    \begin{align*} 
      &\forall e \in \dom \Gamma.\ \Gamma(e)=t \Rightarrow \Gamma' \vdash e:t \hspace{1cm} \text{and} \hspace{1cm}
      \forall e' \in \dom {\Gamma'}.\ \Gamma(e')=t' \Rightarrow \Gamma \vdash e':t' \\
      \text{and} \hspace{1cm} &\exists e\in \dom \Gamma.\ \Gamma(e)=\Empty \Leftrightarrow \exists e'\in \dom {\Gamma'}.\ \Gamma'(e')=\Empty
    \end{align*}
  \end{definition}

  \begin{definition}[Environment inclusion]
    Let $\Gamma$ and $\Gamma'$ two environments. We says that $\Gamma' \leq \Gamma$ iff:
    \begin{align*}
      &\exists e\in \dom \Gamma.\ \Gamma(e)=\Empty \Rightarrow \exists e'\in \dom {\Gamma'}.\ \Gamma'(e')=\Empty\\
      &\forall e \in \dom \Gamma.\ \tyof e {\Gamma'} \leq \Gamma(e)
    \end{align*}
    Note: this relation is a preorder. $\Gamma' \leq \Gamma \text{ and } \Gamma \leq \Gamma' \Leftrightarrow \Gamma \equiv \Gamma'$.
  \end{definition}

  \begin{definition}[Environment substitution]
    Let $\Gamma$ an environment and $\rho$ a substitution from variables to expressions.
    The environment $\Gamma\rho$ is defined by:
    \begin{align*}
      &\dom {\Gamma\rho} = \dom \Gamma \rho\\
      &\forall e' \in \dom {\Gamma\rho}, (\Gamma\rho)(e') = \bigwedge_{e'' \text{ s.t. } e''\rho=e'}\Gamma(e'')
    \end{align*}
  \end{definition}

    \subsection{Theorems}

    \begin{lemma}[Monotonicity]
      Let $\Gamma$ and $\Gamma'$ two environments such that $\Gamma' \leq \Gamma$. Let $e$ an environment such that $\Gamma \vdash e:t$.
      Then:\\
      \begin{align*}
        &\Gamma' \vdash e:t' \text{ with } t' \leq t\\
        &\forall p \forall t' \forall \varpi.\ \Gp p {\Gamma',e,t'} (\varpi) \leq \Gp p {\Gamma,e,t'} (\varpi)
      \end{align*}
    \end{lemma}

    \begin{lemma}[Test substitution lemma]
      TODO
    \end{lemma}

    \begin{lemma}[Substitution lemma]
      Let $\Gamma$ an environment, $x$ a variable, $e$ an exprssion, $e'$ an expression that does not contain $x$, $t$ and $t_x$ two types.
      If $\Gamma \vdash e:t$ and $\Gamma(x) = t_x$ and $\Gamma \vdash e':t_{e'}$ with $t_{e'} \leq t_x$, then $\Gamma \subst x {e'} \vdash e \subst x {e'}:t'$ with $t'\leq t$.
    \end{lemma}

    \begin{corollary}[Substitution lemma 2]
      Let $\Gamma$ an environment, $x$ a variable that does not appear in $\dom \Gamma$, $e$ an exprssion, $e'$ an expression that does not contain $x$, $t$ a type and $t_x$ a type that is not $\Empty$.\\
      If $\Gamma, (x:t_x) \vdash e:t$ and $\Gamma \vdash e':t_{e'}$ with $t_{e'} \leq t_x$, then $\Gamma \vdash e \subst x {e'}:t'$ with $t'\leq t$.
    \end{corollary}

    \begin{theorem}[Subject reduction]
      If $\Gamma \vdash e : t$ and $e \leadsto e'$, then there exists $t' \leq t$ such that $\Gamma \vdash e' : t'$.
    \end{theorem}

    \section{Proofs}

    \subsection{Monotonicity}

    TODO

    \subsection{Substitution lemma}

    Let $\Gamma,e,x,e',t,t_x,t_{e'}$ as in the lemma statement.
    We will denote the substitution $\subst x {e'}$ by $\rho$.

    We proceed by induction on the typing derivation of $\Gamma \vdash e:t$. We proceed by a case analysis on the last derivation rule used:
    
    \begin{description}
      \item[Empty] In this case we know that $\exists e_{\bot} \in \dom\Gamma.\ \Gamma(e_{\bot}) \leq \Empty$.\\
      It is still true after applying the substitution $\rho$, so we can derive $\Gamma\rho \vdash e\rho:\Empty$ by applying the rule $\Rule {Empty}$.
      \item[Occ] We have $e \in \dom \Gamma$ and $\Gamma(e) = t$.
      Thus, we have $e\rho \in \dom {\Gamma\rho}$ and $\Gamma\rho(e\rho)=t\land t'$ (with $t'$ a type), so we can conclude by applying the rule $\Rule {Occ}$.
      \item[Const] This case is trivial.
      \item[Abs] $e=\lambda^{\bigwedge_{i\in I} s_i \rightarrow t_i}y.e_y$. We have, for each $i \in I$, $\Gamma,y:s_i \vdash e_y:t_i$.
      By induction hypothesis, for each $i$, we get  $\Gamma\rho,y:s_i \vdash e_y\rho:t_i$. We conclude by applying $\Rule {Abs}$.
      \item[App] $e=e_1 e_2$. We have $\Gamma\vdash e_1:t_1$ and $\Gamma\vdash e_2:t_2$ with $t_1$ an arrow type and $t_2 \in \dom {t_1}$.
      By induction hypothesis, we get $\Gamma\rho\vdash e_1\rho:t_1$ and $\Gamma\rho\vdash e_2\rho:t_2$. We conclude by applying $\Rule {App}$.
      \item[If] $e=\ite {e_0} {t_{\text{if}}} {e_1}{e_2}$. We have $\Gamma\vdash e_0:t_0$, $\Gamma,\Gamma^+_{\Gamma,e_0,t_{\text{if}}}\vdash e_1 : t_1$ and $\Gamma,\Gamma^-_{\Gamma,e_0,t_{\text{if}}}\vdash e_2 : t_2$.
      \begin{itemize}
        \item By induction hypothesis, we directly get $\Gamma\rho\vdash e_0\rho:t_0$.
        \item We note $\Gamma_1 = \Gamma,\Gamma^+_{\Gamma,e_0,t_{\text{if}}}$. Let $\Gamma_1'=\Gamma_1,(e',\Gamma_1(x))$.
        We have $\Gamma_1'(x)=t_x'\leq t_x$ and, by construction, $\Gamma_1' \vdash e':t_x'$.
        Thus, we get by induction hypothesis: $\Gamma_1'\rho\vdash e_1\rho:t_1'$ with $t_1' \leq t_1$.\\
        We have $\Gamma_1'\rho=\Gamma_1\rho$ because $e'$ does not contain $x$ (and so $e'\rho=e'$).
        We have $\Gamma_1\rho = \Gamma\rho,\Gamma^+_{\Gamma,e_0,t_{\text{if}}}\rho$.
        By the test substitution lemma, we have $\Gamma^+_{\Gamma\rho,e_0\rho,t_{\text{if}}} \leq \Gamma^+_{\Gamma,e_0,t_{\text{if}}}\rho$,
        and so $\Gamma\rho, \Gamma^+_{\Gamma\rho,e_0\rho,t_{\text{if}}} \leq \Gamma_1\rho = \Gamma_1'\rho$.
        We deduce, by using the monotonicity lemma, $\Gamma\rho, \Gamma^+_{\Gamma\rho,e_0\rho,t_{\text{if}}} \vdash e_1\rho:t_1''$ with $t_1'' \leq t_1' \leq t_1$.
        \item We get $\Gamma\rho, \Gamma^-_{\Gamma\rho,e_0\rho,t_{\text{if}}}\vdash e_2\rho:t_2''$ with $t_2'' \leq t_2$ in a similar way.
      \end{itemize}
      We conclude by applying the rule $\Rule {If}$.
    \end{description}

    \subsection{Substitution lemma 2}

    Let $\Gamma,e,x,e',t,t_x,t_{e'}$ as in the lemma statement.
    We will denote the substitution $\subst x {e'}$ by $\rho$.

    By the substitution lemma, we get $(\Gamma, (x:t_x))\rho \vdash e\rho:t'$ with $t' \leq t$.
    As $x$ does not appear in $\dom \Gamma$, we have $(\Gamma, (x:t_x))\rho = \Gamma, (e':t_x)$.
    We deduce $\Gamma, (e':t_x) \vdash e\rho:t'$ with $t' \leq t$.

    As $\Gamma \vdash e':t_{e'}$ and $t_{e'} \leq t_x$, we have $\tyof {e'} \Gamma \leq t_x$.
    Moreover, $t_x \neq \Empty$, so we have $\Gamma \leq \Gamma, (e':t_x)$.
    Thus, by using the monotonicity lemma, we get $\Gamma\vdash e\rho:t''$ with $t'' \leq t' \leq t$.

    \subsection{Subject reduction}

    Let $\Gamma,e,t,e' \text{ s.t. } \Gamma \vdash e : t \text{ and } e \leadsto e'$.
    We will suppose that all lambda abstracted variables are different in $e$ and $e'$. When it is not the case, we can alpha-rename the terms.

    We proceed by induction on the typing derivation of $\Gamma \vdash e : t$. We proceed by a case analysis on the last derivation rule used:

    \begin{description}
      \item[Empty] In this case we know that $\exists e_{\bot} \in \dom\Gamma.\ \Gamma(e_{\bot}) \leq \Empty$.\\
      Thus we can apply the same rule to type $\Gamma \vdash e' : \Empty$.
    \end{description}


\end{document}