\documentclass[a4paper]{article}

\usepackage{setup}

\hypersetup{pdfstartview=XYZ}%         zoom par defaut

\setlength{\droptitle}{-5em}   % This is your set screw
\title{\vspace{1.5cm}Declarative type system}
\author{}
\date{\vspace{-5ex}}

\pagenumbering{gobble}

\theoremstyle{definition}
\newtheorem{theorem}{Theorem}
\newtheorem{lemma}{Lemma}
\newtheorem{definition}{Definition}
\newtheorem{property}{Property}
\newtheorem{corollary}{Corollary}

\begin{document}

  \maketitle

  \subsubsection{Progress}

      \begin{lemma}[Inversion]
        \begin{align*}
          &\valsemantic{\pair {t_1} {t_2}} = \{(v_1,v_2) \alt \vvdash v_1:t_1, \vvdash v_2:t_2 \}\\
          &\valsemantic{b} = \{c \alt \basic{c} \leq b \}\\
          &\valsemantic{\arrow t s} = \{\lambda^{\bigwedge_{i\in I}\arrow {t_i} {s_i}}x.e \alt \bigwedge_{i\in I}\arrow {t_i} {s_i} \leq \arrow t s\}
        \end{align*}
      \end{lemma}
      Proof: lemma 6.21 of semantic\_subtyping (TODO: ref).

      \begin{theorem}[Progress]
      If $\varnothing \vdash e:t$, then either $e$ is a value or there exists $e'$ such that $e \leadsto e'$.
      \end{theorem}

      Proof:

      We proceed by induction on the derivation $\varnothing \vdash e:t$.
      We consider the last rule of this derivation:

      \begin{description}
        \item[\Rule{Env}] This case is impossible (the environment is empty).
        \item[\Rule{Efq}] This case is impossible (the environment is empty).
        \item[\Rule{Inter}] Straightforward application of the induction hypothesis.
        \item[\Rule{Subs}] Straightforward application of the induction hypothesis.
        \item[\Rule{Const}] In this case, $e$ must be a constant so $e$ is a value.
        \item[\Rule{App}] We have $e=e_1\ e_2$, with $\varnothing \vdash e_1: \arrow s t$
        and $\varnothing \vdash e_2 : s$. If one of the $e_i$ can be reduced, then
        $e$ can also be reduced using the reduction rule \Rule{Ctx}.
        
        Otherwise, by using the induction hypothesis we get that both $e_1$ and $e_2$ are values.
        Moreover, by using the inversion lemma, we know that $e_1$ has the form
        $\lambda^{\bigwedge_{i\in I} \arrow {t_i} {s_i}}x.e_0$. In consequence, $e$ is reducible
        (the reduction rule \Rule{App} can be applied).
        \item[\Rule{Abs+}] In this case, $e$ must be a lambda abstraction, so $e$ is a value.
        \item[\Rule{Abs-}] Straightforward application of the induction hypothesis.
        \item[\Rule{Case}] We have $e = \ite {e_0} {t'} {e_1} {e_2}$. If $e_0$ can be reduced,
        then $e$ can also be reduced using the reduction rule \Rule{TestCtx}.

        Otherwise, by using the induction hypothesis we get that $e_0$ is a value.
        In consequence, $e$ is reducible (the reduction rule \Rule{Case} can be applied).
        \item[\Rule{Proj}] We have $e=\pi_i e_0$, $t=t_i$, $\varnothing \vdash e_0: \pair {t_1} {t_2}$.
        If $e_0$ can be reduced, then $e$ can also be reduced using the rule \Rule{Ctx}.
        
        Otherwise, by using the induction hypothesis we get that $e_0$ is a value.
        Moreover, by using the inversion lemma, we know that $e_0$ has the form $(v_1, v_2)$.
        In consequence, $e$ is reducible (the reduction rule \Rule{Proj} can be applied).
        \item[\Rule{Pair}] We have $e=(e_1,e_2)$. If one of the $e_i$ can be reduced, then
        $e$ can also be reduced using the reduction rule \Rule{Ctx}.

        Otherwise, by using the induction hypothesis we get that both $e_1$ and $e_2$ are values.
        In consequence, $e$ is also a value.
      \end{description}

      \qed

\end{document}