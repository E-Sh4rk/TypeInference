\documentclass[a4paper]{article}

\usepackage{setup}

\hypersetup{pdfstartview=XYZ}%         zoom par defaut

\setlength{\droptitle}{-5em}   % This is your set screw
\title{\vspace{1.5cm}Descriptive type system}
\author{}
\date{\vspace{-5ex}}

\pagenumbering{gobble}

\theoremstyle{definition}
\newtheorem{theorem}{Theorem}
\newtheorem{lemma}{Lemma}
\newtheorem{definition}{Definition}
\newtheorem{property}{Property}
\newtheorem{corollary}{Corollary}

\begin{document}

  \maketitle
  
    \section{Definitions}

    \begin{definition}[Bottom environment]
      Let $\Gamma$ an environment.\\
      $\Gamma$ is bottom (noted $\Gamma = \bot$) iff $\exists e\in\dom\Gamma.\ \Gamma(e)\simeq\Empty$.
    \end{definition}

      \begin{definition}[(Pre)order on environments]
        Let $\Gamma$ and $\Gamma'$ two environments. We say that $\Gamma' \leq \Gamma$ iff:
        \begin{align*}
          &\Gamma'=\bot \text{ or } (\Gamma\neq\bot \text{ and } \forall e \in \dom \Gamma.\ \Gamma' \vdash e : \Gamma(e))
        \end{align*}
      \end{definition}
    
      \begin{definition}[Environment substitution]
        Let $\Gamma$ an environment and $\rho$ a substitution from variables to expressions.
        The environment $\Gamma\rho$ is defined by:
        \begin{align*}
          &\dom {\Gamma\rho} = \dom \Gamma \rho\\
          &\forall e \in \dom {\Gamma\rho}, (\Gamma\rho)(e) = \bigwedge_{\{e' \in \dom \Gamma \alt e'\rho\equiv e\}}\Gamma(e')
        \end{align*}
      \end{definition}
    
      \begin{definition}[Ordinary environments]
        We say that an environment $\Gamma$ is ordinary iff its domain only contains variables.
      \end{definition}

    \section{Theorems}

        \begin{property}[$\valsemantic \_$ properties]
          \begin{align*}
            &\forall s.\ \forall t.\ \valsemantic s \subseteq \valsemantic t \Leftrightarrow s \leq t\\
            &\valsemantic \Empty = \varnothing\\
            &\forall t.\ \valsemantic {\neg t} = \values \setminus \valsemantic t\\
            &\forall s.\ \forall t.\ \valsemantic {s\vee t} = \valsemantic s \cup \valsemantic t
          \end{align*}
        \end{property}
        Proof: theorem 5.5, lemmas 6.19, 6.22, 6.23 of semantic_subtyping [TODO: ref].

        \begin{lemma}[Alpha-renaming]
          Both the type system and the semantics are invariant by alpha-renaming.
        \end{lemma}
        Proof: quite straightforward.
        For the type system, it is a consequence of the fact that environments are up to alpha-renaming.
        For the semantics, it is a consequence of the fact that parallel substitutions (for the case of the $\texttt{if}$)
        are up to alpha-renaming.

        \begin{lemma}[$\vvdash$ and $\vdash$ relations]Let $v$ a value, $t$ a type and $\Gamma$ an environment.

          If $\vvdash v:t$ and $v$ is well-typed in $\Gamma$, then $\Gamma\vdash v:t$.

          If $\Gamma \vdash v:t$ and $\Gamma\neq\bot$, then $\vvdash v:t$.
        \end{lemma}
    
        \begin{lemma}[Monotonicity]
          Let $\Gamma$ and $\Gamma'$ two environments such that $\Gamma' \leq \Gamma$.
          Then, we have:
          \begin{align*}
            \forall e,t.\ &\Gamma \vdash e:t \Rightarrow \Gamma' \vdash e:t\\
            \forall e,t,p,\Gamma^p.\ &\Gamma \evdash p e t \Gamma^p \Rightarrow \exists {\Gamma^p}'\leq \Gamma^p.\ \Gamma' \evdash p e t {\Gamma^p}'\\
            \forall e,t,p,\varpi,t'.\ &\pvdash \Gamma p e t \varpi:t' \Rightarrow \pvdash {\Gamma'} p e t \varpi:t'
          \end{align*}
        \end{lemma}
        Proof: this is trivial. We just have to replace every \Rule{Occ} rule in the derivation with $\Gamma$
        by the relevant derivation with $\Gamma'$.

        \begin{corollary}[Preorder relation]
          The relation $\leq$ on environements is a preorder.
        \end{corollary}
    
        \begin{corollary}[Strengthening]
          If $\Gamma, (x:t_1^x) \vdash e:t_1$ and $\Gamma, (x:t_2^x) \vdash e:t_2$, then
          $\Gamma, (x:t_1^x \land t_2^x) \vdash e:t_1\land t_2$.
        \end{corollary}
        Proof: use the \Rule{Intersect} rule and the monotonicity lemma.

        \begin{lemma}[Value refinement 1]
          For any derivable judgement of the form $\pvdash \Gamma p e t \varpi.x:\Empty$ with $x\in\{0,1\}$ and $\occ e \varpi = v$ a value,
          we have $\pvdash \Gamma p e t \varpi:\Empty$.
        \end{lemma}
        Proof: TODO.

        \begin{corollary}[Value refinement 2]
          For any derivable judgement of the form $\Gamma \evdash p e t \Gamma'$, we can construct a derivation that
          never uses the rule \Rule{Path} with $\varpi\neq\epsilon$ and $\occ e \varpi$ refering to a value.
        \end{corollary}
        Proof: if the refined type of a value was not already derivable, then we can also refine this value to $\Empty$. TODO

        \begin{lemma}[Value testing]
          For any derivable judgement of the form $\Gamma \evdash p v t \Gamma'$, if we have $v \in \valsemantic{t} \Leftrightarrow p=+$,
          then we have $\Gamma\leq\Gamma'$.
        \end{lemma}
        Proof: TODO.

        \begin{lemma}[Substitution]
          Let $\Gamma$ an environment. Let $e_a$ and $e_b$ two expressions.

          Let's suppose that $e_b$ is a closed term and that $e_a$ has one of the following form:
          \begin{itemize}
            \item $x$ (variable)
            \item $\ite e t {e_1} {e_2}$ (if expression)
            \item $v$ (value)
            \item $v v$ (value applied to a value)
          \end{itemize}
          Let's also suppose that $\forall t'.\ \Gamma \vdash e_a : t' \Rightarrow \Gamma\subst {e_a} {e_b} \vdash e_b:t'$.
          
          Then, by noting $\rho = \subst {e_a} {e_b}$ we have:
          \begin{align*}
            &\forall e,t.\ \Gamma \vdash e:t \Rightarrow \Gamma\rho \vdash e\rho:t\\
            &\forall e,t,p,\Gamma^p.\ \Gamma \evdash p {e} {t} \Gamma^p \Rightarrow \Gamma\rho \evdash p {e\rho} {t} {\Gamma^p}\rho
            \text{ and we still have } \forall t'.\ \Gamma^p \vdash e_a : t' \Rightarrow \Gamma^p\rho \vdash e_b:t'\\
            &\forall e,t,p,\varpi,t'.\ \pvdash \Gamma p e t \varpi:t' \text{ and }\occ {e\rho} \varpi\text{ is defined} \Rightarrow \pvdash {\Gamma\rho} p {e\rho} t \varpi:t'
          \end{align*}
        \end{lemma}

        \begin{lemma}[Test reduction]
          Let $\Gamma$ an ordinary environment. Let $e$ and $e'$ two expressions, $t_{if}$ a type and $p \in \{+,-\}$.
          If $e \xleadsto{\rho} e'$ and if the subject reduction lemma holds for every subexpression of $e$ that satisfies its hypotheses,
          then $\forall \Gamma^p.\ \Gamma \evdash p e {t_{if}} \Gamma^p \Rightarrow \Gamma \evdash p {e'} {t_{if}} \Gamma^p\rho$.
        \end{lemma}

        \begin{theorem}[Subject reduction]
          Let $\Gamma$ an ordinary environment, $e$ and $e'$ two expressions and $t$ a type.

          If $\Gamma\vdash e:t$ and $e\leadsto e'$, then $\Gamma\vdash e':t$.
        \end{theorem}

        \section{Proofs}

        \subsection{Substitution lemma}

        Let $\Gamma$, $e_a$, $e_b$ as in the statement.

        We note $\rho$ the substitution $\subst {e_a} {e_b}$.

        We consider a derivation of a judgement as in the hypotheses (left-side of the implications).

        By using the value refinement lemma, we can assume without loss of generality that our derivation does not contain
        any rule \Rule{Path} with $\varpi\neq\epsilon$ and $\occ e \varpi$ refering to a value.

        We can also assume w.l.o.g. that every \Rule{Path} rule is such that $\Gamma',(\occ e \varpi:t') \leq \Gamma'$. If it is not the case,
        we can easily transform the derivation by intersecting the $t'$ with $\Gamma'(\occ e \varpi)$
        using the rules \Rule{PIntersect}, \Rule{PTypeof} and \Rule{Occ}.
        The rest of the derivation can easily be adapted by adding some \Rule{Subs} rules when needed.

        Finally, we can assume that, in any environment appearing in the derivation, if the environnement is not bottom,
        then a value $v$ can only be mapped to a type $t$ such that $v\in\valsemantic{t}$. If it is not the case, then we just have to modify the
        \Rule{Path} rule that introduced $(v:t)$ in the environment and add $(v:\Empty)$ instead by using the rules \Rule{PIntersect} and \Rule{PTypeof}.

        We now proceed by induction on this derivation in order to produce the required derivation.
        
        If the last judgement is of the form $\Gamma \vdash e_a: t$, then we can directly conclude with the hypotheses of the lemma.
        Thus, we can suppose it is not the case.

        There are many cases depending on the last rule:

        \begin{description}
          \item[\Rule{Occ}] If $e\in\dom\Gamma$, then we have $e\rho\in\dom{\Gamma\rho}$ and $(\Gamma\rho)(e\rho)\leq\Gamma(e)$.
          Thus we can easily derive $\Gamma\rho\vdash e\rho:t$ with the rule \Rule{Occ} and \Rule{Subs}.
          \item[\Rule{EFQ}] If there exists $e\in\dom\Gamma$ such that $\Gamma(e)=\Empty$, then $(\Gamma\rho)(e\rho)=\Empty$
          so we can easily derive $\Gamma\rho\vdash e\rho:t$ with the rule \Rule{EFQ}.
          \item[\Rule{Intersect}] Trivial (by using the induction hypothesis).
          \item[\Rule{Subs}] Trivial (by using the induction hypothesis).
          \item[\Rule{Const}] In this case, $c\rho = c$ (because $c \neq e_a$). So it is trivial.
          \item[\Rule{App}] We have $(e_1 e_2)\rho = (e_1\rho) (e_2\rho)$ (because $e_1 e_2 \neq e_a$).
          So it is trivial (by using the induction hypothesis).
          \item[\Rule{Abs}] We have $(\lambda^{t'}x.e)\rho = \lambda^{t'}x.(e\rho)$ (because $\lambda^{t'}x.e \neq e_a$).
          
          By alpha-renaming, we can suppose that the variable $x$ is a new fresh variable that does not appear
          in $e_a$ nor $e_b$ ($e_b$ is closed).
          
          We can thus use the induction hypothesis on all the judgements $\Gamma, x:s_i \vdash e:t_i$.
          \item[\Rule{If}]
          We have $(\ite e {t_{if}} {e_1} {e_2})\rho = \ite {e\rho} {t_{if}} {e_1\rho} {e_2\rho}$ (because $\ite e {t_{if}} {e_1} {e_2} \neq e_a$).

          We apply the induction hypothesis on the judgements $\Gamma\vdash e:t_0$ and $\Gamma\evdash + e {t_{if}} \Gamma^+$.
          We get $\Gamma\rho\vdash e\rho:t_0$, $\Gamma\rho\evdash + {e\rho} {t_{if}} \Gamma^+\rho$ and
          $\forall t'.\ \Gamma^+ \vdash e_a : t' \Rightarrow \Gamma^+\rho \vdash e_b:t'$.
          Now, we can apply the induction hypothesis on $\Gamma^+\vdash e_1:t$ and we have $\Gamma^+\rho\vdash e_1\rho:t$.

          We proceed similarly on the judgments $\Gamma\evdash - e {t_{if}} \Gamma^-$ and $\Gamma^-\vdash e_1:t$, and so we have all the premises
          to apply the \Rule{If} rule in order to get $\Gamma\rho \vdash \ite {e\rho} {t_{if}} {e_1\rho} {e_2\rho}:t'$. 

          \item[\Rule{Base}] Trivial.
          \item[\Rule{Path}] First, let's show that we can derive $\Gamma\rho \evdash p {e\rho} {t} {\Gamma^p}\rho$.
          
          There are two cases:
          \begin{itemize}
            \item $\occ e \varpi$ is a strict subexpression of $e_a$.
            
            In this case, it means that among its three possible forms,
            $e_a$ is of the form $v v$. Thus, $\occ e \varpi$ is a value.
            According to the assumptions we made on the derivation, it implies that $\varpi=\epsilon$.
            Hence, $e$ does not contain any occurence of $e_a$, so it is easy to conclude.

            \item $\occ e \varpi$ is not a strict subexpression of $e_a$.
            
            In this case, we know that $\occ {e\rho} \varpi$ is defined.
            We have by using the induction hypothesis $\Gamma\rho \evdash p {e\rho} t \Gamma'\rho$
            and $\forall t''.\ \Gamma' \vdash e_a : t'' \Rightarrow \Gamma'\rho \vdash e_b:t''$.

            Thus we can also apply the induction hypothesis on $\pvdash {\Gamma'} p {e} t \varpi:t'$.
            It gives $\pvdash {\Gamma'\rho} p {e\rho} t \varpi:t'$.
            If $\occ {e\rho}\varpi \in\dom{\Gamma'\rho}$, and $(\Gamma'\rho)(\occ {e\rho}\varpi)=t''\not\geq t'$,
            then we can derive $\pvdash {\Gamma'\rho} p {e\rho} t \varpi:t'\land t''$ just by using the rules
            \Rule{PIntersect}, \Rule{PTypoef} and \Rule{Occ}.

            Using this last judgement together with $\Gamma \evdash p e t \Gamma'$, we can derive with the rule \Rule{Path}
            the wanted $\Gamma\rho \evdash p {e\rho} {t} {\Gamma^p}\rho$.
          \end{itemize}

          Now, let's show that $\forall t'.\ \Gamma^p \vdash e_a : t' \Rightarrow \Gamma^p\rho \vdash e_b:t'$.

          Let $t'$ such that $\Gamma^p\vdash e_a:t'$.

          By induction hypothesis, we have $\Gamma' \vdash e_a : t' \Rightarrow \Gamma'\rho \vdash e_b:t'$,
          with $\Gamma^p = \Gamma',(\occ e \varpi: t')$.

          If $\Gamma^p = \bot$, then $\Gamma^p\rho = \bot$ so we are done. So lets's suppose $\Gamma^p \neq \bot$.

          Let's separate the proof in two cases:
          \begin{itemize}
            \item If $\occ e \varpi \not\equiv e_a$. In this case, let's show that we have  $\Gamma'\vdash e_a:t'$.
            Indeed, in the typing derivation of $\Gamma^p\vdash e_a:t'$, the \Rule{Occ} rules can only be applied on
            subexpressions of $e_a$.
            
            If $\occ e \varpi$ is not a strict subexpression of $e_a$
            (and thus not a subexpression as $\occ e \varpi \not\equiv e_a$), there is no \Rule{Occ} rule applied to $\occ e \varpi$
            in the derivation of $\Gamma^p\vdash e_a:t'$ and thus we can easily derive $\Gamma'\vdash e_a:t'$.

            If $\occ e \varpi$, is a strict subexpression of $e_a$, it must be a value (given the possible forms of $e_a$).
            Moreover, as $\Gamma^p \neq \bot$, we have $\forall v\in\dom{\Gamma^p}.\ v\in\valsemantic{\Gamma^p(v)}$ (recall the assumptions at the beginning of the proof)
            and thus $\forall v\in\dom{\Gamma^p}.\ \Gamma'\vdash v : \Gamma^p(v)$.
            Thus we can derive $\Gamma'\vdash e_a:t'$ just by replacing every \Rule{Occ} rule applied to $\occ e \varpi$ in the derivation of $\Gamma^p\vdash e_a:t'$
            by the relevant derivation.

            From $\Gamma'\vdash e_a:t'$ we deduce $\Gamma'\rho\vdash e_b:t'$.
            As $\Gamma^p\leq\Gamma'$ (according to the assumptions we made on the derivation beginning of this proof)
            and $\dom{\Gamma'}\subseteq\dom{\Gamma^p}$, we have $\Gamma^p\rho\leq\Gamma'\rho$ and thus, by monotonicity,
            $\Gamma^p\rho\vdash e_b:t'$.

            \item            
            If $\occ e \varpi \equiv e_a$. Let's note $t_a=\Gamma^p(e_a)$. This time,
            we can't derive $\Gamma'\vdash e_a:t'$ from $\Gamma^p\vdash e_a:t'$ because the rule \Rule{Occ}
            could be used on $\occ e \varpi=e_a$ (which may not be a value).

            However, the rule \Rule{Occ} can only be used on $e_a$ at the bottom of the derivation of $\Gamma^+\vdash e_a:t'$:
            there can't be any \Rule{App}, \Rule{Abs} or \Rule{If} after because the premises of these rules only contain strict subexpressions of their
            consequence. The only rules that can appear in a branch containing a \Rule{Occ} applied on $e_a$ are \Rule{Occ}, \Rule{Subs} and \Rule{Intersect}
            (not \Rule{EFQ} as $\Gamma^p \neq \bot$).

            Thus, we can (temporarily) remove from the derivation all these branchs
            (for each \Rule{Occ} we want to remove, we go down until we have an \Rule{Intersect} rule,
            we remove this \Rule{Intersect} rule and replace it by its other premise).

            It gives us a derivation for $\Gamma^p \vdash e_a : t''$ such that $t''\land t_a \leq t'$ and without any \Rule{Occ} applied to $e_a$.
            Thus, we can transform it into a derivation of $\Gamma' \vdash e_a : t''$ as in the previous point, and we get $\Gamma'\rho \vdash e_b : t''$.
            Still as before, we get a derivation for $\Gamma^p\rho\vdash e_b : t''$ by monotonicity.
            
            Now, we can append at the end of this derivation a rule \Rule{Intersect} with a rule \Rule{Occ} applied to $e_b$.
            As $(\Gamma^p\rho)(e_b) \leq \Gamma^p(e_a) = t_a$, we obtain by monotonicity a derivation for $\Gamma^p\rho\vdash e_b : t'$ (we can add a final \Rule{Subs} rule if needed).
          \end{itemize}
          
          \item[\Rule{PTypeof}] Trivial (by using the induction hypothesis).
          \item[\Rule{P$\cdots$}] All the remaining rules are trivial.
        \end{description}

        \subsection{Subject reduction}

        Let $\Gamma$, $e$, $e'$ and $t$ as in the statement.

        We construct a derivation for $\Gamma \vdash e':t$ by induction on the derivation of $\Gamma \vdash e:t$.

        We proceed by case analysis on the last rule of the derivation:
        
        \begin{description}
          \item[\Rule{Occ}] As $\Gamma$ is ordinary, it means that $e$ is a variable.
          It contradicts the fact that $e$ reduces to $e'$ so this case is impossible.
          \item[\Rule{EFQ}] As $\Gamma$ is ordinary, it means that $e$ is a variable.
          It contradicts the fact that $e$ reduces to $e'$ so this case is impossible. 
          \item[\Rule{Intersect}] Trivial (by using the induction hypothesis).
          \item[\Rule{Subs}] Trivial (by using the induction hypothesis).
          \item[\Rule{Const}] Impossible case (no reduction possible).
          \item[\Rule{App}] In this case, $e\equiv e_1 e_2$.
          There is three possible cases:
          \begin{itemize}
            \item $e_2$ is not a value. In this case, we must have $e_2\leadsto e_2'$
            and $e'\equiv e_1 e_2'$. We can easily conclude using the induction hypothesis.
            \item $e_2$ is a value and $e_1$ is not. In this case, we must have $e_1\leadsto e_1'$
            and $e'\equiv e_1' e_2$. We can easily conclude using the induction hypothesis.
            \item Both $e_1$ and $e_2$ are values. This is the difficult case.
            We have $e_1\equiv \lambda^{\bigwedge_{i\in I}\arrow{s_i}{t_i}}x.e_x$
            with $\bigwedge_{i\in I}\arrow{s_i}{t_i} \leq \arrow{s}{t}$ and $\Gamma \vdash e_2:s$.
            We can suppose that $x$ is a new fresh variable that does not appear in our environment
            (if it is not the case, we can alpha-rename $e_1$).

            This means that $s\leq \bigvee_{i\in I} s_i$ and that for any non-empty $I'$ such that
            $s\not\leq \bigvee_{i\in I\setminus I'} s_i$, we have $\bigwedge_{i\in I'} t_i \leq t$
            (TODO: ref lemma 6.8 semantic subtyping). Let's take $I'=\{i\in I\alt \vvdash e_2:s_i\}$.
            We have $I'$ not empty: $\vvdash e_2:s$ and $s\leq \bigvee_{i\in I} s_i$, so according to
            $\valsemantic \_$ properties we have at least one $i$ such that $\vvdash e_2:s_i$.
            We also have $s\not\leq \bigvee_{i\in I\setminus I'} s_i$, otherwise there would be a $i\not\in I$
            such that $\vvdash e_2: s_i$ (contradiction with the definition of $I'$).
            As a consequence, we get $\bigwedge_{i\in I'} t_i \leq t$.

            Now, let's prove that $\Gamma \vdash e':\bigwedge_{i\in I'} t_i$ (which, by subsumption,
            yields $\Gamma \vdash e': t$). For that, we show that for any $i\in I'$, $\Gamma \vdash e':t_i$
            (it is then easy to conclude by using the \Rule{Intersect} rule).

            Let $i\in I'$. We have $\vvdash e_2:s_i$, and so $\Gamma \vdash e_2:s_i$ ($e_2$ is well-typed in $\Gamma$).
            As $e_1$ is well-typed in $\Gamma$, there must be in its derivation an application of the rule $\Rule{Abs}$
            which guarantees $\Gamma,(x:s_i) \vdash e_x:t_i$ (recall that $\Gamma$ is ordinary so there is no abstraction in $\dom\Gamma$).
            Let's note $\Gamma'=\Gamma,(x:s_i)$.
            We can deduce, using the substitution lemma, that $\Gamma'\subst x {e_2} \vdash e_x\subst x {e_2}: t_i$.
           
            Moreover, $\Gamma'\subst x {e_2} = \Gamma,(e_2:s_i)$ and $\Gamma \leq \Gamma,(e_2:s_i)$.
            Thus, by monotonicity, we deduce $\Gamma \vdash e_x\subst x {e_2}: t_i$,
            that is $\Gamma \vdash e': t_i$.
          \end{itemize}
          \item[\Rule{Abs}] Impossible case (no reduction possible).
          \item[\Rule{If}] In this case, $e\equiv \ite {e_0} {t_{if}} {e_1} {e_2}$. There are three possible cases:
          \begin{itemize}
            \item $e_0$ is a value and $e_0 \in \valsemantic{t_{if}}$. In this case we have $e' \equiv e_1$.
            We have derivations for $\Gamma \vdash e_0: t_0$, $\Gamma \evdash + {e_0} {t_{if}} \Gamma'$ and $\Gamma'\vdash e_1:t$.
            
            As $e_0$ is a value and $e_0 \in \valsemantic{t_{if}}$, we have $\Gamma\leq\Gamma'$ by using the value testing lemma.
            Thus, by monotonicity, we have $\Gamma\vdash e_1:t$.
            \item $e_0$ is a value and $e_0 \not\in \valsemantic{t}$. This case is very similar to the previous one.
            \item $e_0$ is not a value. TODO
          \end{itemize}
        \end{description}

\end{document}