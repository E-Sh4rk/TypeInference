\documentclass[a4paper]{article}

\usepackage{setup}

\hypersetup{pdfstartview=XYZ}%         zoom par defaut

\setlength{\droptitle}{-5em}   % This is your set screw
\title{\vspace{1.5cm}Descriptive type system}
\author{}
\date{\vspace{-5ex}}

\pagenumbering{gobble}

\theoremstyle{definition}
\newtheorem{theorem}{Theorem}
\newtheorem{lemma}{Lemma}
\newtheorem{definition}{Definition}
\newtheorem{property}{Property}
\newtheorem{corollary}{Corollary}

\begin{document}

  \maketitle

  In the following, the substitutions on expressions that we introduce are up to alpha-renaming and perform only one pass.
  For instance, if our substitution is $\rho=\subst {(\lambda^t x. x) y} {y}$, we have $((\lambda^t x. x)((\lambda^t z. z) y))\rho = (\lambda^t x. x) y$.
  
  The environements also operate up to alpha-renaming.

    \section{Definitions}

    \begin{definition}[Bottom environment]
      Let $\Gamma$ an environment.\\
      $\Gamma$ is bottom (noted $\Gamma = \bot$) iff $\exists e\in\dom\Gamma.\ \Gamma(e)\simeq\Empty$.
    \end{definition}

      \begin{definition}[(Pre)order on environments]
        Let $\Gamma$ and $\Gamma'$ two environments. We say that $\Gamma' \leq \Gamma$ iff:
        \begin{align*}
          &\Gamma'=\bot \text{ or } (\Gamma\neq\bot \text{ and } \forall e \in \dom \Gamma.\ \Gamma' \vdash e : \Gamma(e))
        \end{align*}
      \end{definition}
    
      \begin{definition}[Environment substitution]
        Let $\Gamma$ an environment and $\rho$ a substitution from variables to expressions.
        The environment $\Gamma\rho$ is defined by:
        \begin{align*}
          &\dom {\Gamma\rho} = \dom \Gamma \rho\\
          &\forall e \in \dom {\Gamma\rho}, (\Gamma\rho)(e) = \bigwedge_{\{e' \in \dom \Gamma \alt e'\rho\equiv e\}}\Gamma(e')
        \end{align*}
      \end{definition}
    
      \begin{definition}[Ordinary environments]
        We say that an environment $\Gamma$ is ordinary iff its domain only contains variables.
      \end{definition}

    \section{Theorems}

        \begin{property}[$\valsemantic \_$ properties]
          \begin{align*}
            &\forall s.\ \forall t.\ \valsemantic s \subseteq \valsemantic t \Leftrightarrow s \leq t\\
            &\valsemantic \Empty = \varnothing\\
            &\forall t.\ \valsemantic {\neg t} = \values \setminus \valsemantic t\\
            &\forall s.\ \forall t.\ \valsemantic {s\vee t} = \valsemantic s \cup \valsemantic t
          \end{align*}
        \end{property}
        Proof: theorem 5.5, lemmas 6.19, 6.22, 6.23 of semantic_subtyping [TODO: ref].

        \begin{lemma}[Alpha-renaming]
          Both the type system and the semantics are invariant by alpha-renaming.
        \end{lemma}
        Proof: quite straightforward.
        For the type system, it is a consequence of the fact that environments are up to alpha-renaming.
        For the semantics, it is a consequence of the fact that parallel substitutions (for the case of the $\texttt{if}$)
        are up to alpha-renaming.

        \begin{lemma}[$\vvdash$ and $\vdash$ relations]Let $v$ a value, $t$ a type and $\Gamma$ an environment.

          If $\vvdash v:t$ and $v$ is well-typed in $\Gamma$, then $\Gamma\vdash v:t$.

          If $\Gamma \vdash v:t$ and $\Gamma\neq\bot$, then $\vvdash v:t$.
        \end{lemma}
    
        \begin{lemma}[Monotonicity]
          Let $\Gamma$ and $\Gamma'$ two environments such that $\Gamma' \leq \Gamma$.
          Then, we have:
          \begin{align*}
            \forall e,t.\ &\Gamma \vdash e:t \Rightarrow \Gamma' \vdash e:t\\
            \forall e,t,\Gamma_1.\ &\Gamma \evdash e t \Gamma_1 \Rightarrow \exists {\Gamma_1}'\leq \Gamma_1.\ \Gamma' \evdash e t {\Gamma_1}'\\
            \forall e,t,\varpi,t'.\ &\pvdash \Gamma e t \varpi:t' \Rightarrow\ \pvdash {\Gamma'} e t \varpi:t'
          \end{align*}
        \end{lemma}
        Proof: We just have to replace every \Rule{Env} rule in the derivation with $\Gamma$
        by the relevant derivation with $\Gamma'$, followed by a \Rule{Subs} if needed.

        \begin{corollary}[Preorder relation]
          The relation $\leq$ on environements is a preorder.
        \end{corollary}

        \begin{lemma}[Value refinement 1]
          If we have $\pvdash \Gamma e t \varpi.x:t'$ with $x\in\{0,1,l,r,f,s\}$ (and $e$ well-typed in $\Gamma$) such that $\forall y.\ \occ e {\varpi.y}$ is a value
          and $v = \occ e {\varpi.x} \not\in \valsemantic{t'}$, we can derive $\pvdash \Gamma e t \varpi:\Empty$.
        \end{lemma}

        \begin{corollary}[Value refinement 2]
          For any derivable judgement of the form $\Gamma \evdash e t \Gamma'$ (with $e$ well-typed in $\Gamma$), we can construct a derivation of $\Gamma \evdash e t \Gamma''$ with $\Gamma''\leq\Gamma'$
          that never uses the rule \Rule{Path} on a path $\varpi.x$ such that $\forall y.\ \occ e {\varpi.y}$ refers to a value.
        \end{corollary}
        Proof: We can easily remove every such rule from the derivation. If $\occ e {\varpi.x} \in \valsemantic{t'}$, the \Rule{Path} rule is useless
        and we can freely remove it. Otherwise, if $\occ e {\varpi.x} \not\in \valsemantic{t'}$, we can use the previous lemma to
        replace it with a \Rule{Path} rule on $\varpi$.

        \begin{lemma}[Value testing]
          For any derivable judgement of the form $\Gamma \evdash v t \Gamma'$ (with $v$ a value),
          we have $v \in \valsemantic{t} \Rightarrow \Gamma\leq\Gamma'$.
        \end{lemma}

        \begin{lemma}[Substitution]
          Let $\Gamma$ an environment. Let $e_a$ and $e_b$ two expressions.

          Let's suppose that $e_b$ is closed and that $e_a$ has one of the following form:
          \begin{itemize}
            \item $x$ (variable)
            \item $\ite e t {e_1} {e_2}$ (if expression)
            \item $v$ (value)
            \item $v v$ (application of two values)
            \item $(v,v)$ (product of two values)
          \end{itemize}
          Let's also suppose that $\forall t.\ \Gamma \vdash e_a : t \Rightarrow \Gamma\subst {e_a} {e_b} \vdash e_b:t$.
          
          Then, by noting $\rho = \subst {e_a} {e_b}$ we have:
          \begin{align*}
            &\forall e,t.\ \Gamma \vdash e:t \Rightarrow \Gamma\rho \vdash e\rho:t
          \end{align*}
        \end{lemma}

        \begin{theorem}[Subject reduction]
          Let $\Gamma$ an ordinary environment, $e$ and $e'$ two expressions and $t$ a type.

          If $\Gamma\vdash e:t$ and $e\leadsto e'$, then $\Gamma\vdash e':t$.
        \end{theorem}

        \section{Proofs}

        \subsection{Value refinement 1}

        We proceed by induction on the derivation of $\pvdash \Gamma e t \varpi.x:t'$.

        We perform a case analysis on the last rule:
        \begin{description}
          \item[\Rule{PTypeof}] In this case we have $\Gamma \vdash \occ e {\varpi.x} : t'$ with $v \not\in \valsemantic{t'}$.
          Thus we can derive $\Gamma \vdash \occ e {\varpi.x} : \Empty$ by using the rule \Rule{Inter} and the rules \Rule{Abs+}, \Rule{Abs-} or \Rule{Const}.

          Let's show that we also have $\Gamma \vdash \occ e \varpi : \Empty$.
          \begin{itemize}
            \item If $x=0$, we know that $\occ e \varpi$ is an application, and we can conclude easily given that $\Empty \leq \arrow \Any \Empty$.
            \item If $x=1$, we know that $\occ e \varpi$ is an application, and we can conclude easily given that $\arrow \Empty \Empty \simeq \arrow \Empty \Any$.
            \item If $x=f$ or $x=s$, we know that $\occ e \varpi$ is a projection, and we can conclude easily given that $\Empty \simeq \pair \Empty \Empty$.
            \item If $x=l$ or $x=r$, we know that $\occ e \varpi$ is a pair, and we can conclude easily given that $\pair \Empty \Any \simeq \pair \Any \Empty \simeq \Empty$.
          \end{itemize}
          Hence we can derive $\Gamma \vdash \occ e \varpi : \Empty$.

          \item[\Rule{PInter}] We must have $v \not\in \valsemantic{t_1 \land t_2}$. It implies $v \not\in \valsemantic{t_1} \cap \valsemantic{t_2}$
          and thus $v \not\in \valsemantic{t_1}$ or $v \not\in \valsemantic{t_2}$. Hence, we can conclude just by applying the induction hypothesis.
          \item[\Rule{PSubs}] Trivial (we use the induction hypothesis).
          \item[\Rule{PEps}] This case is impossible.
          \item[\Rule{PAppL}] We have $v \not\in \valsemantic{\neg (\arrow {t_1} {\neg t_2})}$. Thus, we have $v \in \valsemantic{\arrow {t_1} {\neg t_2}}$
          and in consequence we can derive $\Gamma \vdash v:\arrow {t_1} {\neg t_2}$.

          Recall that $\occ e {\varpi.1}$ is necessarily a value (by hypothesis).
          By using the induction hypothesis on $\pvdash \Gamma e t \varpi.1:t_1$, we can suppose $\occ e {\varpi.1} \in \valsemantic{t_1}$ (otherwise, we can conclude directly).
          Thus, we can derive $\Gamma \vdash \occ e {\varpi.1}: t_1$.

          From $\Gamma \vdash v:\arrow {t_1} {\neg t_2}$ and $\Gamma \vdash \occ e {\varpi.1}: t_1$, we can derive $\Gamma \vdash \occ e {\varpi}: \neg t_2$
          using the rule \Rule{App}.

          Now, by starting from the premise $\pvdash \Gamma e t \varpi:t_2$ and using the rules \Rule{PInter} and \Rule{PTypeof}, we can derive
          $\pvdash \Gamma e t \varpi:\Empty$.
        
          \item[\Rule{PAppR}] We have $v \not\in \valsemantic{\neg t_1}$. Thus, we have $v \in \valsemantic{t_1}$ and in consequence
          we can derive $\Gamma \vdash v:t_1$. 

          Recall that $\occ e {\varpi.0}$ is necessarily a value (by hypothesis).
          By using the induction hypothesis on $\pvdash \Gamma e t \varpi.0:\arrow {t_1} {t_2}$, we can suppose $\occ e {\varpi.0} \in \valsemantic{\arrow {t_1} {t_2}}$ (otherwise, we can conclude directly).
          Thus, we can derive $\Gamma \vdash \occ e {\varpi.0}: \arrow {t_1} {t_2}$.

          From $\Gamma \vdash v:t_1$ and $\Gamma \vdash \occ e {\varpi.0}: \arrow {t_1} {t_2}$, we can derive $\Gamma \vdash \occ e {\varpi}: t_2$
          using the rule \Rule{App}.

          Now, by starting from the premise $\pvdash \Gamma e t \varpi:t_2'$ and using the rules \Rule{PInter} and \Rule{PTypeof}, we can derive
          $\pvdash \Gamma e t \varpi:\Empty$.

          \item[\Rule{PPairL}] We have $v \not\in \valsemantic{t_1}$. Thus, we have $v \in \valsemantic{\neg t_1}$
          and in consequence we can derive $\Gamma \vdash v:\neg t_1$.

          Hence, we can derive $\Gamma \vdash \occ e \varpi:\pair {\neg t_1} \Any$ ($e$ is well-typed in $\Gamma$).
     
          Now, by starting from the premise $\pvdash \Gamma e t \varpi:\pair {t_1} {t_2}$ and using the rules \Rule{PInter} and \Rule{PTypeof},
          we can derive $\pvdash \Gamma e t \varpi:\Empty$.

          \item[\Rule{PPairR}] Similar to the previous case.

          \item[\Rule{PFst}] We have $v \not\in \valsemantic{\pair {t'} \Any}$.
          As we also have $v \in \valsemantic{\pair \Any \Any}$ (because $e$ is well-typed in $\Gamma$),
          we can deduce $v \in \valsemantic{\pair {(\neg t')} \Any}$.

          Hence, we can derive $\Gamma \vdash v:\pair {(\neg t')} \Any$ and then $\Gamma \vdash \occ e \varpi:\neg t'$.
          
          Now, by starting from the premise $\pvdash \Gamma e t \varpi:t'$ and using the rules \Rule{PInter} and \Rule{PTypeof},
          we can derive $\pvdash \Gamma e t \varpi:\Empty$.
          \item[\Rule{PSnd}] Similar to the previous case.
        \end{description}

        \subsection{Value testing}

        As $v$ is a value, the applications of \Rule{Path} have a path $\varpi$ only composed of $l$ and $r$
        and such that $\occ e \varpi$ is a value.

        Thus, any derivation $\pvdash \Gamma v t \varpi:t'$ can only contains the rules
        \Rule{PTypeof}, \Rule{PInter}, \Rule{PSubs}, \Rule{PEps}, \Rule{PPairL} and \Rule{PPairR}.

        Moreover, as $v \in \valsemantic{t}$, the rules \Rule{PEps} can be replaced by a \Rule{PTypeof}.
        Thus we can easily derive $\Gamma \vdash v:t'$ (we replace \Rule{PTypeof} by \Rule{Typeof},
        \Rule{PInter} by \Rule{Inter}, etc.).

        \subsection{Substitution lemma}

        Let $\Gamma$, $e_a$, $e_b$ as in the statement.

        We note $\rho$ the substitution $\subst {e_a} {e_b}$.

        We consider a derivation of $\Gamma \vdash e:t$.

        By using the value refinement lemma, we can assume without loss of generality that our derivation does not contain
        any rule \Rule{Path} on a path $\varpi.x$ such that $\forall y.\ \occ e {\varpi.y}$ refers to a value.

        We can also assume w.l.o.g. that every application of the \Rule{Path} rule is such that $\Gamma',(\occ e \varpi:t') \leq \Gamma'$. If it is not the case,
        we can easily transform the derivation by intersecting $t'$ with $\Gamma'(\occ e \varpi)$
        using the rules \Rule{PInter}, \Rule{PTypeof} and \Rule{Env}.
        The rest of the derivation can easily be adapted by adding some \Rule{Subs} rules when needed.

        Finally, we can assume that, in any environment appearing in the derivation, if the environnement is not bottom,
        then a value $v$ can only be mapped to a type $t$ such that $v\in\valsemantic{t}$. If it is not the case, then we just have to change the
        \Rule{Path} rule that introduce $(v:t)$ into a path rule that introduce $(v:\Empty)$,
        by using the rules \Rule{PInter} and \Rule{PTypeof} (if $v\not\in\valsemantic{t}$, then $v\in\valsemantic{\neg t}$
        and thus $\Gamma \vdash v:\neg t$ is derivable).

        Now, let's prove by induction on the derivation the following properties:

        \begin{align*}
          &\forall e,t.\ \Gamma \vdash e:t \Rightarrow \Gamma\rho \vdash e\rho:t\\
          &\forall e,t,\Gamma'.\ \Gamma \evdash {e} {t} \Gamma' \Rightarrow \Gamma\rho \evdash {e\rho} {t} {\Gamma'}\rho
          \text{ and we still have } \forall t.\ \Gamma' \vdash e_a : t \Rightarrow \Gamma'\rho \vdash e_b:t\\
          &\forall e,t,\varpi,t' \text{ s.t. $\occ {e\rho} \varpi$ is defined}.\ \pvdash \Gamma e t \varpi:t' \Rightarrow \pvdash {\Gamma\rho} {e\rho} t \varpi:t'
        \end{align*}

        We proceed by case analysis on the last rule of the derivation at the left of the $\Rightarrow$ in order to construct the derivation at the right.
        
        If the last judgement is of the form $\Gamma \vdash e_a: t$, then we can directly conclude with the hypotheses of the lemma.
        Thus, we can suppose it is not the case.

        There are many cases depending on the last rule:

        \begin{description}
          \item[\Rule{Env}] If $e\in\dom\Gamma$, then we have $e\rho\in\dom{\Gamma\rho}$ and $(\Gamma\rho)(e\rho)\leq\Gamma(e)$.
          Thus we can easily derive $\Gamma\rho\vdash e\rho:t$ with the rule \Rule{Env} and \Rule{Subs}.
          \item[\Rule{Efq}] If there exists $e\in\dom\Gamma$ such that $\Gamma(e)=\Empty$, then $(\Gamma\rho)(e\rho)=\Empty$
          so we can easily derive $\Gamma\rho\vdash e\rho:t$ with the rule \Rule{Efq}.
          \item[\Rule{Inter}] Trivial (by using the induction hypothesis).
          \item[\Rule{Subs}] Trivial (by using the induction hypothesis).
          \item[\Rule{Const}] In this case, $c\rho = c$ (because $c \neq e_a$). Thus it is trivial.
          \item[\Rule{App}] We have $(e_1 e_2)\rho = (e_1\rho) (e_2\rho)$ (because $e_1 e_2 \neq e_a$).
          Thus we can directly conclude by using the induction hypothesis.
          \item[\Rule{Abs+}] We have $(\lambda^{t'}x.e)\rho = \lambda^{t'}x.(e\rho)$ (because $\lambda^{t'}x.e \neq e_a$).
          
          By alpha-renaming, we can suppose that the variable $x$ is a new fresh variable that does not appear
          in $e_a$ nor $e_b$ ($e_b$ is closed).
          
          We can thus use the induction hypothesis on all the judgements $\Gamma, x:s_i \vdash e:t_i$.
          \item[\Rule{Abs-}] Trivial (by using the induction hypothesis).
          \item[\Rule{Proj}] We have $(\pi_i e)\rho = \pi_i (e\rho)$ (because $\pi_i e \neq e_a$).
          Thus we can directly conclude by using the induction hypothesis.
          \item[\Rule{Pair}] We have $(e_1,e_2)\rho = (e_1\rho,e_2\rho)$ (because $(e_1,e_2) \neq e_a$).
          Thus we can directly conclude by using the induction hypothesis.
          \item[\Rule{Case}]
          We have $(\ite e {t_{if}} {e_1} {e_2})\rho = \ite {e\rho} {t_{if}} {e_1\rho} {e_2\rho}$ (because $\ite e {t_{if}} {e_1} {e_2} \neq e_a$).

          We apply the induction hypothesis on the judgements $\Gamma\vdash e:t_0$ and $\Gamma\evdash e {t_{if}} \Gamma_1$.
          We get $\Gamma\rho\vdash e\rho:t_0$, $\Gamma\rho\evdash {e\rho} {t_{if}} \Gamma_1\rho$ and
          $\forall t'.\ \Gamma_1 \vdash e_a : t' \Rightarrow \Gamma_1\rho \vdash e_b:t'$.
          Now, we can apply the induction hypothesis on $\Gamma_1\vdash e_1:t$ and we have $\Gamma_1\rho\vdash e_1\rho:t$.

          We proceed similarly on the judgments $\Gamma\evdash e {\neg t_{if}} \Gamma_2$ and $\Gamma_2\vdash e_2:t$, and so we have all the premises
          to apply the \Rule{Case} rule in order to get $\Gamma\rho \vdash \ite {e\rho} {t_{if}} {e_1\rho} {e_2\rho}:t'$. 

          \item[\Rule{Base}] Trivial.
          \item[\Rule{Path}] We have by using the induction hypothesis $\Gamma\rho \evdash {e\rho} t \Gamma'\rho$
          and $\forall t''.\ \Gamma' \vdash e_a : t'' \Rightarrow \Gamma'\rho \vdash e_b:t''$.
          
          First, let's show that we can derive $\Gamma\rho \evdash {e\rho} {t} {\Gamma''}\rho$ with $\Gamma''=\Gamma',(\occ e \varpi:t')$.
          
          There are two cases:
          \begin{itemize}
            \item $\occ e \varpi$ is a strict subexpression of $e_a$.
            
            In this case, it means that among its three possible forms,
            $e_a$ is of the form $v v$ or $(v,v)$.
            According to the assumptions we made on the derivation at the beginning of the proof,
            it implies that $\varpi=\epsilon$.
            Hence, $e$ does not contain any occurence of $e_a$, so it is easy to conclude.

            \item $\occ e \varpi$ is not a strict subexpression of $e_a$.
            
            In this case, we know that $\occ {e\rho} \varpi$ is defined.

            Thus we can apply the induction hypothesis on $\pvdash {\Gamma'} {e} t \varpi:t'$.
            It gives $\pvdash {\Gamma'\rho} {e\rho} t \varpi:t'$.
            If $\occ {e\rho}\varpi \in\dom{\Gamma'\rho}$, and $(\Gamma'\rho)(\occ {e\rho}\varpi)=t''\not\geq t'$,
            then we can derive $\pvdash {\Gamma'\rho} {e\rho} t \varpi:t'\land t''$ just by using the rules
            \Rule{PInter}, \Rule{PTypeof} and \Rule{Env}.

            Using this last judgement together with $\Gamma \evdash e t \Gamma'$, we can derive with the rule \Rule{Path}
            the wanted $\Gamma\rho \evdash {e\rho} {t} {\Gamma''}\rho$.
          \end{itemize}

          Now, let's show that $\forall t'.\ \Gamma'' \vdash e_a : t' \Rightarrow \Gamma''\rho \vdash e_b:t'$.

          Let $t'$ such that $\Gamma''\vdash e_a:t'$.

          Recall that we have $\Gamma' \vdash e_a : t' \Rightarrow \Gamma'\rho \vdash e_b:t'$.

          If $\Gamma'' = \bot$, then $\Gamma''\rho = \bot$ so we are done. So lets's suppose $\Gamma'' \neq \bot$.

          Let's separate the proof in two cases:
          \begin{itemize}
            \item If $\occ e \varpi \not\equiv e_a$. In this case, let's show that we have  $\Gamma'\vdash e_a:t'$.
            Indeed, in the typing derivation of $\Gamma''\vdash e_a:t'$, the \Rule{Env} rules can only be applied on
            subexpressions of $e_a$.
            
            If $\occ e \varpi$ is not a strict subexpression of $e_a$
            (and thus not a subexpression as $\occ e \varpi \not\equiv e_a$), there is no \Rule{Env} rule applied to $\occ e \varpi$
            in the derivation of $\Gamma''\vdash e_a:t'$ and thus we can easily derive $\Gamma'\vdash e_a:t'$.

            If $\occ e \varpi$, is a strict subexpression of $e_a$, it must be a value (given the possible forms of $e_a$).
            Moreover, as $\Gamma'' \neq \bot$, we have $\forall v\in\dom{\Gamma''}.\ v\in\valsemantic{\Gamma''(v)}$ (recall the assumptions at the beginning of the proof)
            and thus $\forall v\in\dom{\Gamma''}.\ \Gamma'\vdash v : \Gamma''(v)$.
            Thus we can derive $\Gamma'\vdash e_a:t'$ just by replacing every \Rule{Env} rule applied to $\occ e \varpi$ in the derivation of $\Gamma''\vdash e_a:t'$
            by the relevant derivation.

            From $\Gamma'\vdash e_a:t'$ we deduce $\Gamma'\rho\vdash e_b:t'$.
            As $\Gamma''\leq\Gamma'$ (according to the assumptions we made on the derivation at the beginning of the proof)
            and $\dom{\Gamma'}\subseteq\dom{\Gamma''}$, we have $\Gamma''\rho\leq\Gamma'\rho$ and thus, by monotonicity,
            $\Gamma''\rho\vdash e_b:t'$.

            \item            
            If $\occ e \varpi \equiv e_a$. Let's note $t_a=\Gamma''(e_a)$. This time,
            we can't derive $\Gamma'\vdash e_a:t'$ from $\Gamma''\vdash e_a:t'$ because the rule \Rule{Env}
            could be used on $\occ e \varpi=e_a$ (which may not be a value).

            However, the rule \Rule{Env} can only be used on $e_a$ at the end of the derivation of $\Gamma''\vdash e_a:t'$:
            there can't be any \Rule{App}, \Rule{Abs+}, \Rule{Proj}, \Rule{Pair} or \Rule{Case} after because the premises of these rules only contain strict subexpressions of their
            consequence.
            Thus, we can easily transform the derivation so that every \Rule{Env} applied on $e_a$ is directly followed by an \Rule{Inter}:
            if there is any \Rule{Abs-} or \Rule{Subs} between, we can move it after.

            Then, we can (temporarily) remove from the derivation all \Rule{Env} applied on $e_a$:
            for each, we just replace the following \Rule{Inter} rule by its other premise.

            It yields a derivation for $\Gamma'' \vdash e_a : t''$ such that $t''\land t_a \leq t'$ and without any \Rule{Env} applied to $e_a$.
            Thus, we can transform it into a derivation of $\Gamma' \vdash e_a : t''$ as in the previous point, and we get $\Gamma'\rho \vdash e_b : t''$.
            Still as before, we get a derivation for $\Gamma''\rho\vdash e_b : t''$ by monotonicity.
            
            Now, we can append at the end of this derivation a rule \Rule{Inter} with a rule \Rule{Env} applied to $e_b$.
            As $(\Gamma''\rho)(e_b) \leq \Gamma''(e_a) = t_a$, we obtain a derivation for $\Gamma''\rho\vdash e_b : t'$ (we can add a final \Rule{Subs} rule if needed).
          \end{itemize}
          
          \item[\Rule{PTypeof}] Trivial (by using the induction hypothesis).
          \item[\Rule{P$\cdots$}] All the remaining rules are trivial.
        \end{description}

        \subsection{Subject reduction}

        Let $\Gamma$, $e$, $e'$ and $t$ as in the statement.

        We construct a derivation for $\Gamma \vdash e':t$ by induction on the derivation of $\Gamma \vdash e:t$.

        If $\Gamma = \bot$ this theorem is trivial, so we can suppose $\Gamma \neq \bot$.

        We proceed by case analysis on the last rule of the derivation:
        
        \begin{description}
          \item[\Rule{Env}] As $\Gamma$ is ordinary, it means that $e$ is a variable.
          It contradicts the fact that $e$ reduces to $e'$ so this case is impossible.
          \item[\Rule{Efq}] This case is impossible as $\Gamma \neq \bot$.
          \item[\Rule{Inter}] Trivial (by using the induction hypothesis).
          \item[\Rule{Subs}] Trivial (by using the induction hypothesis).
          \item[\Rule{Const}] Impossible case (no reduction possible).
          \item[\Rule{App}] In this case, $e\equiv e_1 e_2$. There are three possible cases:
          \begin{itemize}
            \item $e_2$ is not a value. In this case, we must have $e_2\leadsto e_2'$
            and $e'\equiv e_1 e_2'$. We can easily conclude using the induction hypothesis.
            \item $e_2$ is a value and $e_1$ is not. In this case, we must have $e_1\leadsto e_1'$
            and $e'\equiv e_1' e_2$. We can easily conclude using the induction hypothesis.
            \item Both $e_1$ and $e_2$ are values. This is the difficult case.
            We have $e_1\equiv \lambda^{\bigwedge_{i\in I}\arrow{s_i}{t_i}}x.e_x$
            with $\bigwedge_{i\in I}\arrow{s_i}{t_i} \leq \arrow{s}{t}$ and $\Gamma \vdash e_2:s$.
            We can suppose that $x$ is a new fresh variable that does not appear in our environment
            (if it is not the case, we can alpha-rename $e_1$).

            This means that $s\leq \bigvee_{i\in I} s_i$ and that for any non-empty $I'$ such that
            $s\not\leq \bigvee_{i\in I\setminus I'} s_i$, we have $\bigwedge_{i\in I'} t_i \leq t$
            (TODO: ref lemma 6.8 semantic subtyping). Let's take $I'=\{i\in I\alt \vvdash e_2:s_i\}$.
            We have $I'$ not empty: $\vvdash e_2:s$ and $s\leq \bigvee_{i\in I} s_i$, so according to
            $\valsemantic \_$ properties we have at least one $i$ such that $\vvdash e_2:s_i$.
            We also have $s\not\leq \bigvee_{i\in I\setminus I'} s_i$, otherwise there would be a $i\not\in I$
            such that $\vvdash e_2: s_i$ (contradiction with the definition of $I'$).
            As a consequence, we get $\bigwedge_{i\in I'} t_i \leq t$.

            Now, let's prove that $\Gamma \vdash e':\bigwedge_{i\in I'} t_i$ (which, by subsumption,
            yields $\Gamma \vdash e': t$). For that, we show that for any $i\in I'$, $\Gamma \vdash e':t_i$
            (it is then easy to conclude by using the \Rule{Inter} rule).

            Let $i\in I'$. We have $\vvdash e_2:s_i$, and so $\Gamma \vdash e_2:s_i$ ($e_2$ is well-typed in $\Gamma$).
            As $e_1$ is well-typed in $\Gamma$, there must be in its derivation an application of the rule \Rule{Abs+}
            which guarantees $\Gamma,(x:s_i) \vdash e_x:t_i$ (recall that $\Gamma \neq \bot$ and $\Gamma$ is ordinary so there is no abstraction in $\dom\Gamma$).
            Let's note $\Gamma'=\Gamma,(x:s_i)$.
            We can deduce, using the substitution lemma, that $\Gamma'\subst x {e_2} \vdash e_x\subst x {e_2}: t_i$.
           
            Moreover, $\Gamma'\subst x {e_2} = \Gamma,(e_2:s_i)$ and $\Gamma \leq \Gamma,(e_2:s_i)$.
            Thus, by monotonicity, we deduce $\Gamma \vdash e_x\subst x {e_2}: t_i$,
            that is $\Gamma \vdash e': t_i$.
          \end{itemize}
          \item[\Rule{Abs+}] Impossible case (no reduction possible).
          \item[\Rule{Abs-}] Impossible case (no reduction possible).
          \item[\Rule{Proj}] In this case, $e\equiv \pi_i e_0$. There are two possible cases:
          \begin{itemize}
            \item $e_0$ is not a value. In this case, we must have $e_0\leadsto e_0'$
            and $e'\equiv \pi_i e_0'$. We can easily conclude using the induction hypothesis.
            \item $e_0$ is a value.
            Given that $e_0 \leq \pair \Any \Any$, we have $e_0 = (v_1,v_2)$ with $v_1$ and $v_2$ two values.
            We also have $e \leadsto v_i$.

            The derivation of $\Gamma \vdash (v_1,v_2): \pair {t_1} {t_2}$ must contain a rule \Rule{Pair}
            which guarantees $\Gamma \vdash v_i: t_i$ (recall that $\Gamma \neq \bot$ and $\Gamma$ is ordinary so there is no pair in $\dom\Gamma$).
            It concludes this case.
          \end{itemize} 
          \item[\Rule{Pair}] In this case, $e\equiv (e_1,e_2)$. There are two possible cases:
          \begin{itemize}
            \item $e_2$ is not a value. In this case, we must have $e_2\leadsto e_2'$
            and $e'\equiv (e_1, e_2')$. We can easily conclude using the induction hypothesis.
            \item $e_2$ is a value and $e_1$ is not. In this case, we must have $e_1\leadsto e_1'$
            and $e'\equiv (e_1', e_2)$. We can easily conclude using the induction hypothesis.
          \end{itemize}
          \item[\Rule{Case}] In this case, $e\equiv \ite {e_0} {t_{if}} {e_1} {e_2}$. There are three possible cases:
          \begin{itemize}
            \item $e_0$ is a value and $e_0 \in \valsemantic{t_{if}}$. In this case we have $e' \equiv e_1$.
            We have derivations for $\Gamma \vdash e_0: t_0$, $\Gamma \evdash {e_0} {t_{if}} \Gamma'$ and $\Gamma'\vdash e_1:t$.
            
            As $e_0$ is a value and $e_0 \in \valsemantic{t_{if}}$, we have $\Gamma\leq\Gamma'$ by using the value testing lemma.
            Thus, by monotonicity, we have $\Gamma\vdash e_1:t$.
            \item $e_0$ is a value and $e_0 \not\in \valsemantic{t}$. This case is similar to the previous one (we replace $t_{if}$ by $\neg t_{if}$ and $e_1$ by $e_2$).
            \item $e_0$ is not a value.
            In this case, we have $e_0\xleadsto{e_a\mapsto e_b} e_0'$ and $e'\equiv \ite {e_0\rho} {t_{if}} {e_1\rho} {e_2\rho}\equiv e\rho$
            with $\rho = \subst{e_a}{e_b}$.
            
            First, let's notice that we have $e_b$ closed (only closed expressions are reducible),
            and $e_a$ has one of the following forms:
            \begin{itemize}
              \item $\ite e t {e_1} {e_2}$ (if expression)
              \item $v v$ (application of two values)
              \item $(v,v)$ (product of two values)
            \end{itemize}
            It can be easily proved by induction on the derivation of the reduction step.

            Secondly, as $e_a\leadsto e_b$ and as the derivation of this reduction is a strict subderivation of that of $e\leadsto e'$,
            we can use the induction hypothesis on $e_a\leadsto e_b$ and we obtain $\forall t'.\ \Gamma \vdash e_a : t' \Rightarrow \Gamma\rho \vdash e_b:t'$.
            
            Thus, we can conclude directly by using the substitution lemma on $e$ and $\rho$.
          \end{itemize}
        \end{description}

\end{document}