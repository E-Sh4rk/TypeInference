\documentclass[a4paper]{article}

\usepackage{setup}

\hypersetup{pdfstartview=XYZ}%         zoom par defaut

\setlength{\droptitle}{-5em}   % This is your set screw
\title{\vspace{1.5cm}Type inference - M2 Internship}
\author{Mickael LAURENT}
\date{\vspace{-5ex}}

\pagenumbering{gobble}

\theoremstyle{definition}
\newtheorem{theorem}{Theorem}
\newtheorem{lemma}{Lemma}
\newtheorem{definition}{Definition}
\newtheorem{property}{Property}
\newtheorem{corollary}{Corollary}

\begin{document}

	\maketitle

    \section{New typing system}

    \[
      \begin{array}{lrcl}
      \textbf{Types} & t & ::= & b\alt \arrow t t\alt t\vee t \alt \neg t \alt \Empty\\
      \textbf{Atoms} & t_A & ::= & b\alt \arrow t t\\
      \textbf{Non-Functional Types} & t_b & ::= & b \alt t_b\vee t_b \alt \neg t_b \alt \Empty\\
      \textbf{Normal Types} & \hat t & ::= & t_b \alt \arrow {\hat t} {\hat t}\alt \hat t \wedge \hat t \alt \hat t\vee \hat t
      \end{array}
    \]

    \[
      \norm t = \text{remove negative arrows from DNF}
    \]

    \[
    \begin{array}{lcl}
      \typep+\epsilon{\Gamma,e,t} & = & t\\
      \typep-\epsilon{\Gamma,e,t} & = & \neg t\\
      \typep{p}{\varpi.0}{\Gamma,e,t} & = & \neg(\arrow{\Gp p{\Gamma,e,t}{(\varpi.1)}}{\neg \Gp p {\Gamma,e,t} (\varpi)})\\
      \typep{p}{\varpi.1}{\Gamma,e,t} & = & \worra{\tyof{\occ e{\varpi.0}}\Gamma}{\Gp p {\Gamma,e,t} (\varpi)}\\ \\
      \Gp p {\Gamma,e,t} (\varpi) & = & \norm {\typep p \varpi {\Gamma,e,t} \land \tyof {\occ e \varpi} \Gamma
      \underbrace{\land \tyof {\occ e \varpi} {\Gamma\setminus\{\occ e \varpi\}}}_\text{if $\occ e \varpi$ is not a variable}}\\
    \end{array}
    \]

    \begin{align*}
      &(\Refine p {(e,t)} \Gamma)(e') = 
        \left\{\begin{array}{ll}
          \bigwedge_{\{\varpi \alt \occ e \varpi \equiv e'\}} \Gp p {\Gamma,e,t} (\varpi) & \text{if } \exists \varpi.\ \occ e \varpi \equiv e' \\
          \Gamma(e') & \text{otherwise, if } e' \in \dom \Gamma\\
          \text{undefined} & \text{otherwise}
        \end{array}\right.\\&\\
      &\Gaux p {\varnothing} \Gamma = \Gamma\\
      &\Gaux p {\cons {(e,t)} {\vec i}} \Gamma = \Gaux p {\vec i} {\Refine p {(e,t)} \Gamma}\\&\\
      &\Genv p {\vec {i}} \Gamma=\fixpoint_\Gamma (\Gaa p {\vec {i}})
    \end{align*}

    \section{New (intermediate) semantics}

    \[
      \begin{array}{lrcl}
      \textbf{Expressions} & e & ::= & c\alt x \alt \lambda^{\bigwedge \arrow {\hat t} {\hat t}}x.e \alt \ite e {\hat t} e e \alt e e\\
      \textbf{Values} & v & ::= & c \alt \lambda^{\bigwedge \arrow {\hat t} {\hat t}}x.e
      \end{array}
    \]

    \[
      \begin{array}{lrcl}
      \textbf{Context} & C[] & ::= & e [] \alt [] v
      \end{array}
    \]

    For convenience, we denote $E\xleadsto{E\mapsto E'}E'$ by $E\idleadsto E'$.

    \begin{mathpar}
      \Infer[Ctx]
      {e \xleadsto{E\mapsto E'} e'}
      {C[e] \xleadsto{E\mapsto E'} C[e']}
      {}
      \qquad
      \Infer[App]
      { }
      {(\lambda^tx.e) v \idleadsto e\subst x v}
      {}\\
      \Infer[TestCtx]
      {e \xleadsto{E\mapsto E'} e'}
      {\ite e t {e_1} {e_2} \idleadsto \ite {e'} t {e_1\subst E {E'}} {e_2\subst E {E'}}}
      {}\\
      \Infer[Ite]
      {i \in \{1,2\}\hspace{1cm} (i=1 \Leftrightarrow v \in \valsemantic t)}
      {\ite v t {e_1} {e_2} \idleadsto e_i}
      {}
    \end{mathpar}

    \[\valsemantic t = \{v \alt \vvdash v : t\}\]

    \begin{mathpar}
      \Infer[Subsum]
          {\vvdash v:t'\\t'\leq t}
          {\vvdash v:t}
          {}
      \qquad
      \Infer[Const]
          { }
          {\vvdash c:\basic{c}}
          {}
      \\
      \Infer[Abs]
          {t=(\wedge_{i\in I}\arrow {s_i} {t_i})\land (\wedge_{j\in J} \neg (\arrow {s'_j} {t'_j}))\\t\not\leq \Empty}
          {\vvdash\lambda^{\wedge_{i\in I}\arrow {s_i} {t_i}}x.e:t}
          {}
      \\
      \end{mathpar}

    TODO: Add a certain condition on the type of lambda abstractions (and tests) in order to make the app case to work.

    \subsection{Recall: Type System}

    \begin{mathpar}
      
      \Infer[Occ]
          {
          }
          { \Gamma \vdash e: \Gamma(e) }
          { e\in\dom\Gamma}
      \qquad
      \Infer[Const]
          { }
          {\Gamma\vdash c:\basic{c}}
          {c\not\in\dom\Gamma}
       \\
      \Infer[Abs]
          {\Gamma,x:s_i\vdash e:t_i'\\t_i'\leq t_i}
          {
          \Gamma\vdash\lambda^{\wedge_{i\in I}\arrow {s_i} {t_i}}x.e:\textstyle\bigwedge_{i\in I}\arrow {s_i} {t_i}
          }
          {\lambda^{\wedge_{i\in I}\arrow {s_i} {t_i}}x.e\not\in\dom\Gamma}
          \\
      \Infer[App]
          {
            \Gamma \vdash e_1: t_1 \\
            \Gamma\vdash e_2: t_2\\
            t_1 \leq \arrow \Empty \Any\\
            t_2 \leq \dom {t_1}
          }
          { \Gamma \vdash {e_1}{e_2}: t_1 \circ t_2 }
          { {e_1}{e_2}\not\in\dom\Gamma}
          \\
        \Infer[If]
          {\Gamma\vdash e:t_0\\
            \left\{
              \makebox{$\begin{array}{ll} %\Gamma,
              \Genv + {e,t} \Gamma \vdash e_1 : t_1 & \text{ if } t_0 \not\leq \neg t\\
              t_1 = \Empty & \text{ otherwise}
            \end{array}$}\right.\\
            \left\{
              \makebox{$\begin{array}{ll} %\Gamma,
              \Genv - {e,t} \Gamma \vdash e_2 : t_2 & \text{ if } t_0 \not\leq t\\
              t_2 = \Empty & \text{ otherwise}
            \end{array}$}\right.}
          {\Gamma\vdash \ite {e} t {e_1}{e_2}: t_1\vee t_2}
          %{\ite {e} t {e_1}{e_2}\not\in\dom\Gamma}
          {\texttt{if}\dots\not\in\dom\Gamma}
        \\
      \end{mathpar}

      \subsection{Definitions}
      
      \begin{definition}[Well formed environment]
        An environment $\Gamma$ is well-formed iff:
        \begin{align*}
          &\forall v \in \dom \Gamma \text{ a value. } v \in \valsemantic {\Gamma(v)}\\
          &\forall e \in \dom \Gamma \text{ not a variable. } \exists t.\ \Gamma\setminus(e:\Gamma(e)) \vdash e : t \text{ and } \Gamma(e) \leq t
        \end{align*}
        In particular, this last property implies:
        \[
          \forall e \in \dom \Gamma.\ \forall t.\ \Gamma\setminus(e:\Gamma(e)) \vdash e : t \Rightarrow \Gamma(e) \leq t
        \]
      \end{definition}

      \begin{definition}[Environment equivalence]
        Two well-formed environments $\Gamma$ and $\Gamma'$ are equivalent (noted $\Gamma \equiv \Gamma'$) iff:
        \begin{align*} 
          &\forall e \in \dom \Gamma.\ \Gamma(e)=t \Rightarrow \Gamma' \vdash e:t \hspace{1cm} \text{and} \hspace{1cm}
          \forall e' \in \dom {\Gamma'}.\ \Gamma(e')=t' \Rightarrow \Gamma \vdash e':t' \\
          \text{and} \hspace{1cm} &\exists e\in \dom \Gamma.\ \Gamma(e)=\Empty \Leftrightarrow \exists e'\in \dom {\Gamma'}.\ \Gamma'(e')=\Empty
        \end{align*}
      \end{definition}
    
      \begin{definition}[Environment inclusion]
        Let $\Gamma$ and $\Gamma'$ two well-formed environments. We says that $\Gamma' \leq \Gamma$ iff:
        \begin{align*}
          &\exists e\in \dom \Gamma.\ \Gamma(e)=\Empty \Rightarrow \exists e'\in \dom {\Gamma'}.\ \Gamma'(e')=\Empty\\
          &\forall e \in \dom \Gamma.\ \Gamma' \vdash e : t \text{ with } t \leq \Gamma(e)
        \end{align*}
        Note: this relation is a preorder (the transitivity can easily be deduced from the monotonicity lemma).
        We also have: $\Gamma' \leq \Gamma \text{ and } \Gamma \leq \Gamma' \Leftrightarrow \Gamma \equiv \Gamma'$.
      \end{definition}
    
      \begin{definition}[Environment substitution]
        Let $\Gamma$ an environment and $\rho$ a substitution from variables to expressions.
        The environment $\Gamma\rho$ is defined by:
        \begin{align*}
          &\dom {\Gamma\rho} = \dom \Gamma \rho\\
          &\forall e' \in \dom {\Gamma\rho}, (\Gamma\rho)(e') = \bigwedge_{e'' \in \dom \Gamma \text{ s.t. } e''\rho=e'}\Gamma(e'')
        \end{align*}
      \end{definition}
    
      \begin{definition}[Ordinary environments]
        We say that an environment $\Gamma$ is ordinary iff its domain only contains variables.
      \end{definition}
    
        \subsection{Theorems}

        TODO: "Normalization" of types?

        \begin{property}[$\valsemantic \_$ properties]
          \begin{align*}
            &\forall s.\ \forall t.\ \valsemantic s \subseteq \valsemantic t \Leftrightarrow s \leq t\\
            &\valsemantic \Empty = \varnothing\\
            &\forall t.\ \valsemantic {\neg t} = \values \setminus \valsemantic t\\
            &\forall s.\ \forall t.\ \valsemantic {s\vee t} = \valsemantic s \cup \valsemantic t
          \end{align*}
        \end{property}
        Proof: theorem 5.5, lemmas 6.19, 6.22, 6.23 of semantic_subtyping [TODO: ref].

        \begin{lemma}[$\Ga p {} $ well-formedness]
          Let $\Gamma$ a well-formed environment. Let $e$ a well-typed expression in $\Gamma$, $t_{if}$ a type and $p\in\{-,+\}$.\\
          Thus, $\Genv p {e,t_{if}} \Gamma$ is well defined and $\Gamma,\Genv p {e,t_{if}} \Gamma$ is well-formed.
        \end{lemma}

        \begin{lemma}[Well-formedness preservation]
          Let $\Gamma$ a well-formed environment. Let $\rho$ a substitution.
          Then $\Gamma\rho$ is a well-formed environment.
        \end{lemma}
    
        \begin{lemma}[Monotonicity]
          Let $\Gamma$ and $\Gamma'$ two well-formed environments such that $\Gamma' \leq \Gamma$. Let $e$ an expression such that $\Gamma \vdash e:t$.
          Then:\\
          \begin{align*}
            &\Gamma' \vdash e:t' \text{ with } t' \leq t\\
            &\forall p \forall t' \forall \varpi.\ \Gp p {\Gamma',e,t'} (\varpi) \leq \Gp p {\Gamma,e,t'} (\varpi)
          \end{align*}
        \end{lemma}
    
        \begin{corollary}[Equivalence]
          If $\Gamma \vdash e:t$ and $\Gamma' \equiv \Gamma$, then $\Gamma' \vdash e:t$.
        \end{corollary}
    
        \begin{corollary}[Strengthening]
          If $\Gamma, (x:t_1^x) \vdash e:t_1$ and $\Gamma, (x:t_2^x) \vdash e:t_2$, then
          $\Gamma, (x:t_1^x\land t_2^x) \vdash e:t$ with $t \leq t_1\land t_2$.
        \end{corollary}
    
        \begin{lemma}[Substitution]
          Let $\Gamma$ a well-formed environment, $e$, $e_a$ and $e_b$ some expressions, $t$, $t_a$ and $t_b$ somes types.
          If $e \in \dom \Gamma$, $\Gamma \vdash e:t$ and $\Gamma \vdash e_a : t_a$ and $\Gamma\subst {e_a} {e_b} \vdash e_b:t_b$ with $t_b \leq t_a$, then:
          \begin{align*}
            &\Gamma \subst {e_a} {e_b} \vdash e \subst {e_a} {e_b}:t' \text{ with } t'\leq t\\
            &\forall p \forall t_{if} \forall \varpi.\ \Gp p {\Gamma\subst {e_a} {e_b},e\subst {e_a} {e_b},t_{if}} (\varpi) \leq \Gp p {\Gamma,e,t_{if}} (\varpi)
          \end{align*}
        \end{lemma}
    
        TODO: beyond this point, statements and proofs must be checked/updated (a lot of changes have been made).

        \begin{corollary}[Substitution 2]
          Let $\Gamma$ an environment, $x$ a variable that does not appear in $\dom \Gamma$, $e$ an expression, $e'$ an expression that does not contain $x$, $t$ a type and $t_x$ a type that is not $\Empty$.\\
          If $\Gamma, (x:t_x) \vdash e:t$ and $\Gamma \vdash e':t_{e'}$ with $t_{e'} \leq t_x$, then $\Gamma \vdash e \subst x {e'}:t'$ with $t'\leq t$.
        \end{corollary}
    
        \begin{lemma}[Test monotonicity]
          Let $\Gamma$ an environment. Let $e$ a well typed expression and $t$, $t_{if}$, $t_{if}'$ some types.
          If $\Gamma \vdash e : t$ and $t_{if}' \leq t_{if}$ then $\Genv + {e,t_{if}'} \Gamma \leq \Genv + {e,t_{if}} \Gamma$
          and $\Genv - {e,t_{if}} \Gamma \leq \Genv - {e,t_{if}'} \Gamma$.
        \end{lemma}
    
        \begin{lemma}[Test reduction]
          Let $\Gamma$ an ordinary environment, $e$ and $e'$ two expressions and $t$, $t_{if}$ two types.
          If $\Gamma \vdash e : t$, $e \leadsto e'$ and
          if the subject reduction lemma holds for every subexpression of $e$ that satisfies its hypotheses,
          then $\forall p.\ \Genv p {e',t_{if}} \Gamma \leq \Genv p {e,t_{if}} \Gamma$.
        \end{lemma}
    
        \begin{theorem}[Subject reduction]
          Let $\Gamma$ an ordinary environment, $e$ and $e'$ two expressions and $t$ a type.
          If $\Gamma \vdash e : t$ and $e \leadsto e'$, then there exists $t' \leq t$ such that $\Gamma \vdash e' : t'$.
        \end{theorem}
    
        \section{Proofs}
    
        \subsection{Monotonicity}
    
        We proceed by induction on $e$. We suppose that, for any $e'$ strict subexpression of $e$, for any $\Gamma' \leq \Gamma$,
        \begin{align*}
          &\Gamma \vdash e':t' \Rightarrow \Gamma' \vdash e':t'' \text{ with } t'' \leq t'\\
          &\Gamma \vdash e':t' \Rightarrow \forall p \forall t'' \forall \varpi.\ \Gp p {\Gamma',e',t''} (\varpi) \leq \Gp p {\Gamma,e',t''} (\varpi)
        \end{align*}
    
        Let $\Gamma$, $\Gamma'$ and $e$ as in the lemma statement.
        If $\exists e' \in \dom {\Gamma'}.\ \Gamma'(e') = \Empty$, then the properties are trivial.
        So let's suppose it is not the case (in particular, we know that the last rule applied to type $\Gamma \vdash e:t$ can't be $\Rule{Empty}$).
    
        First, let's show that $\Gamma' \vdash e:t' \text{ with } t' \leq t$.
    
        If $e\in\dom\Gamma$, we know that $\Gamma(e)=t$ (as $\Gamma \vdash e:t$). The definition of $\Gamma' \leq \Gamma$ suffices to conclude.
        So let's suppose $e\not\in\dom\Gamma$ (in particular, we know that $e$ is not a variable and that the last rule applied to type $\Gamma \vdash e:t$ is not $\Rule{Occ}$).
    
        We can also suppose $e\not\in\dom{\Gamma'}$. If it is not the case, we can just consider $\Gamma'\setminus(e:\Gamma'(e))$ instead of $\Gamma'$
        and obtain the wanted result on $\Gamma'$ by using its well-formedness.

        We now proceed by case analysis on $e$:
        \begin{description}
          \item[$c$] Trivial.
          \item[$x$] Impossible case, as we have already treated the rules $\Rule{Empty}$ and $\Rule{Occ}$.
          \item[$\lambda^{\bigwedge_{i\in I} \arrow {s_i} {t_i}}x.e_x$] We know that the last rule in the typing derivation of $\Gamma \vdash e:t$ is $\Rule {Abs}$.
          We can build a derivation for $\Gamma' \vdash e:t'$ (with $t'\leq t$) with $\Rule {Abs}$ as last rule.
          For that, we just have to notice that $\forall i.\ \Gamma',x:s_i \leq \Gamma,x:s_i$ and we apply the induction hypothesis.
          
          TODO: What if x is in dom(gamma)? It would not be well-formed anymore.

          \item[$e_1\ e_2$] We know that the last rule in the typing derivation of $\Gamma \vdash e:t$ is $\Rule {App}$.
          We can build a derivation for $\Gamma' \vdash e:t'$ (with $t'\leq t$) with $\Rule {App}$ as last rule.
          For that, we just have to apply the induction hypothesis and notice that the $\circ$ operator is monotonic.
          \item[$\ite {e_0} {t_{if}} {e_1} {e_2}$]
          We know that the last rule in the typing derivation of $\Gamma \vdash e:t$ is $\Rule {If}$.

          Let's build a derivation for $\Gamma' \vdash e:t'$ (with $t'\leq t$) with $\Rule {If}$ as last rule.

          By the $\Ga p {}$ well-formedness lemma, we know that $\Genv p {e_0,t_{if}} {\Gamma'}$ is defined and that $\Gamma',\Genv p {e_0,t_{if}} {\Gamma'}$ is well formed.
          Using the induction hypothesis, we get $\Genv p {e_0,t_{if}} {\Gamma'} \leq \Genv p {e_0,t_{if}} \Gamma$ (it follows from the monotonicity of $\Gp {} {}$).
          Thus, we have $\Gamma',\Genv p {e_0,t_{if}} {\Gamma'} \leq \Gamma,\Genv p {e_0,t_{if}} \Gamma$.
          We can conclude by applying the induction hypothesis again.\\
        \end{description}
    
        Now, let's show that $\forall p \forall t' \forall \varpi.\ \Gp p {\Gamma',e,t'} (\varpi) \leq \Gp p {\Gamma,e,t'} (\varpi)$.
    
        Let $p\in \{+,-\}$ and $t'$ a type.
        We proceed by induction on $\varpi$.
        
        The base case is straightforward (as we already proved that $\Gamma' \vdash e:t' \text{ with } t' \leq t$).\\
        For the case $\varpi.1$, we apply the outer induction hypothesis to get $\tyof {\Gamma'} {\occ e {\varpi.0}} \leq \tyof {\Gamma} {\occ e {\varpi.0}}$.
        We apply the induction hypothesis to get $\Gp p {\Gamma',e,t'} (\varpi) \leq \Gp p {\Gamma,e,t'} (\varpi)$. We can then conclude using the monotonicity of $\worra {} {}$.\\
        The case $\varpi.0$ follows from the previous case and the induction hypothesis.
    
        \qed
    
        \subsection{Substitution lemma}
    
        Let $e_a,e_b$ as in the lemma statement. We will denote the substitution $\subst {e_a} {e_b}$ by $\rho$.
    
        We proceed by induction on $e$. We suppose that, for any $e_s$ strict subexpression of $e$,
        \begin{align*}
          \forall \Gamma \forall t_s.\ &\Gamma \vdash e_s:t_s \Rightarrow \exists t'.\ \Gamma \rho \vdash e_s \rho:t' \text{ with } t'\leq t_s\\
          \forall \Gamma \forall t_s.\ &\Gamma \vdash e_s:t_s \Rightarrow \forall p \forall t_{if} \forall \varpi.\ \Gp p {\Gamma\rho,e_s\rho,t_{if}} (\varpi) \leq \Gp p {\Gamma,e_s,t_{if}} (\varpi)
        \end{align*}
    
        Let $\Gamma,t,t_a,t_b$ as in the lemma statement.
    
        If $\exists e_\bot \in \dom {\Gamma}.\ \Gamma(e_\bot) = \Empty$, then $(\Gamma\rho)(e_\bot\rho) = \Empty$ and thus the properties are trivial.
        So let's suppose it is not the case (in particular, we know that the last rule applied to type $\Gamma \vdash e:t$ can't be $\Rule{Empty}$).
    
        First, let's show that $\Gamma \rho \vdash e \rho:t'$ with $t'\leq t$.
    
        If $e\in\dom\Gamma$, we know that $e\rho\in\dom{\Gamma\rho}$ and $(\Gamma\rho)(e\rho) \leq \Gamma(e)$.
        As $\Gamma(e) = t$ (because $\Gamma \vdash e:t$), we can derive the wanted result using the rule $\Rule {Occ}$.
        So let's suppose now that $e\not\in\dom\Gamma$ (in particular, we know that the last rule applied to type $\Gamma \vdash e:t$ is not $\Rule{Occ}$).
        
        TODO: What if $e\rho\in\dom{\Gamma\rho}$ ? 

        We now proceed by a case analysis on $e$:
        
        \begin{description}
          \item[$c$] Trivial.
          \item[$x$] Impossible case, as we have already treated the rules $\Rule{Empty}$ and $\Rule{Occ}$.
          \item[$\lambda^{\bigwedge_{i\in I} \arrow {s_i} {t_i}}y.e_y$]
          We know that the last rule in the typing derivation of $\Gamma \vdash e:t$ is $\Rule {Abs}$.
          We have, for each $i \in I$, $\Gamma,y:s_i \vdash e_y:t_i'$ with $t_i'\leq t_i$.
          
          TODO: What if y is in dom(gamma)? It would not be well-formed anymore.

          By induction hypothesis, for each $i$, we get  $\Gamma\rho,y:s_i \vdash e_y\rho:t_i''$ with $t_i''\leq t_i'\leq t_i$. We conclude by applying $\Rule {Abs}$.
          \item[$e_1\ e_2$] We know that the last rule in the typing derivation of $\Gamma \vdash e:t$ is $\Rule {App}$.
          We have $\Gamma\vdash e_1:t_1$ and $\Gamma\vdash e_2:t_2$ with $t_1$ an arrow type and $t_2 \in \dom {t_1}$.
          By induction hypothesis, we get $\Gamma\rho\vdash e_1\rho:t_1'$ with $t_1' \leq t_1$ and $\Gamma\rho\vdash e_2\rho:t_2'$ with $t_2' \leq t_2$.
          We conclude by applying $\Rule {App}$ and noticing that the operator $\circ$ is monotonic.
          \item[$\ite {e_0} {t_{\text{if}}} {e_1}{e_2}$] We know that the last rule in the typing derivation of $\Gamma \vdash e:t$ is $\Rule {If}$.
          We have $\Gamma\vdash e_0:t_0$, $\Gamma,\Genv + {e_0,t_{\text{if}}} \Gamma\vdash e_1 : t_1$ and $\Gamma,\Genv - {e_0,t_{\text{if}}} \Gamma\vdash e_2 : t_2$.
          \begin{itemize}
            \item By induction hypothesis, we directly get $\Gamma\rho\vdash e_0\rho:t_0'$ with $t_0'\leq t_0$.
            \item Let $\Gamma_1 = \Gamma,\Genv + {e_0,t_{\text{if}}} \Gamma$.
            %Let $\Gamma_1'=\Gamma_1,(e_a:\tyof {e_a} {\Gamma_1}), (e_b:\tyof {e_a} {\Gamma_1}\land\tyof {e_b} {\Gamma_1}
            %\land\bigwedge_{\{e'\in \dom {\Gamma_1}\alt e'\rho=e_b\}} \Gamma_1(e')
            %)$.
            
            %In particular, $\Gamma_1'(e_b)\leq \tyof {e_a} {\Gamma_1} = \Gamma_1'(e_a)$.
            $(\Gamma_1\rho)(e_b) \leq (\Gamma\rho)(e_b) \leq \Gamma(e_a) \leq (\Gamma_1)(e_a)$ ?? TODO
            
            %Moreover, we have $\Gamma_1' \leq \Gamma_1$.
            %As $\Gamma_1 \vdash e_1:t_1$, we have by using the monotonicity lemma $\Gamma_1' \vdash e_1:t_1'$ with $t_1'\leq t_1$.
    
            %Thus, we get by induction hypothesis $\Gamma_1'\rho\vdash e_1\rho:t_1'$ with $t_1' \leq t_1$.
    
            Let's show that we have $\Genv + {e_0\rho,t_{\text{if}}} {\Gamma\rho} \leq \Genv + {e_0,t_{\text{if}}} \Gamma \rho$.
            For any $e_s$ subexpression of $e_0$ such that $\exists \varpi.\ \occ {e_0} \varpi=e_s$, we have (the induction hypothesis is used at the third step):
            \begin{align*}
              \Genv + {e_0\rho,t_{\text{if}}} {\Gamma\rho}(e_s\rho) &= \bigwedge_{\{\varpi\alt \occ {e_0\rho} \varpi=e_s\rho\}} \Gp + {\Gamma\rho,e_0\rho,t_{\text{if}}} (\varpi)\\
              &\leq \bigwedge_{\{\varpi\alt (\occ {e_0} \varpi)\rho=e_s\rho\}} \Gp + {\Gamma\rho,e_0\rho,t_{\text{if}}} (\varpi)\\
              &= \bigwedge_{\substack{e_s' \text{ subexpression of } e_0\\\text{s.t. } \dots \text{ and } e_s'\rho=e_s\rho}} \left(\bigwedge_{\{\varpi\alt \occ {e_0} \varpi=e_s'\}} \Gp + {\Gamma\rho,e_0\rho,t_{\text{if}}} (\varpi)\right)\\
              &\leq \bigwedge_{\substack{e_s' \text{ subexpression of } e_0\\\text{s.t. } \dots \text{ and } e_s'\rho=e_s\rho}} \left(\bigwedge_{\{\varpi\alt \occ {e_0} \varpi=e_s'\}} \Gp + {\Gamma,e_0,t_{\text{if}}} (\varpi)\right)\\
              &= \bigwedge_{\substack{e_s' \text{ subexpression of } e_0\\\text{s.t. } \dots \text{ and } e_s'\rho=e_s\rho}} \Genv + {e_0,t_{\text{if}}} \Gamma (e_s')\\
              &= (\Genv + {e_0,t_{\text{if}}} \Gamma \rho)(e_s\rho)
            \end{align*}
    
            %Thus, we have $\Gamma\rho, \Genv + {e_0\rho,t_{\text{if}}} {\Gamma\rho} \leq \Gamma\rho,\Genv + {e_0,t_{\text{if}}}\Gamma\rho = \Gamma_1\rho$.
    
            %Let's show that $\Gamma_1\rho \vdash e_1\rho:t_1''$ with $t_1''\leq t_1'$.
            TODO
            %Moreover, we have $\Gamma_1'\rho=\Gamma_1\rho$ (recall that $e'\rho=e'$).
            %It gives $\Gamma\rho, \Genv + {e_0\rho,t_{\text{if}}} {\Gamma\rho} \leq \Gamma_1\rho = \Gamma_1'\rho$.
    
            %We conclude, by using the monotonicity lemma, $\Gamma\rho, \Genv + {e_0\rho,t_{\text{if}}} {\Gamma\rho} \vdash e_1\rho:t_1'''$ with $t_1''' \leq t_1$.
            \item We get $\Gamma\rho, \Genv - {e_0\rho,t_{\text{if}}} {\Gamma\rho}\vdash e_2\rho:t_2''$ with $t_2'' \leq t_2$ in a similar way.
          \end{itemize}
          We conclude by applying the rule $\Rule {If}$.
        \end{description}
    
        Now, let's show that $\forall p \forall t_{if} \forall \varpi.\ \Gp p {\Gamma\rho,e\rho,t_{if}} (\varpi) \leq \Gp p {\Gamma,e,t_{if}} (\varpi)$.
    
        Let $p\in \{+,-\}$ and $t'$ a type.
        We proceed by induction on $\varpi$.
    
        The base case is straightforward (as we already proved that $\Gamma \rho \vdash e \rho:t'$ with $t'\leq t$).\\
        The inductive cases are also straightfoward using the induction hypothesis (same reasonning as for the proof of monotonicity).
    
        \qed
    
        \subsection{Substitution lemma 2}

        TODO: Update to match new substitution lemma
    
        Let $\Gamma,e,x,e',t,t_x,t_{e'}$ as in the lemma statement.
        We will denote the substitution $\subst x {e'}$ by $\rho$.
    
        By the substitution lemma, we get $(\Gamma, (x:t_x))\rho \vdash e\rho:t'$ with $t' \leq t$.
        As $x$ does not appear in $\dom \Gamma$, we have $(\Gamma, (x:t_x))\rho = \Gamma, (e':t_x)$.
        We deduce $\Gamma, (e':t_x) \vdash e\rho:t'$ with $t' \leq t$.
    
        As $\Gamma \vdash e':t_{e'}$ with $t_{e'} \leq t_x$ and $t_x \neq \Empty$, we have $\Gamma \leq \Gamma, (e':t_x)$.
        Thus, by using the monotonicity lemma, we get $\Gamma\vdash e\rho:t''$ with $t'' \leq t' \leq t$.
    
        \qed
    
        \subsection{Test reduction}
    
        We proceed by induction on $e$.
    
        Let $\Gamma,t,t_{if},e'$ as in the theorem statement.
        
        We will prove here that $\Genv + {e',t_{if}} \Gamma \leq \Genv + {e,t_{if}} \Gamma$.
        The proof for $\Genv - {\dots} \Gamma$ follows as $\Genv - {e,t_{if}} \Gamma=\Genv + {e,\neg t_{if}} \Gamma$.
    
        %We will suppose that all lambda abstracted variables are different in $e$ and $e'$. When it is not the case, we can alpha-rename the terms.
        Let $t'$ such that $\Gamma \vdash e':t'$ and $t'\leq t$ (using subject reduction).
    
        If $\exists e_{\bot} \in \dom\Gamma.\ \Gamma(e_{\bot}) = \Empty$, thus the property is trivial.
        Let's suppose it is not the case (in particular we know that the last rule applied to type $\Gamma \vdash e:t$ can't be $\Rule{Empty}$).
    
        If we had $e\in\dom\Gamma$, then $e$ would not be reducible (because $\Gamma$ is ordinary, and variables are not reducible).
        Thus, we can suppose that $e\not\in\dom\Gamma$ (in particular, we know that the last rule applied to type $\Gamma \vdash e:t$ can't be $\Rule{Occ}$).
    
        We now proceed by case analysis on $e$ (impossible cases are not shown)::
    
        \begin{description}
          \item[$v_1\ v_2$] In this case, we must have $v_1=\lambda^{\bigwedge_{i\in I} \arrow {s_i} {t_i}}x.e_x$ and $e'=e_x\subst x {v_2}$.
    
          The last rule applied to derive $\Gamma \vdash e:t$ is $\Rule{App}$.
          Thus, we know that $\Gamma \vdash v_1 : t_1$ with $t_1$ an arrow type, $\Gamma \vdash v_2 : t_2$, $t_2 \in \dom {t_1}$ and $t=\apply {t_1} {t_2}$.
          
          TODO
    
          \item[$e_1\ v_2$] In this case, only $e_1$ can be reduced, and the last rule in the derivation of $\Gamma \vdash e:t$ is $\Rule{App}$.
          
          So we have $\Gamma \vdash e_1:t_1$ and $\Gamma \vdash v_2:t_2$ with $t=\apply {t_1} {t_2}$.
    
          We also have $e' = e_1' v_2$ with $e_1 \leadsto e_1'$.
    
          \begin{itemize}
            \item Let $t_\epsilon = \Gp + {\Gamma,e,t_{if}} (\epsilon)$ and $t_\epsilon' = \Gp + {\Gamma,e',t_{if}} (\epsilon)$.
            
            Notice that $t_\epsilon' = t_{if} \land t' \leq t_{if} \land t = t_\epsilon$.
    
            \item Let $t_{left} = \Gp + {\Gamma,e,t_{if}} (0)$ and $t_{left}' = \Gp + {\Gamma,e',t_{if}} (0)$.
    
            We have $t_{left}' \leq t_{left}$ (direct consequence of $t_\epsilon' \leq t_\epsilon$).
      
            Using the induction hypothesis, we get $\Genv + {e_1',t_{left}'} \Gamma \leq \Genv + {e_1,t_{left}'} \Gamma$,
            and using the test monotonicity lemma we deduce $\Genv + {e_1,t_{left}'} \Gamma \leq \Genv + {e_1,t_{left}} \Gamma$.
    
            Thus $\Genv + {e_1',t_{left}'} \Gamma \leq \Genv + {e_1,t_{left}} \Gamma$.
    
            \item Let $t_{right} = \Gp + {\Gamma,e,t_{if}} (1)$ and $t_{right}' = \Gp + {\Gamma,e',t_{if}} (1)$.
            
            As in the previous point, we also have $t_{right}' \leq t_{right}$.
    
            Thus, using the test monotonicity lemma, we deduce $\Genv + {v_2,t_{right}'} \Gamma \leq \Genv + {v_2,t_{right}} \Gamma$.
          \end{itemize}
          
          $\Genv + {e,t_{if}} \Gamma = \Genv + {v_2,t_{right}} \Gamma \land \Genv + {e_1,t_{left}} \Gamma \land (e:t_\epsilon)$ TODO
    
          \item[$e_1\ e_2$] This case is similar to the previous one.
          \item[$\ite {v_0} {t_{if}} {e_1} {e_2}$] This case is trivial as $\Genv + {e,t_{if}} \Gamma = (e, t\land t_{if})$. TODO: Actually not trivial.
          \item[$\ite {e_0} {t_{if}} {e_1} {e_2}$] This case is similar to the previous one.
        \end{description}
    
        \qed
    
        \subsection{Subject reduction}
    
        We proceed by induction on $e$.
    
        Let $\Gamma,t,e'$ as in the theorem statement.
        We will suppose that all lambda abstracted variables are different in $e$ and $e'$. When it is not the case, we can alpha-rename the terms.
    
        If $\exists e_{\bot} \in \dom\Gamma.\ \Gamma(e_{\bot}) = \Empty$, thus the property is trivial
        (we can apply the rule $\Rule {Empty}$) to type the beta-reduced expression).
        Let's suppose it is not the case (in particular we know that the last rule applied to type $\Gamma \vdash e:t$ can't be $\Rule{Empty}$).
    
        If we had $e\in\dom\Gamma$, then $e$ would not be reducible (because $\Gamma$ is ordinary, and variables are not reducible).
        Thus, we can suppose that $e\not\in\dom\Gamma$ (in particular, we know that the last rule applied to type $\Gamma \vdash e:t$ can't be $\Rule{Occ}$).
    
        We now proceed by case analysis on $e$ (impossible cases are not shown):
    
        \begin{description}
          \item[$v_1\ v_2$] In this case, we must have $v_1=\lambda^{\bigwedge_{i\in I} \arrow {s_i} {t_i}}x.e_x$ and $e'=e_x\subst x {v_2}$.
    
          The last rule applied to derive $\Gamma \vdash e:t$ is $\Rule{App}$.
          Thus, we know that $\Gamma \vdash v_1 : t_1$ with $t_1$ an arrow type, $\Gamma \vdash v_2 : t_2$, $t_2 \in \dom {t_1}$ and $t=\apply {t_1} {t_2}$.
    
          From this last equality, we deduce $t_1 \leq \arrow {t_2} t$ using the declarative definition of $\circ$.
          It gives $\bigwedge_{i\in I} \arrow {s_i} {t_i} \leq \arrow {t_2} t$.
          Let's show that $\Gamma \vdash e_x\subst x {v_2}:t''$ for a certain type $t''\leq t$.
    
          Let $I'=\{i\in I\alt t_2 \leq s_i\}$. As $t_2 \in \dom {t_1}$, we have $t_2 \leq \bigvee_{i\in I} s_i$.
          As $\Gamma$ is well-formed, we have $v_2 \in \bigvee_{i\in I} s_i$ and by using the $\valsemantic \_$ properties,
          we can deduce that there is at least one $i$ such that $v_2 \in s_i$ and thus $t_2 \leq s_i$ (thanks to the restrictions on lambda-abstraction type annotations).
          It follows that $I'$ is not empty.
    
          Moreover, $t_2 \not\leq \bigvee_{i\in I\setminus I'} s_i$, otherwise it would contradict the definition of $I'$.
          As a consequence, as $\bigwedge_{i\in I} \arrow {s_i} {t_i} \leq \arrow {t_2} t$,
          we can deduce $\bigwedge_{i \in I'}t_i \leq t$.
    
          The last rule of the derivation of $\Gamma \vdash v_1 : t_1$ can only be $\Rule{Abs}$ (it can't be $\Rule{Occ}$ because $\Gamma$ is ordinary).
          Thus, we can derive $\Gamma,(x:s_i) \vdash e_x : t_i'$ with $t_i' \leq t_i$ for any $i\in I'$. By using the strengthening lemma, we deduce that
          $\Gamma, (x:\bigwedge_{i\in I'} s_i) \vdash e_x:t'$ with $t'\leq \bigwedge_{i\in I'} t_i' \leq \bigwedge_{i\in I'} t_i \leq t$.
          As $t_2 \leq \bigwedge_{i\in I'} s_i$, we can apply the substitution lemma to get $\Gamma \vdash e_x\subst x {v_2}:t''$ with $t'' \leq t' \leq t$.
    
          \item[$e_1\ v_2$] In this case, only $e_1$ can be reduced, and the last rule in the derivation of $\Gamma \vdash e:t$ is $\Rule{App}$.
    
          We just apply the induction hypothesis on $e_1$ and conclude using the monotonicity of the $\circ$ operator.
          \item[$e_1\ e_2$] In this case, only $e_2$ can be reduced, and the last rule in the derivation of $\Gamma \vdash e:t$ is $\Rule{App}$.
    
          We just apply the induction hypothesis on $e_2$ and conclude using the monotonicity of the $\circ$ operator.
          \item[$\ite {v_0} {t_{if}} {e_1} {e_2}$] In this case, the last rule in the derivation of $\Gamma \vdash e:t$ is $\Rule{If}$.
           
          So we have $\Gamma \vdash v_0 : t_0$, $\Gamma, \Genv + {v_0,t_{if}} \Gamma\vdash e_1 : t_1$ and $\Gamma, \Genv - {v_0,t_{if}} \Gamma\vdash e_2 : t_2$ with $t=t_1\vee t_2$.
    
          There are two cases:
          \begin{description}
            \item[$v_0 \in t_{if}$] In this case, we have $e'=e_1$.
            By the value test lemma, we have $t_0 \leq t_{if}$.
            Thus, we can easily show that $\Genv + {v_0,t_{if}} \Gamma \leq \Gamma$:
            For each expression in its domain, we proceed by induction on the associated paths.
            Since $v_0$ is a value, the paths $\varpi.0$ and $\varpi.1$ are never encountered. Other cases are trivial.
    
            Thus $\Gamma, \Genv + {v_0,t_{if}} \Gamma \equiv \Gamma$.
    
            By using the equivalence lemma, we can deduce that $\Gamma \vdash e_1 : t_1$, which ends this case since $t_1 \leq t$.
            \item[$v_0 \not\in t_{if}$] This case is similar to the previous one.
          \end{description}
            
          \item[$\ite {e_0} {t_{if}} {e_1} {e_2}$] In this case, only $e_0$ can be reduced, and the last rule in the derivation of $\Gamma \vdash e:t$ is $\Rule{If}$.
          
          So we have $\Gamma \vdash e_0 : t_0$, $\Gamma, \Genv + {e_0,t_{if}} \Gamma\vdash e_1 : t_1$ and $\Gamma, \Genv - {e_0,t_{if}} \Gamma\vdash e_2 : t_2$ with $t=t_1\vee t_2$.
          
          We also have $e'=\ite {e_0'} {t_{if}} {e_1} {e_2}$ with $e_0\leadsto e_0'$.
    
          By applying the test reduction lemma, we get $\Genv + {e_0',t_{if}} \Gamma \leq \Genv + {e_0,t_{if}} \Gamma$ and
          $\Genv - {e_0',t_{if}} \Gamma \leq \Genv + {e_0,t_{if}} \Gamma$.
    
          Using the monotonicity lemma, we get $\Gamma, \Genv + {e_0',t_{if}} \Gamma\vdash e_1 : t_1'$ with $t_1'\leq t_1$
          and $\Gamma, \Genv - {e_0',t_{if}} \Gamma\vdash e_2 : t_2'$ with $t_2'\leq t_2$, and so we conclude easily.
        \end{description}
    
        \qed

\end{document}