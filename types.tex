\documentclass[a4paper]{article}

\usepackage{setup}

\hypersetup{pdfstartview=XYZ}%         zoom par defaut

\setlength{\droptitle}{-5em}   % This is your set screw
\title{\vspace{1.5cm}Type inference - M2 Internship}
\author{Mickael LAURENT}
\date{\vspace{-5ex}}

\pagenumbering{gobble}

\theoremstyle{definition}
\newtheorem{theorem}{Theorem}
\newtheorem{lemma}{Lemma}
\newtheorem{definition}{Definition}
\newtheorem{property}{Property}
\newtheorem{corollary}{Corollary}

\begin{document}

  \maketitle
  
    \section{Types}

    \[
        \begin{array}{lrcl}
        \textbf{Types} & t & ::= & b\alt \arrow t t\alt \pair t t\alt t\vee t \alt \neg t \alt \Empty\\
        \textbf{Atoms} & t_A & ::= & b\alt \arrow t t\alt \pair t t\\
        \textbf{Type schemes} & \ts & ::= & t \alt \tsfun {\arrow t t ; \cdots ; \arrow t t} \alt \ts \tstimes \ts \alt \ts \tsor \ts \alt \tsempty
        \end{array}
    \]

    \section{Definitions}

    \begin{definition}[DNF]
      A disjonctive normal form $\tau$ is a finite set of pairs of finite sets atoms such that:
      \[ \forall (P,N)\in \tau.\ \semantic {\bigwedge_{t_A \in P} t_A \land \bigwedge_{t_A \in N} \neg t_A} \neq \varnothing \]

      The interpretation $\semantic \tau$ of the DNF $\tau$ is the following:
      \[
        \semantic \tau = \semantic {\bigvee_{(P,N)\in \tau} \left(\bigwedge_{t_A \in P} t_A \land \bigwedge_{t_A \in N} \neg t_A \right)}
      \]
    \end{definition}

    \begin{definition}[Interpretation of type schemes]
      We define a function $\tsint {\_}$ that maps type schemes to set of types.

      \begin{align*}
        \begin{array}{lcl}
        \tsint t &=& \{s\alt t \leq s\}\\
        \tsint {\tsfunone {t_i} {s_i}_{i=1..n}} &=& \{s\alt
        \exists s_0 = \bigwedge_{i=1..n} \arrow {t_i} {s_i}
        \land \bigwedge_{j=1..m} \neg (\arrow {t_j'} {s_j'}).\ 
        \Empty \not\simeq s_0 \leq s \}\\
        \tsint {\ts_1 \tstimes \ts_2} &=& \{s\alt \exists t_1 \in \tsint {\ts_1}\ 
        \exists t_2 \in \tsint {\ts_2}.\ \pair {t_1} {t_2} \leq s\}\\
        \tsint {\ts_1 \tsor \ts_2} &=& \{s\alt \exists t_1 \in \tsint {\ts_1}\ 
        \exists t_2 \in \tsint {\ts_2}.\ {t_1} \vee {t_2} \leq s\}\\
        \tsint \tsempty &=& \varnothing
        \end{array}
      \end{align*}

    Note that $\tsint \ts$ is closed under subsumption and intersection.
    \end{definition}

    \section{Theorems}

    \begin{property}
      For every type $t$, we can compute a DNF $\tau$ such that $\semantic t = \semantic \tau$.
    \end{property}

    \begin{lemma}
      Let $\ts$ be a type scheme and $t$ a type. We can compute a type scheme, written $t \tsand \ts$, such that:
      \[\tsint {t \tsand \ts} = \{s \alt \exists t' \in \tsint \ts.\ t \land t' \leq s \}\]
    \end{lemma}
    Proof: lemma 6.40 semantic\_subtyping (p45) [TODO: ref]

    \begin{lemma}[Intersection of type schemes]
      Let $\ts_1$ and $\ts_2$ be two type schemes. We can compute a type scheme, written $\ts_1 \tsand \ts_2$, such that:
      \[\tsint {\ts_1 \tsand \ts_2} = \{s\alt \exists t_1 \in \tsint {\ts_1}\ 
      \exists t_2 \in \tsint {\ts_2}.\ {t_1} \land {t_2} \leq s\}\]
    \end{lemma}

    \section{Proofs}

    \subsection{Intersection of type schemes}

    TODO
    
\end{document}