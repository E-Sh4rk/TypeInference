\documentclass[a4paper]{article}

\usepackage{setup}

\hypersetup{pdfstartview=XYZ}%         zoom par defaut

\setlength{\droptitle}{-5em}   % This is your set screw
\title{\vspace{1.5cm}Type inference - M2 Internship}
\author{Mickael LAURENT}
\date{\vspace{-5ex}}

\pagenumbering{gobble}

\theoremstyle{definition}
\newtheorem{theorem}{Theorem}
\newtheorem{lemma}{Lemma}
\newtheorem{definition}{Definition}
\newtheorem{property}{Property}
\newtheorem{corollary}{Corollary}

% TODO: use =def notation when relevant
% \def\Tiny{\fontsize{4pt}{4pt}\selectfont}
% \newcommand*{\eqdef}{\ensuremath{\overset{\mathclap{\text{\Tiny def}}}{=}}}

\begin{document}

  \maketitle

  \subsection{Algorithmic type system}

  This section is about the algorithmic type system (soundness and some completeness properties).

  Note that we use a different but more convenient definition for $\tyof e \Gamma$ that the one
  in Section~\ref{sec:typenv}:
  \begin{align*}
    \tyof e \Gamma = 
    \left\{\begin{array}{ll}
      \ts & \text{if } \Gamma \vdashA e:\ts \\
      \tsempty & \text{otherwise}
    \end{array}\right.
  \end{align*}

  In this way, $\tyof e \Gamma$ is always defined but is equal to $\tsempty$ when $e$ is not
  well-typed in $\Gamma$. 

  We will reuse the definitions and notations introduced in the previous proofs.

  \subsubsection{Soundness}

  \begin{theorem}[Soundness of the algorithm]
    \begin{align*}
      &\forall \Gamma, e, t.\ \tyof e \Gamma \leq t \Rightarrow \Gamma \vdash e:t\\
      &\forall \Gamma, e, t, \varpi.\ \pvdash \Gamma e t \varpi:\env {\Gamma,e,t} (\varpi)\\
      &\forall \Gamma, e, t.\ \Gamma \evdash e t \Refine {e,t} \Gamma
    \end{align*}
  \end{theorem}

  Proof:

  We proceed by induction over the structure of $e$
  and, for two identical $e$, on the domain of $\Gamma$ (with the inclusion order).

  Let's prove the first property.
  Let $t$ such that $\tsint{\tyof e \Gamma} \leq t$.

  If $\Gamma = \bot$, we trivially have $\Gamma \vdash e:t$ with the rule \Rule{Efq}.
  Let's assume $\Gamma \neq \bot$.

  If $e=x$ a variable, then the last rule used is \Rule{Var\Aa}.
  We can derive $\Gamma \vdash x:t$ by using the rule \Rule{Env} and \Rule{Subs}.
  So let's assume that $e$ is not a variable.

  If $e\in\dom\Gamma$, then the last rule used is \Rule{Env\Aa}.
  Let $t'\in\tsint{\ts}$ such that $t'\land\Gamma(e)\leq t$.
  The induction hypothesis gives $\Gamma\setminus\{e\} \vdash e:t'$
  (the premise uses the same $e$ but the domain of $\Gamma$ is strictly smaller).
  Thus, we can build a derivation $\Gamma \vdash e:t$ by using the rules \Rule{Subs}, \Rule{Inter},
  \Rule{Env} and the derivation $\Gamma\setminus\{e\} \vdash e:t'$.

  Now, let's suppose that $e\not\in\dom\Gamma$.

  \begin{description}
    \item[$e=c$] The last rule is \Rule{Const\Aa}. We derive easily $\Gamma \vdash c:t$ with \Rule{Const} and \Rule{Subs}.
    \item[$e=x$] Already treated.
    \item[$e=\lambda^{\bigwedge_{i\in I} \arrow {t_i}{s_i}}x.e'$]
    The last rule is \Rule{Abs\Aa}.
    We have $\arrow {t_i}{s_i} \leq t$.
    Using the definition of type schemes, let $t'=\bigwedge_{i\in I} \arrow {t_i}{s_i} \land \bigwedge_{j\in J} \neg \arrow {t'_j}{s'_j}$ such that $\Empty \neq t' \leq t$.
    The induction hypothesis gives, for all $i\in I$, $\Gamma,x:s_i\vdash e':t_i$.
    
    Thus, we can derive $\Gamma\vdash e:\bigwedge_{i\in I} \arrow {t_i}{s_i}$ using the rule \Rule{Abs+}, and with \Rule{Inter} and
    \Rule{Abs-} we can derive $\Gamma\vdash e:t'$. We can conclude by applying \Rule{Subs}.
    \item[$e=e_1 e_2$] The last rule is \Rule{App\Aa}.
    We have $\apply {\ts_1} {\ts_2} \leq t$. Thus, let $t_1$ and $t_2$ such that $\ts_1 \leq t_1$, $\ts_2 \leq t_2$ and $\apply {t_1} {t_2} \leq t$.
    We know, according to the descriptive definition of $\apply {} {}$, that there exists $s\leq t$ such that $t_1 \leq \arrow {t_2} s$.

    By using the induction hypothesis, we have $\Gamma\vdash e_1:t_1$ and $\Gamma\vdash e_2:t_2$. We can thus derive
    $\Gamma\vdash e_1:\arrow {t_2} s$ using \Rule{Subs}, and together with $\Gamma\vdash e_2:t_2$ it gives
    $\Gamma\vdash e_1\ e_2:s$ with \Rule{App}. We conclude with \Rule{Subs}.

    \item[$e=\pi_i e'$] The last rule is \Rule{Proj\Aa}. We have $\bpi_i \ts \leq t$. Thus, let $t'$ such that $\ts \leq t'$ and $\bpi_i t' \leq t$.
    We know, according to the descriptive definition of $\bpi_i$, that there exists $t_i\leq t$ such that $t' \leq \pair \Any {t_i}$ (for $i=2$) or $t' \leq \pair {t_i} \Any$ (for $i=1$).
    
    By using the induction hypothesis, we have $\Gamma\vdash e':t'$, and thus we easily conclude using \Rule{Subs} and \Rule{Proj}
    (for instance for the case $i=1$, we can derive $\Gamma\vdash e':\pair {t_i} \Any$ with \Rule{Subs} and then use \Rule{Proj}).

    \item[$e=(e_1,e_2)$] The last rule is \Rule{Pair\Aa}. We conclude easily with the induction hypothesis and the rules \Rule{Subs} and \Rule{Pair}.

    \item[$\ite {e_0} t {e_1} {e_2}$] The last rule is \Rule{Case\Aa}. We conclude easily with the induction hypothesis and the rules
    \Rule{Subs} and \Rule{Case} (for the application of \Rule{Case}, $t'$ must be taken equal to $t_1 \vee t_2$ with $t_1$ and $t_2$ such that $\ts_1\leq t_1$, $\ts_2\leq t_2$ and $t_1 \vee t_2 \leq t$).
  \end{description}\ \\

  Now, let's prove the second property.
  We perform a (nested) induction on $\varpi$.

  Recall that $\env {\Gamma,e,t} (\varpi) = \constr \varpi {\Gamma,e,t} \land \tsrep {\tyof {\occ e \varpi} \Gamma}$.
  We can easily derive $\pvdash \Gamma e t \varpi : \tsrep {\tyof {\occ e \varpi} \Gamma}$ by using the outer induction hypothesis
  (by definition of $\tsrep {\_}$ we have $\forall \ts.\ \ts \leq \tsrep \ts$).

  We also have to derive $\pvdash \Gamma e t \varpi : \constr \varpi {\Gamma,e,t}$ (then it will be easy to conclude using the rule \Rule{PInter}).
  \begin{description}
    \item[$\varpi=\epsilon$] We use the rule \Rule{PEps}.
    \item[$\varpi=\varpi'.1$]
    Let's note $f=\tsrep {\tyof {\occ e {\varpi'.0}} \Gamma}$, $s=\env {\Gamma,e,t} (\varpi')$ and $t_{\text{res}} = \worra f s$. 
    
    By using the outer and inner induction hypotheses, we can derive $\pvdash \Gamma e t \varpi'.0 : f$ and $\pvdash \Gamma e t \varpi' : s$.

    By using the descriptive definition of $\worra {} {}$, we have $t' = \apply f {(\dom f \setminus t_{\text{res}})} \leq \neg s$.

    Moreover, by using the descriptive definition of $\apply {} {}$ on $t'$, we have
    $f \leq \arrow {(\dom f \setminus t_{\text{res}})} {t'}$.

    As $t'\leq \neg s$, it gives $f \leq \arrow {(\dom f \setminus t_{\text{res}})} {\neg s}$.

    Let's note $t_1 = \dom f \setminus t_{\text{res}}$ and $t_2 = \neg s$. The above inequality can be rewritten $f \leq \arrow {t_1} {t_2}$.

    Thus, by using \Rule{PSubs} on the derivation $\pvdash \Gamma e t \varpi'.0 : f$, we can derive $\pvdash \Gamma e t \varpi'.0 : \arrow {t_1} {t_2}$.
    We have:
    \begin{itemize}
      \item $t_2 \land s \simeq \Empty$ (as $t_2 = \neg s$)
      \item $\neg t_1 = t_{\text{res}} \vee \neg \dom f = t_{\text{res}}$
    \end{itemize}
    
    In consequence, we can conclude by applying the rule \Rule{PAppR}
    with the premises $\pvdash \Gamma e t \varpi'.0 : \arrow {t_1} {t_2}$ and $\pvdash \Gamma e t \varpi' : s$.

    \item[$\varpi=\varpi'.0$] By using the inner induction hypothesis and the previous case we've just proved, we can derive
    $\pvdash \Gamma e t \varpi' : \env {\Gamma,e,t} (\varpi')$ and $\pvdash \Gamma e t \varpi'.1 : \env {\Gamma,e,t} (\varpi'.1)$.
    Hence we can apply \Rule{PAppL}.
    \item[$\varpi=\varpi'.l$] Let's note $t_1=\bpi_1 \env {\Gamma,e,t} (\varpi')$.
    According to the descriptive definition of $\bpi_1$, we have $\env {\Gamma,e,t} (\varpi') \leq \pair {t_1} {\Any}$.

    The inner induction hypothesis gives $\pvdash \Gamma e t \varpi' : \env {\Gamma,e,t} (\varpi')$, and thus using the rule \Rule{PSubs}
    we can derive $\pvdash \Gamma e t \varpi' : \pair {t_1} \Any$. We can conclude just by applying the rule \Rule{PPairL} to this premise.
    \item[$\varpi=\varpi'.r$] This case is similar to the previous.
    \item[$\varpi=\varpi'.f$] The inner induction hypothesis gives $\pvdash \Gamma e t \varpi' : \env {\Gamma,e,t} (\varpi')$,
    so we can conclude by applying \Rule{PFst}.
    \item[$\varpi=\varpi'.s$] The inner induction hypothesis gives $\pvdash \Gamma e t \varpi' : \env {\Gamma,e,t} (\varpi')$,
    so we can conclude by applying \Rule{PSnd}.
  \end{description}\ \\

  Finally, let's prove the third property.
  Let $\Gamma' = \Refine {e,t} \Gamma = {\RefineStep{e,t}}^{n_0}(\Gamma)$.
  We want to show that $\Gamma \evdash e t \Gamma'$ is derivable.

  First, let's notive that $\evdash e t$ is transitive:
  if $\Gamma \evdash e t \Gamma'$ and $\Gamma' \evdash e t \Gamma''$, then $\Gamma \evdash e t \Gamma''$.
  The proof is quite easy: we can just start from the derivation of $\Gamma \evdash e t \Gamma'$, and we add
  at the end a slightly modified version of the derivation of $\Gamma' \evdash e t \Gamma''$ where:
  \begin{itemize}
    \item the initial \Rule{Base} rule has been removed in order to be able to do the junction,
    \item all the $\Gamma'$ at the left of $\evdash e t$ are replaced by $\Gamma$
    (the proof is still valid as this $\Gamma'$ at the left is never used in any rule)
  \end{itemize}

  Thanks to this property, we can suppose that $n_0 = 1$ (and so $\Gamma' = \RefineStep{e,t}(\Gamma)$).
  If it is not the case, we just have to proceed by induction on $n_0$ and use the transitivity property.

  Let's build a derivation for $\Gamma \evdash e t \Gamma'$.

  By using the proof of the second property on $e$ that we've done just before, we get:
  $\forall \varpi.\ \pvdash \Gamma e t \varpi: \env {\Gamma,e,t} (\varpi)$.

  Let's recall a monotonicity property: for any $\Gamma_1$ and $\Gamma_2$ such that $\Gamma_2 \leq \Gamma_1$, we have
  $\forall t'.\ \pvdash {\Gamma_1} e t \varpi:t' \Rightarrow \pvdash {\Gamma_2} e t \varpi:t'$.\\  
  Moreover, when we also have $\occ e \varpi \in\dom{\Gamma_2}$, we can derive $\pvdash {\Gamma_2} e t \varpi:t'\land\Gamma_2(\occ e \varpi)$
  (just by adding a \Rule{PInter} rule with a \Rule{PTypeof} and a \Rule{Env}).

  Hence, we can apply successively a \Rule{Path} rule for all valid $\varpi$ in $e$,
  with the following premises ($\Gamma_\varpi$ being the previous environment, that trivially verifies $\Gamma_\varpi\leq\Gamma$):\\
  
  \begin{tabular}{lll}
    If $\occ e \varpi \in\dom{\Gamma_\varpi}$&$\pvdash {\Gamma_\varpi} e t \varpi: \env {\Gamma,e,t} (\varpi)\land\Gamma_\varpi(\occ e \varpi)$&$\Gamma\evdash e t {\Gamma_\varpi}$\\
    Otherwise&$\pvdash {\Gamma_\varpi} e t \varpi: \env {\Gamma,e,t} (\varpi)$&$\Gamma\evdash e t {\Gamma_\varpi}$
  \end{tabular}

  At the end, it gives the judgement $\Gamma \evdash e t \Gamma'$, so it concludes the proof.

  \qed

\end{document}