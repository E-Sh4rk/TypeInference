\documentclass[a4paper]{article}

\usepackage{setup}

\hypersetup{pdfstartview=XYZ}%         zoom par defaut

\setlength{\droptitle}{-5em}   % This is your set screw
\title{\vspace{1.5cm}Type inference - M2 Internship}
\author{Mickael LAURENT}
\date{\vspace{-5ex}}

\pagenumbering{gobble}

\theoremstyle{definition}
\newtheorem{theorem}{Theorem}
\newtheorem{lemma}{Lemma}
\newtheorem{definition}{Definition}
\newtheorem{property}{Property}
\newtheorem{corollary}{Corollary}

\begin{document}

  \maketitle

  \subsubsection{Some definitions}

    \begin{definition}[Bottom environment]
      Let $\Gamma$ an environment.\\
      $\Gamma$ is bottom (noted $\Gamma = \bot$) iff $\exists e\in\dom\Gamma.\ \Gamma(e)\simeq\Empty$.
    \end{definition}

    \begin{definition}[Algorithmic (pre)order on environments]
    Let $\Gamma$ and $\Gamma'$ two environments. We write $\Gamma' \leqA \Gamma$ iff:
    \begin{align*}
        &\Gamma'=\bot \text{ or } (\Gamma\neq\bot \text{ and } \forall e \in \dom \Gamma.\ \tyof e \Gamma \leq \Gamma(e))
    \end{align*}

    For an expression $e$, we write $\Gamma' \leqA^e \Gamma$ iff:
    \begin{align*}
      &\Gamma'=\bot \text{ or } (\Gamma\neq\bot \text{ and } \forall e' \in \dom \Gamma \text{ such that $e'$ is a subexpression of $e$}.\ \tyof {e'} \Gamma \leq \Gamma({e'}))
    \end{align*}

    Note that if $\Gamma' \leqA \Gamma$, then $\Gamma' \leqA^e \Gamma$ for any $e$.
    \end{definition}

    \begin{definition}[Order relation for type schemes]
      Let $\ts_1$ and $\ts_2$ two type schemes. We write $\ts_2 \leq \ts_1$ iff $\tsint {\ts_1} \subseteq \tsint{\ts_2}$.
    \end{definition}

  \subsubsection{Major lemmas and completeness}

  \begin{lemma}[Monotonicity] Let $\Gamma$, $\Gamma'$ and $e$ such that $\Gamma'\leqA^e \Gamma$ and $\tyof e \Gamma \neq \tsempty$. We have:
    \begin{align*}
      &\tyof e {\Gamma'} \leq \tyof e {\Gamma} \text{ and } \tsrep{\tyof e {\Gamma'}} \leq \tsrep{\tyof e {\Gamma}}\\
      &\forall t,\varpi.\ \env {\Gamma',e,t} (\varpi) \leq \env {\Gamma,e,t} (\varpi)\\
      &\forall t.\ \Refine {e,t} {\Gamma'} \leqA^e \Refine {e,t} \Gamma\\
    \end{align*}
  \end{lemma}

  Proof:

  We proceed by induction over the structure of $e$
  and, for two identical $e$, on the domains of $\Gamma$ and $\Gamma'$ (with the lexicographical inclusion order).

  Let's prove the first property: $\tyof e {\Gamma'} \leq \tyof e {\Gamma} \text{ and } \tsrep{\tyof e {\Gamma'}} \leq \tsrep{\tyof e {\Gamma}}$.
  We will focus on showing $\tyof e {\Gamma'} \leq \tyof e {\Gamma}$. The property $\tsrep{\tyof e {\Gamma'}} \leq \tsrep{\tyof e {\Gamma}}$
  follows for each case by noticing that $\tsrep {t \tsand \ts} \leq t \land \tsrep {\ts}$ and
  $\tsrep {\ts_1 \tsor \ts_2} = \tsrep {\ts_1} \vee \tsrep {\ts_2}$ (the only rule that introduces the type scheme constructor $\tsfun {\_}$ is \Rule{Abs\Aa}).

  If $\Gamma' = \bot$ we can conclude directly with the rule \Rule{Efq}.
  So let's assume $\Gamma' \neq \bot$ and $\Gamma \neq \bot$
  (as $\Gamma = \bot \Rightarrow \Gamma' = \bot$ by definition of $\leqA^e$).

  If $e=x$ is a variable, then the last rule used in $\tyof e \Gamma$ and $\tyof e {\Gamma'}$ is \Rule{Var\Aa}.
  As $\Gamma' \leqA^e \Gamma$, we have $\Gamma'(e) \leq \Gamma(e)$ and thus
  we can conclude with the rule \Rule{Var\Aa}.
  So let's assume that $e$ is not a variable.

  If $e\in\dom\Gamma$, then the last rule used in $\tyof e \Gamma$ is \Rule{Env\Aa}.
  As $\Gamma' \leqA^e \Gamma$, we have $\tyof e {\Gamma'} \leq \Gamma(e)$.
  Moreover, by applying the induction hypothesis, we get $\tyof e {\Gamma'\setminus \{e\}} \leq \tyof e {\Gamma\setminus \{e\}}$
  (we can easily verify that $\Gamma'\setminus\{e\} \leqA^e \Gamma\setminus\{e\}$).
  \begin{itemize}
    \item If we have $e\in\dom{\Gamma'}$, we have according to the rule \Rule{Env\Aa}
    $\tyof e {\Gamma'} \leq \tyof e {\Gamma'\setminus \{e\}} \leq \tyof e {\Gamma\setminus \{e\}}$.
    Together with $\tyof e {\Gamma'} \leq \Gamma(e)$,
    we deduce $\tyof e {\Gamma'} \leq \Gamma(e) \tsand \tyof e {\Gamma\setminus \{e\}} = \tyof e {\Gamma}$.
    \item  Otherwise, we have $e\not\in\dom{\Gamma'}$. Thus
    $\tyof e {\Gamma'} = \tyof e {\Gamma'\setminus \{e\}} \leq \Gamma(e) \tsand \tyof e {\Gamma\setminus \{e\}}=\tyof e {\Gamma}$.
  \end{itemize}

  If $e\not\in\dom\Gamma$ and $e\in\dom{\Gamma'}$, the last rule is \Rule{Env\Aa} for $\tyof e {\Gamma'}$.
  As $\Gamma'\setminus\{e\} \leqA^e \Gamma\setminus\{e\} = \Gamma$,
  we have $\tyof e {\Gamma'} \leq \tyof e {\Gamma'\setminus \{e\}} \leq \tyof e {\Gamma}$ by induction hypothesis.

  Thus, let's suppose that $e\not\in\dom\Gamma$ and $e\not\in\dom{\Gamma'}$.
  From now we know that the last rule in $\tyof e {\Gamma}$ and $\tyof e {\Gamma'}$ is the same.

  \begin{description}
    \item[$e=c$] The last rule is \Rule{Const\Aa}. It does not depend on $\Gamma$ so this case is trivial.
    \item[$e=x$] Already treated.
    \item[$e=\lambda^{\bigwedge_{i\in I} \arrow {t_i}{s_i}}x.e'$]
    The last rule is \Rule{Abs\Aa}.
    We have $\forall i\in I.\ \Gamma', (x:s_i) \leqA^{e'} \Gamma, (x:s_i)$ (quite straightforward)
    so by applying the induction hypothesis we have $\forall i\in I.\ \tyof {e'} {\Gamma', (x:s_i)} \leq \tyof {e'} {\Gamma, (x:s_i)}$.

    \item[$e=e_1 e_2$] The last rule is \Rule{App\Aa}.
    We can conclude immediately by using the induction hypothesis and noticing that $\apply {} {}$ is monotonic for both of its arguments.

    \item[$e=\pi_i e'$] The last rule is \Rule{Proj\Aa}.
    We can conclude immediately by using the induction hypothesis and noticing that $\bpi_i$ is monotonic.
     
    \item[$e=(e_1,e_2)$] The last rule is \Rule{Pair\Aa}.
    We can conclude immediately by using the induction hypothesis.

    \item[$\ite {e_0} t {e_1} {e_2}$] The last rule is \Rule{Case\Aa}.
    By using the induction hypothesis we get $\Refine {e_0,t} {\Gamma'} \leqA^{e_0} \Refine {e_0,t} \Gamma$.
    We also have $\Gamma' \leqA^{e_1} \Gamma$ (as $e_1$ is a subexpression of $e$).

    From those two properties, let's show that we can deduce $\Refine {e_0,t} {\Gamma'} \leqA^{e_1} \Refine {e_0,t} \Gamma$:

    Let $e'\in\dom{\Refine {e_0,t} \Gamma}$ a subexpression of $e_1$.
    \begin{itemize}
      \item If $e'$ is also a subexpression of $e_0$, we can directly deduce
      $\tyof {e'} {\Refine {e_0,t} \Gamma'} \leq (\Refine {e_0,t} \Gamma) (e')$
      by using $\Refine {e_0,t} {\Gamma'} \leqA^{e_0} \Refine {e_0,t} \Gamma$.

      \item Otherwise, as $\Refine {e_0,t} {\_}$ is extensive,
      we have $\Refine {e_0,t} {\Gamma'} \leqA \Gamma'$ and thus by using the induction hypothesis
      $\tyof {e'} {\Refine {e_0,t} {\Gamma'}} \leq \tyof {e'} {\Gamma'}$.
      We also have $\tyof {e'} {\Gamma'} \leq \Gamma(e')$ by using $\Gamma' \leqA^{e_1} \Gamma$.
      We deduce $\tyof {e'} {\Refine {e_0,t} {\Gamma'}} \leq \Gamma(e') = (\Refine {e_0,t} \Gamma) (e')$.
    \end{itemize}
    
    So we have $\Refine {e_0,t} {\Gamma'} \leqA^{e_1} \Refine {e_0,t} \Gamma$.
    Consequently, we can apply the induction hypothesis again to get
    $\tyof {e_1} {\Refine {e_0,t} {\Gamma'}} \leq \tyof {e_1} {\Refine {e_0,t} {\Gamma}}$.

    We proceed the same way for the last premise.
  \end{description}\ \\

  Now, let's prove the second property.
  We perform a (nested) induction on $\varpi$.

  TODO

  Finally, let's prove the third property.

  TODO

  \qed

\end{document}