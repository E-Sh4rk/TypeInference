\documentclass[a4paper]{article}

\usepackage{setup}

\hypersetup{pdfstartview=XYZ}%         zoom par defaut

\setlength{\droptitle}{-5em}   % This is your set screw
\title{\vspace{1.5cm}Type inference - M2 Internship}
\author{Mickael LAURENT}
\date{\vspace{-5ex}}

\pagenumbering{gobble}

\theoremstyle{definition}
\newtheorem{theorem}{Theorem}
\newtheorem{lemma}{Lemma}
\newtheorem{definition}{Definition}
\newtheorem{property}{Property}
\newtheorem{corollary}{Corollary}

\begin{document}

    \maketitle

    \section{Typing system}

    \[
      \begin{array}{lrcl}
      \textbf{Types} & t & ::= & b\alt \arrow t t\alt \pair t t\alt t\vee t \alt \neg t \alt \Empty\\
      \textbf{Atoms} & t_A & ::= & b\alt \arrow t t\alt \pair t t\\
      \textbf{Non-Functional Types} & t_b & ::= & b \alt \pair {t_b} {t_b} \alt t_b\vee t_b \alt \neg t_b \alt \Empty\\
      \textbf{Normal Types} & \hat t & ::= & t_b \alt \arrow {\hat t} {\hat t}\alt \hat t \wedge \hat t \alt \hat t\vee \hat t
      \end{array}
    \]

    \[
      \norm t = \text{DNF of } t \text{ + remove negative arrows}
    \]

    \[
    \begin{array}{lcl}
      \typep+\epsilon{\Gamma,e,t} & = & t\\
      \typep-\epsilon{\Gamma,e,t} & = & \neg t\\
      \typep{p}{\varpi.0}{\Gamma,e,t} & = & \neg(\arrow{\Gp p{\Gamma,e,t}{(\varpi.1)}}{\neg \Gp p {\Gamma,e,t} (\varpi)})\\
      \typep{p}{\varpi.1}{\Gamma,e,t} & = & \worra{\tyof{\occ e{\varpi.0}}\Gamma}{\Gp p {\Gamma,e,t} (\varpi)}\\
      \typep{p}{\varpi.l}{\Gamma,e,t} & = & \bpl{\Gp p {\Gamma,e,t} (\varpi)}\\
      \typep{p}{\varpi.r}{\Gamma,e,t} & = & \bpr{\Gp p {\Gamma,e,t} (\varpi)}\\
      \typep{p}{\varpi.f}{\Gamma,e,t} & = & \pair{\Gp p {\Gamma,e,t} (\varpi)}\Any\\
      \typep{p}{\varpi.s}{\Gamma,e,t} & = & \pair\Any{\Gp p {\Gamma,e,t} (\varpi)}\\ \\
      \Gp p {\Gamma,e,t} (\varpi) & = & \norm {\typep p \varpi {\Gamma,e,t} \land \tyof {\occ e \varpi} \Gamma
    \underbrace{\land \tyof {\occ e \varpi} {\Gamma\setminus\{\occ e \varpi\}}}_{\text{if $\occ e \varpi$ is not a variable}}}\\
    \end{array}
    \]

    \begin{align*}
      &(\Refine p {e,t} \Gamma)(e') = 
        \left\{\begin{array}{ll}
          \bigwedge_{\{\varpi \alt \occ e \varpi \equiv e'\}} \Gp p {\Gamma,e,t} (\varpi) & \text{if } \exists \varpi.\ \occ e \varpi \equiv e' \\
          \Gamma(e') & \text{otherwise, if } e' \in \dom \Gamma\\
          \text{undefined} & \text{otherwise}
        \end{array}\right.\\&\\
      &\Gaux {\varnothing} \Gamma = \Gamma\\
      &\Gaux {\cons {(e,t)} \tenv} \Gamma = \Gaux \tenv {\Refine p {e,t} \Gamma}\\&\\
      &\Genv \tenv \Gamma=\fixpoint_\Gamma (\Gaa \tenv)\qquad \text{(if defined)}
    \end{align*}

    \begin{mathpar}

        \Infer[Occ]
            {
            }
            { \Gamma \tvdash e: \Gamma(e) }
            { e\in\dom\Gamma}
        \qquad
        \Infer[Const]
            { }
            {\Gamma\tvdash c:\basic{c}}
            {c\not\in\dom\Gamma}
        \qquad
        \Infer[Abs]
            { }
            {
            \Gamma\tvdash\lambda^{t}x.e:t
            }
            {\lambda^{t}x.e\not\in\dom\Gamma}
            \\
        \Infer[App]
            {
              \Gamma \tvdash e_1: t_1\\
              \Gamma \tvdash e_2: t_2\\
              t_1 \leq \arrow \Empty \Any\\
              t_2 \leq \dom {t_1}
            }
            { \Gamma \tvdash {e_1}{e_2}: t_1 \circ t_2 }
            { {e_1}{e_2}\not\in\dom\Gamma }
            \\
        \Infer[Proj]
        {\Gamma \tvdash e:t\and t\leq\pair\Any\Any}
        {\Gamma \tvdash \pi_i e:\bpi_{\mathbf{i}}(t)}
        {\pi_i e\not\in\dom\Gamma}

        \Infer[Pair]
        {\Gamma \tvdash e_1:t_1 \and \Gamma \tvdash e_2:t_2}
        {\Gamma \tvdash (e_1,e_2):\pair{t_1}{t_2}}
        {(e_1,e_2)\not\in\dom\Gamma}
    \end{mathpar}

    \section{Definitions}
      
    \begin{definition}[Well formed environment]
      An environment $\Gamma$ is well-formed iff:
      \begin{align*}
        &\forall e \in \dom \Gamma.\ e \text{ is normal (it has a DNF without negative arrow)}\\
        &\forall e \in \dom \Gamma \text{ not a variable. } \exists t.\ \Gamma\setminus(e:\Gamma(e)) \tvdash e : t \text{ and } \Gamma(e) \leq t
      \end{align*}
      In particular, this last property implies:
      \[
        \forall e \in \dom \Gamma.\ \forall t.\ \Gamma\setminus(e:\Gamma(e)) \tvdash e : t \Rightarrow \Gamma(e) \leq t
      \]
    \end{definition}
  
    \begin{definition}[Environment inclusion]
      Let $\Gamma$ and $\Gamma'$ two well-formed environments. We says that $\Gamma' \leq \Gamma$ iff:
      \begin{align*}
        &\forall e \in \dom \Gamma.\ \Gamma' \tvdash e : t \text{ with } t \leq \Gamma(e)
      \end{align*}
    \end{definition}

    \begin{definition}[DNF]
      A disjonctive normal form $\tau$ is a finite set of pairs of finite sets atoms such that:
      \[ \forall (P,N)\in \tau.\ \semantic {\bigwedge_{t_A \in P} t_A \land \bigwedge_{t_A \in N} \neg t_A} \neq \varnothing \]

      The interpretation $\semantic \tau$ of $\tau$ is the following:
      \[
        \semantic \tau = \semantic {\bigvee_{(P,N)\in \tau} \left(\bigwedge_{t_A \in P} t_A \land \bigwedge_{t_A \in N} \neg t_A \right)}
      \]
    \end{definition}
    
    \begin{definition}[Plinth]
      A plinth $S$ is a set of types such that:
      \begin{itemize}
        \item $S$ is finite.
        \item $\Empty \in S$ and $S$ is stable by $\neg, \land, \vee$. 
        \item If $\tau$ is a DNF of $t\in S$, let's introduce $A=\{t_a \alt \exists (P,N) \in \tau \text{ s.t. } t_a \in P \text{ or } t_a \in N \}$. Thus:
        \begin{itemize}
          \item For all $t_a \in A$, we have $t_a \in S$.
          \item For all $t_1$, $t_2$ such that $\arrow {t_1} {t_2} \in A$ or $\pair {t_1} {t_2} \in A$,
          we have $t_1 \in S$ and $t_2 \in S$.
        \end{itemize}
      \end{itemize}
    \end{definition}

    \section{Theorems}

        \begin{lemma}[$\Ga {} $ well-formedness]
          Let $\Gamma$ a well-formed environment. Let $\tenv$ a set of tests.\\
          If $\Genv \tenv \Gamma$ is defined, then it is well-formed.
        \end{lemma}

        \begin{lemma}[$\tvdash$ monotonicity]
          Let $\Gamma$ and $\Gamma'$ two well-formed environments such that $\Gamma' \leq \Gamma$.
          Let $e$ an expression and $t$ a type.\\
          If $\Gamma \tvdash e : t$, then there exists $t'\leq t \text{ such that } \Gamma' \tvdash e : t'$.
        \end{lemma}

        \begin{corollary}[Relation order]
          The relation $\leq$ on environements is a preorder.
        \end{corollary}

        \begin{lemma}[$\Gaa {} $ existence]
          Let $\Gamma$ a well-formed environment.
          Let $\tenv$ a set of tests such that for each test $(e,t,p)$, $e$ is well-typed in $\Gamma$ (according to $\tvdash$).\\
          Then $\Gaux \tenv \Gamma$ is well-defined.
        \end{lemma}

        \begin{lemma}[$\Gaa {}$ monotonicity]
          Let $\Gamma$ and $\Gamma'$ two well-formed environments such that $\Gamma' \leq \Gamma$.
          Let $\tenv$ a set of well-typed tests.\\
          Then, $\Gaux \tenv {\Gamma'} \leq \Gaux \tenv \Gamma$.
        \end{lemma}

        \begin{theorem}[$\Ga {}$ existence]
            If $\Gamma$ is well-formed and $\tenv$ is well typed, then $\Genv \tenv \Gamma$ is well defined and well-formed.
        \end{theorem}

        \begin{property}
          For every type $t$, we can compute a DNF $\tau$ such that $\semantic t = \semantic \tau$.
        \end{property}

        \begin{theorem}
          Every finite set of types is included in a plinth.
        \end{theorem}
        Proof: Alain Frisch thesis, theorem 3.8 (p64) [TODO: ref].

        \begin{theorem}[$\Ga {}$ computation]
          $\Genv \tenv \Gamma=\fixpoint_\Gamma (\Gaa \tenv)$ can be reached in a finite number of iterations.
        \end{theorem}

    \section{Proofs}

    \subsection{Monotonicity}
    
      TODO: update

      We proceed by induction on $e$. We suppose that, for any $e'$ strict subexpression of $e$, for any $\Gamma' \leq \Gamma$,
      \begin{align*}
        &\Gamma \vdash e':t' \Rightarrow \Gamma' \vdash e':t'' \text{ with } t'' \leq t'\\
        &\Gamma \vdash e':t' \Rightarrow \forall p \forall t'' \forall \varpi.\ \Gp p {\Gamma',e',t''} (\varpi) \leq \Gp p {\Gamma,e',t''} (\varpi)
      \end{align*}
  
      Let $\tenv$, $\Gamma$, $\Gamma'$ and $e$ as in the lemma statement.

      First, let's show that $\Gamma' \vdash e:t' \text{ with } t' \leq t$.
  
      If $e\in\dom\Gamma$, we know that $\Gamma(e)=t$ (as $\Gamma \vdash e:t$). The definition of $\Gamma' \leq \Gamma$ suffices to conclude.
      So let's suppose $e\not\in\dom\Gamma$ (in particular, we know that $e$ is not a variable and that the last rule applied to type $\Gamma \vdash e:t$ is not $\Rule{Occ}$).
  
      We can also suppose $e\not\in\dom{\Gamma'}$. If it is not the case, we can just consider $\Gamma'\setminus(e:\Gamma'(e))$ instead of $\Gamma'$
      and obtain the wanted result on $\Gamma'$ by using its well-formedness.

      We now proceed by case analysis on $e$:
      \begin{description}
        \item[$c$] Trivial.
        \item[$x$] Impossible case, as we have already treated the rule $\Rule{Occ}$.
        \item[$\lambda^{\bigwedge_{i\in I} \arrow {s_i} {t_i}}x.e_x$] We know that the last rule in the typing derivation of $\Gamma \vdash e:t$ is $\Rule {Abs}$.
        Let's build a derivation for $\Gamma' \vdash e:t'$ (with $t'\leq t$) with $\Rule {Abs}$ as last rule.
        For that, we just have to notice that $\forall i.\ \Gamma',x:s_i \leq \Gamma,x:s_i$ and we apply the induction hypothesis.
        Note that if $x \in \dom \Gamma$ or $x \in \dom {\Gamma'}$, $\Gamma',x:s_i$ or $\Gamma,x:s_i$ could be not well-formed.
        In this case, we must alpha-rename the current lambda-abstraction.

        \item[$e_1\ e_2$] We know that the last rule in the typing derivation of $\Gamma \vdash e:t$ is $\Rule {App}$.
        We can build a derivation for $\Gamma' \vdash e:t'$ (with $t'\leq t$) with $\Rule {App}$ as last rule.
        For that, we just have to apply the induction hypothesis and notice that the $\circ$ operator is monotonic.
        \item[$\ite {e_0} {t_{if}} {e_1} {e_2}$]
        We know that the last rule in the typing derivation of $\Gamma \vdash e:t$ is $\Rule {If}$.

        Let's build a derivation for $\Gamma' \vdash e:t'$ (with $t'\leq t$) with $\Rule {If}$ as last rule.

        By the $\Ga {}$ existence theorem, we know that $\Genv {\cons {(e_0,t_{if},p)} \tenv} {\Gamma'}$ is defined and well-formed.

        We can deduce that $\Genv {\cons {(e_0,t_{if},p)} \tenv} {\Gamma'} \leq \Genv {\cons {(e_0,t_{if},p)} \tenv} \Gamma$.

        Again we can conclude by applying the induction hypothesis.\\
      \end{description}
  
      Now, let's show that $\forall p \forall t' \forall \varpi.\ \Gp p {\Gamma',e,t'} (\varpi) \leq \Gp p {\Gamma,e,t'} (\varpi)$.
  
      Let $p\in \{+,-\}$ and $t'$ a type.
      We proceed by induction on $\varpi$.
      
      The base case is straightforward:\\
      we have $\tyof {\occ e \varpi} {\Gamma'} \leq \tyof {\occ e \varpi} {\Gamma}$ (we just proved it)
      and similarly $\tyof {\occ e \varpi} {\Gamma'\setminus\{\occ e \varpi\}} \leq \tyof {\occ e \varpi} {\Gamma\setminus\{\occ e \varpi\}}$.\\
      For the case $\varpi.1$, we apply the outer induction hypothesis to get $\tyof {\Gamma'} {\occ e {\varpi.0}} \leq \tyof {\Gamma} {\occ e {\varpi.0}}$.
      We apply the induction hypothesis to get $\Gp p {\Gamma',e,t'} (\varpi) \leq \Gp p {\Gamma,e,t'} (\varpi)$. We can then conclude using the monotonicity of $\worra {} {}$.\\
      The case $\varpi.0$ follows from the previous case and the induction hypothesis.
  
      \qed

    \subsection{$\Ga {}$ computation}

    Let $\Gamma$ (well-formed) and $\tenv$ (well-typed).

    Let's prove that we can reach $\Genv \tenv \Gamma=\fixpoint_\Gamma (\Gaa \tenv)$ by applying successively the function $\Gaa \tenv$ a finite number of times.

    We will show that, starting from $\Gamma$, successive applications of $\Gaa \tenv$ stay in a finite set of environments.

    Let's consider a finite set of types $T$ such that:
    \begin{itemize}
      \item $\forall (e,t) \in \tenv$, $t \in T$
      \item $\forall (e,t) \in \tenv\ \forall \varpi$, if $\occ e \varpi$ is a lambda abstraction $\lambda^tx.e_x$, then $t \in T$
      \item $\forall (e,t) \in \tenv\ \forall \varpi$, if $\occ e \varpi$ is a constant $c$, then $\basic{c} \in T$
      \item $\forall (e,t) \in \tenv\ \forall \varpi$, if $\occ e \varpi \in \dom \Gamma$, then $\Gamma(\occ e \varpi) \in T$
    \end{itemize}

    Using the previous theorem, let $S$ by a plinth such that $T \subseteq S$. In particular we know that $S$ is finite.

    We define the "product context" of a type $t$ by:
    \[
      \texttt{pctx}(t)=\left\{
        \begin{array}{ll}
          \pair {\texttt{pctx}(\bpl t)} {\texttt{pctx}(\bpr t)} & \text{ if } t \leq \pair \Any \Any\\
          {[}] & \text{ otherwise}
        \end{array}  
      \right.
    \]

    Let's prove the following invariant. After any number of iteration, the resulting environment $\Gamma'$ is such that:\\
    $\forall (e,t) \in \tenv\ \forall \varpi$, if $\occ e \varpi \in \dom {\Gamma'}$, then $\Gamma'(\occ e \varpi)$ has a DNF with only atoms of the form
    $\texttt{pctx}(\tyof {\occ e \varpi} \Gamma)[\vec s]$ (with any vector $\vec s$ of elements of $S$).
    Note that there is finitely many such atoms.
    
    We prove it by induction on the number of iterations.

    The base case (0 iteration) is trivial.

    For the inductive case, let's take $\Gamma'$ (well-formed) that satifies the induction hypothesis
    and show that $\Gamma'' = \Gaux \tenv {\Gamma'}$ also satisfies the induction hypothesis.

    Let's fix $(e,t)$ and $\varpi$.\\
    We need to show that there exists $\vec s$ a vector of elements of $S$ such that $\Gp p {\Gamma'',e,t} (\varpi) = \texttt{pctx}(\tyof {\occ e \varpi} \Gamma)[\vec s]$.
    
    We proceed by induction on $\varpi$.

    Before treating the base case and the inductive case, we have to show that both $\tyof {\occ e \varpi} {\Gamma'}$
    and $\tyof {\occ e \varpi} {\Gamma'\setminus\{\occ e \varpi\}}$ (if relevant) have DNF with only atoms of the form
    $\texttt{pctx}(\tyof {\occ e \varpi} \Gamma)[\vec s]$ (with any vector $\vec s$ of elements of $S$).

    It can be easily proven by induction on the derivation.\\
    If the last rule is $\Rule{App}$, we proceed by induction and by noticing that plinths are stable by $\circ$.\\
    Other rules are trivial.

    Now, the base case of our induction (for $\varpi=\epsilon$) becomes trivial.

    Here are the inductive cases:
    \begin{description}
      \item[$\varpi.0$] Although a new arrow type is created, it is a negative one and so it will disappear after normalisation.
      \item[$\varpi.1$] We just have to notice that plinths are stable by $\worra {} {}$.
      \item[$\varpi.l$ and $\varpi.r$] Trivial (as $\Gamma$ is well-formed, we have $\texttt{pctx}(\occ e {\varpi}) = \pair {\texttt{pctx}(\occ e {\varpi.l})} {\dots}$).
      \item[$\varpi.f$ and $\varpi.s$] Trivial (as $\Gamma$ is well-formed, we have $\texttt{pctx}(\occ e {\varpi.f}) = \pair {\texttt{pctx}(\occ e {\varpi})} {\dots}$).
    \end{description}

    It concludes the induction.

    So we can deduce that there is finitely many possible environements that can be produced by the successive applications of $\Gaa \tenv$.
    As $\Gaa \tenv$ is reductive, we always reach a fixpoint in a finite number of iterations.
    By construction, this fixpoint is the greatest (as $\Gaa \tenv$ is reductive and monotonic).

\end{document}