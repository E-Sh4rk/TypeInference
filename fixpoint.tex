\documentclass[a4paper]{article}

\usepackage{setup}

\hypersetup{pdfstartview=XYZ}%         zoom par defaut

\setlength{\droptitle}{-5em}   % This is your set screw
\title{\vspace{1.5cm}Type inference - M2 Internship}
\author{Mickael LAURENT}
\date{\vspace{-5ex}}

\pagenumbering{gobble}

\theoremstyle{definition}
\newtheorem{theorem}{Theorem}
\newtheorem{lemma}{Lemma}
\newtheorem{definition}{Definition}
\newtheorem{property}{Property}
\newtheorem{corollary}{Corollary}

\begin{document}

    \maketitle

    \section{Type refinement system}

    \[
    \begin{array}{lcl}
      \typep+\epsilon{\Gamma,e,t} & = & t\\
      \typep-\epsilon{\Gamma,e,t} & = & \neg t\\
      \typep{p}{\varpi.0}{\Gamma,e,t} & = & \neg(\arrow{\Gp p{\Gamma,e,t}{(\varpi.1)}}{\neg \Gp p {\Gamma,e,t} (\varpi)})\\
      \typep{p}{\varpi.1}{\Gamma,e,t} & = & \worra{\tsrep {\tyof{\occ e{\varpi.0}}\Gamma}}{\Gp p {\Gamma,e,t} (\varpi)}\\
      \typep{p}{\varpi.l}{\Gamma,e,t} & = & \bpl{\Gp p {\Gamma,e,t} (\varpi)}\\
      \typep{p}{\varpi.r}{\Gamma,e,t} & = & \bpr{\Gp p {\Gamma,e,t} (\varpi)}\\
      \typep{p}{\varpi.f}{\Gamma,e,t} & = & \pair{\Gp p {\Gamma,e,t} (\varpi)}\Any\\
      \typep{p}{\varpi.s}{\Gamma,e,t} & = & \pair\Any{\Gp p {\Gamma,e,t} (\varpi)}\\ \\
      \Gp p {\Gamma,e,t} (\varpi) & = & \typep p \varpi {\Gamma,e,t} \land \tsrep {\tyof {\occ e \varpi} \Gamma}\\
      %\underbrace{\land \tyof {\occ e \varpi} {\Gamma\setminus\{\occ e \varpi\}}}_{\text{if $\occ e \varpi$ is not a variable}}}\\
    \end{array}
    \]

    \begin{align*}
      &(\Refine p {e,t} \Gamma)(e') = 
        \left\{\begin{array}{ll}
          \tyof {e'} \Gamma \tsand \bigwedge_{\{\varpi \alt \occ e \varpi \equiv e'\}} \Gp p {\Gamma,e,t} (\varpi) & \text{if } \exists \varpi.\ \occ e \varpi \equiv e' \\
          \Gamma(e') & \text{otherwise, if } e' \in \dom \Gamma\\
          \text{undefined} & \text{otherwise}
        \end{array}\right.\\&\\
      &\Gaux {\varnothing} \Gamma = \Gamma\\
      &\Gaux {\cons {(e,t,p)} \tenv} \Gamma = \Gaux \tenv {\Refine p {e,t} \Gamma}\\&\\
      &\Genv \tenv \Gamma=\fixpoint_\Gamma (\Gaa \tenv)\qquad \text{(if defined)}
    \end{align*}

    \begin{align*}
      \begin{array}{llclr}
      \text{If } e \in \dom \Gamma \text{, } & \tyof x \Gamma &=& \Gamma(x)\\
      & \tyof e \Gamma &=& \Gamma(e) \tsand \tyof e {\Gamma\setminus\{e\}}\\
      \text{Otherwise,} & \tyof c \Gamma &=& \basic{c}\\
      &\tyof {\lambda^{\bigwedge_{i\in I} \arrow {s_i} {t_i}}x.e} \Gamma &=& \tsfun {\arrow {s_i} {t_i}}_{i\in I}\\
      &\tyof {\aite e t {e_1} {e_2} \ts} \Gamma &=& \ts\\
      &\tyof {{e_1} {e_2}} \Gamma &=& \apply {\tyof {e_1} \Gamma} {\tyof {e_2} \Gamma} & \text{(if defined)}\\
      &\tyof {\pi_i e} \Gamma &=& \bpi_{\mathbf{i}}(\tyof e \Gamma) & \text{(if defined)}\\
      &\tyof {(e_1,e_2)} \Gamma &=& \tyof {e_1} \Gamma \tstimes \tyof {e_2} \Gamma\\%\pair{\tyof {e_1} \Gamma}{\tyof {e_2} \Gamma}
      &\tyof {e} \Gamma &=& \tsempty & \text{(if other rules do not apply)}
    \end{array}
    \end{align*}

    \comment{
      \begin{mathpar}

        \Infer[Occ]
            {
            }
            { \Gamma \tvdash e: \Gamma(e) }
            { e\in\dom\Gamma}
        \qquad
        \Infer[Const]
            { }
            {\Gamma\tvdash c:\basic{c}}
            {c\not\in\dom\Gamma}
        \qquad
        \Infer[Abs]
            { }
            {
            \Gamma\tvdash\lambda^{t}x.e: \tsfun t
            }
            {\lambda^{t}x.e\not\in\dom\Gamma}
            \\
        \Infer[App]
            {
              \Gamma \tvdash e_1: t_1\\
              \Gamma \tvdash e_2: t_2\\
              t_1 \leq \arrow \Empty \Any\\
              t_2 \leq \dom {t_1}
            }
            { \Gamma \tvdash {e_1}{e_2}: t_1 \circ t_2 }
            { {e_1}{e_2}\not\in\dom\Gamma }
            \\
        \Infer[Proj]
        {\Gamma \tvdash e:t\and t\leq\pair\Any\Any}
        {\Gamma \tvdash \pi_i e:\bpi_{\mathbf{i}}(t)}
        {\pi_i e\not\in\dom\Gamma}

        \Infer[Pair]
        {\Gamma \tvdash e_1:t_1 \and \Gamma \tvdash e_2:t_2}
        {\Gamma \tvdash (e_1,e_2):\pair{t_1}{t_2}}
        {(e_1,e_2)\not\in\dom\Gamma}
    \end{mathpar}
    }
  
    \section{Definitions}
      
    \begin{definition}[Well-formed environment]
      An environment $\Gamma$ is well-formed iff:
      \begin{align*}
        &\forall e \in \dom \Gamma \text{ not a variable. } \tsint{\tyof {e} {\Gamma\setminus\{e\}}} \neq \varnothing \text{ and } \Gamma(e) \leq \tyof {e} {\Gamma\setminus\{e\}}
      \end{align*}
    \end{definition}
  
    \begin{definition}[Environment inclusion]
      Let $\Gamma$ and $\Gamma'$ two well-formed environments. We says that $\Gamma' \leq \Gamma$ iff:
      \begin{align*}
        &\forall e \in \dom \Gamma.\ \tyof e {\Gamma'} \leq \Gamma(e)
      \end{align*}
    \end{definition}
    
    \begin{definition}[Plinth]
      A plinth $S$ is a set of types such that:
      \begin{itemize}
        \item $S$ is finite.
        \item $\Empty \in S$ and $S$ is stable by $\neg, \land, \vee$. 
        \item If $\tau$ is a DNF of $t\in S$, let's introduce $A=\{t_a \alt \exists (P,N) \in \tau \text{ s.t. } t_a \in P \text{ or } t_a \in N \}$. Thus:
        \begin{itemize}
          \item For all $t_a \in A$, we have $t_a \in S$.
          \item For all $t_1$, $t_2$ such that $\arrow {t_1} {t_2} \in A$ or $\pair {t_1} {t_2} \in A$,
          we have $t_1 \in S$ and $t_2 \in S$.
        \end{itemize}
      \end{itemize}
    \end{definition}

    \section{Theorems}

        \begin{lemma}[$\Ga {} $ well-formedness]
          Let $\Gamma$ a well-formed environment. Let $\tenv$ a set of tests.\\
          If $\Genv \tenv \Gamma$ is defined, then it is well-formed.
        \end{lemma}

        \begin{lemma}[$\tyof {\_} {}$ monotonicity]
          Let $\Gamma$ and $\Gamma'$ two well-formed environments such that $\Gamma' \leq \Gamma$.
          Let $e$ an expression.\\
          We have $\tyof e {\Gamma'} \leq \tyof e \Gamma$ and
          $\tsrep {\tyof e {\Gamma'}} \leq \tsrep {\tyof e {\Gamma}}$ (if defined).
        \end{lemma}

        \begin{corollary}[Relation order]
          The relation $\leq$ on environements is a preorder.
        \end{corollary}

        \begin{lemma}[$\Gaa {} $ existence]
          Let $\Gamma$ a well-formed environment.
          Let $\tenv$ a set of tests such that for each test $(e,t,p)$, $\tsint {\tyof e \Gamma} \neq \varnothing$
          (we say that these tests are well-typed by $\Gamma$).\\
          Then $\Gaux \tenv \Gamma$ is well-defined.
        \end{lemma}

        \begin{lemma}[$\Gaa {}$ monotonicity]
          Let $\Gamma$ and $\Gamma'$ two well-formed environments such that $\Gamma' \leq \Gamma$.
          Let $\tenv$ a set of tests well-typed by $\Gamma$. Then, $\Gaux \tenv {\Gamma'} \leq \Gaux \tenv \Gamma$.
        \end{lemma}

        \begin{theorem}[$\Ga {}$ existence]
            If $\Gamma$ is well-formed and $\tenv$ is well-typed by $\Gamma$, then $\Genv \tenv \Gamma$ is well defined and well-formed.
        \end{theorem}

        \begin{lemma}
          Every finite set of types is included in a plinth.
        \end{lemma}
        Proof: Alain Frisch thesis, theorem 3.8 (p64) [TODO: ref].

        \begin{lemma}
          Let $t$ a type. There is finitely many different type schemes $\ts$ such that $\tsrep \ts = t$ (up to equivalence).
        \end{lemma}

        \begin{theorem}[$\Ga {}$ computation]
          $\Genv \tenv \Gamma=\fixpoint_\Gamma (\Gaa \tenv)$ can be computed.
          In particular, the fixpoint can be reached in a finite number of iterations.
        \end{theorem}

    \section{Proofs}

    \subsection{$\tyof {\_} {}$ monotonicity}
    
    TODO

    \subsection{$\Gaa {}$ monotonicity}

      Let $\tenv$, $\Gamma$, $\Gamma'$ as in the lemma statement.

      We have to prove that for all relevant $p$, $e$, $t$ and $\varpi$,
      we have $\tyof {\occ e \varpi} {\Gamma'} \tsand \Gp p {\Gamma',e,t} (\varpi) \leq \tyof {\occ e \varpi} {\Gamma} \tsand \Gp p {\Gamma,e,t} (\varpi)$.

      We proceed by induction on $\varpi$. 

      By monotonicity of $\tyof {\_} {}$, we get that
      $\tyof {\occ e \varpi} {\Gamma'} \leq \tyof {\occ e \varpi} {\Gamma}$
      and $\tsrep {\tyof {\occ e \varpi} {\Gamma'}} \leq \tsrep {\tyof {\occ e \varpi} {\Gamma}}$.

      We proceed by case analysis on $\varpi$:
      \begin{description}
        \item[$\epsilon$] Trivial.
        \item[$\varpi.0$] We use the induction hypothesis and the contravariance/covariance of $\arrow {} {}$.
        \item[$\varpi.1$] We use the induction hypothesis.\\
        First, we must show that the $\worra {} {}$ operator is monotonic for its second argument.
        It is quite trivial given its declarative definition.

        Secondly, we must show that for any function types $f_1 \leq f_2$, and for any type $t'$,
        we have $(\worra {f_1} {t'}) \land \dom {f_2} \leq \worra {f_2} {t'}$. By the absurd, let's suppose it is not true.
        Let's note $t'' = (\worra {f_1} {t'}) \land \dom {f_2}$.
        Then we have $t'' \leq \dom {f_2} \leq \dom {f_1}$ and $f_2 \leq \arrow {t''} {t'}$ and
        $f_1 \not\leq \arrow {t''} {t'}$, which contradicts $f_1 \leq f_2$.

        From these two properties, we can conclude, as $\tyof {\occ e {\varpi.1}} {\Gamma'} \leq \dom {\tsrep {\tyof{\occ e{\varpi.0}}\Gamma}}$
        (it follows from the fact that tests are well-typed by $\Gamma$).
        \item[$\varpi.l$ or $\varpi.r$] We use the induction hypothesis and the monotonicity of $\bpi_{\mathbf{i}}$.
        \item[$\varpi.f$ or $\varpi.s$] We use the induction hypothesis and the covariance of $\pair {} {}$.
      \end{description}
    
      \qed

    \subsection{$\Ga {}$ computation}

    Let $\Gamma$ (well-formed) and $\tenv$ (well-typed).

    Let's prove that we can reach $\Genv \tenv \Gamma=\fixpoint_\Gamma (\Gaa \tenv)$ by applying successively the function $\Gaa \tenv$ a finite number of times.

    We will prove that, starting from $\Gamma$, successive applications of $\Gaa \tenv$ stay in a finite set of environments (modulo equivalence).

    Let's consider a finite set of types $T$ such that:
    \begin{itemize}
      \item $\forall (e,t,p) \in \tenv$, $t \in T$
      \item $\forall (e,t,p) \in \tenv\ \forall \varpi$, if $\occ e \varpi$ is a lambda abstraction or an annoted $\texttt{ite}$ of type $\ts$, then $\tsrep{\ts} \in T$
      \item $\forall (e,t,p) \in \tenv\ \forall \varpi$, if $\occ e \varpi$ is a constant $c$, then $\basic{c} \in T$
      \item $\forall (e,t,p) \in \tenv\ \forall \varpi$, if $\occ e \varpi \in \dom \Gamma$, then $\tsrep {\Gamma(\occ e \varpi)} \in T$
    \end{itemize}

    Using the previous theorem, let $S$ by a plinth such that $T \subseteq S$. In particular we know that $S$ is finite.

    \newcommand{\struct}[0]{\mathcal{S}}
    \newcommand{\pathstr}[2]{\texttt{path\_struct}_{#1}(#2)}
    \newcommand{\structof}[2]{\texttt{structof}_{#1}(#2)}
    \newcommand{\refinestr}[2]{\texttt{RefineStr}_{#1}(#2)}
    \newcommand{\auxstr}[1]{\texttt{AuxStr}_{#1}}
    \newcommand{\structs}[2]{\texttt{Struct}_{#1}(#2)}

    -----

    We call \textit{structural environment} a partial function from variables, constants and lambda abstractions to sets of \textit{structural contexts}.
    For each structural environment $\struct$ and expression $e$, we define $\structof \struct e$ by:
    \[
      \left\{
        \begin{array}{lcl}
          \structof{\struct}{c} & = & \struct(c) \\
          \structof{\struct}{x} & = & \struct(x) \\
          \structof{\struct}{\lambda^tx.e} & = & \struct(\lambda^tx.e) \\
          \structof{\struct}{(e_1,e_2)} & = & \bigcup_{s_1\in\structof{\struct}{e_1}}\bigcup_{s_2\in\structof{\struct}{e_2}}  \{ \pair {s_1} {s_2} \} \\
          \structof{\struct}{e_1 e_2} & = & \bigcup_{\arrow s t\in\structof{\struct}{e_1}} \{ t \} \\
          \structof{\struct}{\bpi_i e} & = & \bigcup_{\pair {t_1} {t_2}\in\structof{\struct}{e}} \{ t_i \}
        \end{array}  
      \right.
    \]

    We define the \textit{structures} of a path $\varpi$ by:
    \[
      \left\{
        \begin{array}{lcl}
          \pathstr{\struct,e}{\epsilon} & = & \structof{\struct}{e} \\
          \pathstr{\struct,e}{\varpi.0} & = & \bigcup_{s\in\pathstr{\struct,e}{\varpi.1}} \bigcup_{t\in\pathstr{\struct,e}{\varpi}} \{\arrow {s} {t}\} \\
          \pathstr{\struct,e}{\varpi.1} & = & \bigcup_{\arrow s t\in\structof{\struct}{\occ e {\varpi.0}}} \{s\}\\
          \pathstr{\struct,e}{\varpi.l} & = & \bigcup_{\pair {s_1} {s_2}\in\pathstr{\struct,e}{\varpi}} \{s_1\} \\
          \pathstr{\struct,e}{\varpi.r} & = & \bigcup_{\pair {s_1} {s_2}\in\pathstr{\struct,e}{\varpi}} \{s_2\} \\
          \pathstr{\struct,e}{\varpi.f} & = & \bigcup_{s_1\in\pathstr{\struct,e}{\varpi}} \{\pair {{[}]} {s_2}\} \\
          \pathstr{\struct,e}{\varpi.s} & = & \bigcup_{s_2\in\pathstr{\struct,e}{\varpi}} \{\pair {s_1} {{[}]}\}
        \end{array}  
      \right.
    \]

    \begin{align*}
      &(\refinestr e \struct)(e') = 
        \left\{\begin{array}{ll}
          \text{undefined} & \text{if } e' \text{ not a variable, constant or $\lambda$-abstraction}\\
          \struct(e') \cup \bigcup_{\{\varpi \alt \occ e \varpi \equiv e'\}} \pathstr{\struct,e}{\varpi} & \text{otherwise, if } e' \in \dom \struct\\
          \text{undefined} & \text{otherwise}
        \end{array}\right.\\&\\
      &\auxstr {\varnothing} (\struct) = \struct\\
      &\auxstr {\cons {(e,t,p)} \tenv} \struct = \auxstr \tenv (\refinestr e \struct)\\&\\
      &\structs \tenv \struct=\textsf{lfp}_\struct (\auxstr \tenv)
    \end{align*}

    Let $\struct_I$ be the structural environment that maps every occurence of variable, constant or $\lambda$-abstraction
    in $\tenv$ to $[]$.

    $\struct = \structs \tenv {\struct_I}$ is well-defined as $\auxstr \tenv$ is monotonic (and even extensive).
    Moreover, for any $e\in\dom\struct$, $\struct(e)$ is finite. TODO

    -----

    Let's prove the following invariant. After any number of iteration, the resulting environment $\Gamma'$ is such that:\\
    for any occurence $e'$ in $\tenv$ such that $e' \in \dom {\Gamma'}$, $\tsrep {\Gamma'(e')}$ can be written with a DNF with only atoms
    of the form $\text{ctx}[\vec s]$ with $\vec s$ a vector of elements of $S$ and $\text{ctx}$ an element of $\structof{\struct}{e'}$.
    Note that there is finitely many such DNF.
        
    We prove it by induction on the number of iterations.

    The base case (0 iteration) is trivial.

    For the inductive case, let's take $\Gamma'$ (well-formed) that satifies the induction hypothesis
    and show that $\Gamma'' = \Gaux \tenv {\Gamma'}$ also satisfies the induction hypothesis.

    Let's fix $(e,t,p)$ and $\varpi$.\\
    We need to show that $\Gp p {\Gamma'',e,t} (\varpi)$ has a DNF as described above.    
    We proceed by induction on $\varpi$.

    Before treating the base case and the inductive case, we must show that $\tsrep {\tyof {\occ e \varpi} {\Gamma'}}$
    has the required form.

    It can be easily proven by induction on the derivation (all cases are quite straightfoward given the form of $\structof {} {\_}$).

    Now, the base case of our induction (for $\varpi=\epsilon$) is trivial.

    Here are the inductive cases:
    \begin{description}
      \item[$\varpi.1$] TODO
      \item[$\varpi.0$] TODO
      \item[$\varpi.l$ and $\varpi.r$] TODO
      \item[$\varpi.f$ and $\varpi.s$] TODO
    \end{description}

    It concludes the induction.

    From this result, we can deduce that there is finitely many possible environements that can be produced by the successive applications of $\Gaa \tenv$.
    Indeed, for any relevant expression $e$, there is only finitely many different $\tsrep {\Gamma(e)}$ in the successive environments $\Gamma$ (up to equivalence).
    Using the previous lemma, we deduce that there is finitely many different $\Gamma(e)$ and so finitely many different environments.

    As $\Gaa \tenv$ is reductive, we always reach a fixpoint in a finite number of iterations.
    By construction, this fixpoint is the greatest (as $\Gaa \tenv$ is reductive and monotonic).

    Thus, in order to compute $\fixpoint_\Gamma (\Gaa \tenv)$ we can successively call $\Gaa \tenv$ until we reach a fixpoint.
    In order to know when we have reached a fixpoint, we just have to check whether, for all $(e,t,p) \in \tenv$, for all $e'$ such that
    $\exists \varpi. \occ e \varpi \equiv e'$, we have $\bigwedge_{\{\varpi \alt \occ e \varpi \equiv e'\}} \Gp p {\Gamma,e,t} (\varpi) \in \tsint {\tyof {e'} \Gamma}$.

\end{document}